\section{Jordan-Normalform}

Im Folgenden sei $f \colon V \to V$ ein Endomorphismus eines endlichdimensionalen $K$-Vek\-tor\-raums $V$.





\subsection{Definition}

\begin{definition}
  Für alle $n \in \Natural$ und $\lambda \in K$ ist
  \[
              J_n(\lambda)
    \coloneqq \begin{pmatrix}
                \lambda &         &         &         \\
                1       & \ddots  &         &         \\
                        & \ddots  & \ddots  &         \\
                        &         & 1       & \lambda
              \end{pmatrix}
    \in       \matrices{n}{K}
  \]
  der \emph{Jordanblock} zu $\lambda$ von Größe $n$.
\end{definition}

\begin{definition}
  Eine Matrix $J$ der Form
  \[
      J
    = \begin{pmatrix}
        J_{n_1}(\lambda_1)  &         &                     \\
                            & \ddots  &                     \\
                            &         & J_{n_t}(\lambda_t)
      \end{pmatrix}
  \]
  ist in \emph{Jordan-Normalform}.
\end{definition}

\begin{definition}
  Eine \emph{Jordan-Normalform} einer Matrix $A \in \matrices{n}{K}$ ist eine zu $A$ ähnliche Matrix $J \in \matrices{n}{K}$, so dass $J$ in Jordan-Normalform ist.
  
  Eine Jordan-Normalform von $f$ ist eine Jordan-Normalform der darstellenden Matrix $\repmatrixendo{f}{\basis{B}}$ bezüglich einer Basis $\basis{B}$ von $V$.
\end{definition}

Der Endomorphismus $f$ besitzt genau dann eine Jordan-Normalform $J \in \matrices{n}{K}$, falls es eine Basis $\basis{B}$ von $V$ gibt, so dass $\repmatrixendo{f}{\basis{B}} = J$ gilt.
Wir bezeichnen eine solche Basis als \emph{Jordanbasis} von $f$.

Eine \emph{Jordanbasis} $\basis{B} = (v_1, \dotsc, v_n)$ einer Matrix $A \in \matrices{n}{K}$ ist eine Jordanbasis der zu $A$ (bezüglich der Standardbasis) gehörigen linearen Abbildung $f_A \colon K^n \to K^n$, $x \mapsto Ax$.
Dies ist äquivalent dazu, dass für die Matrix $C = (v_1 | \cdots | v_n) \in \GL{n}{K}$ die Matrix $S^{-1} A S = \repmatrixendo{f_A}{\basis{B}}$ in Jordan-Normalform ist.





\subsection{Eindeutigkeit}

Es sei $J$ eine Matrix in Jordan-Normalform, also
\[
    J
  = \begin{pmatrix}
      J_{n_1}(\lambda_1)  &         &                     \\
                          & \ddots  &                     \\
                          &         & J_{n_t}(\lambda_t)
    \end{pmatrix}.
\]
Für alle $\lambda \in K$ gilt dann
\[
    \dim \ker (J - \lambda \Id)^k
  = \sum_{k'=1}^k \text{Anzahl der Jordanblöcke zu $\lambda$ von Größe $\geq k'$}.
\]
Für die Zahlen $d_k(\lambda) \coloneqq \dim \ker (J - \lambda \Id)^k$ gilt deshalb
\begin{gather*}
    d_k(\lambda) - d_{k-1}(\lambda)
  = \text{Anzahl der Jordanblöcke zu $\lambda$ von Größe $\geq k$}
\shortintertext{und somit}
  \begin{aligned}
     &\,    \text{Anzahl der Jordanblöcke zu $\lambda$ von Größe $k$}           \\
    =&\,    \text{Anzahl der Jordanblöcke zu $\lambda$ von Größe $\geq k$}      \\
     &\,  - \text{Anzahl der Jordanblöcke zu $\lambda$ von Größe $\geq (k+1)$}  \\
    =&\,    ( d_k(\lambda) - d_{k-1}(\lambda) )
          - ( d_{k+1}(\lambda) - d_k(\lambda) )                                 \\
    =&\,  2 d_k(\lambda) - d_{k-1}(\lambda) - d_{k+1}(\lambda).
  \end{aligned}
\end{gather*}

Ist $A \in \matrices{n}{K}$ und $\lambda \in K$ eine Jordan-Normalform von $A$, so sind $A$ und $J$ ähnlich, weshalb für alle $\lambda \in K$ und $k \geq 0$ auch $(A - \lambda \Id)^k$ und $(J - \lambda \Id)^k$ ähnlich sind.
Für alle $\lambda \in K$ und $k \geq 0$ gilt deshalb $\dim \ker (A - \lambda)^k = \dim \ker (J - \lambda)^k$.
Aus der obigen Berechnung ergibt sich deshalb für die Zahlen $d_k(\lambda) \coloneqq \ker (A - \lambda \Id)^k$, dass
\[
    \text{Anzahl der Jordanblöcke zu $\lambda$ von Größe $k$ in $J$}
  = 2 d_k(\lambda) - d_{k-1}(\lambda) - d_{k+1}(\lambda).
\]
Damit ergibt sich inbesondere die folgende Eindeutigkeit der Jordannormalform:

\begin{proposition}
  Je zwei Jordannormalformen einer Matrix, bzw.\ eines Endomorphismus stimmen bis auf Permutation der Jordanblöcke überein.
\end{proposition}

Es ergibt daher Sinn, von \emph{der} Jordannormalform einer Matrix, bzw.\ eines Endomorphismus zu sprechen.





\subsection{Existenz}

\begin{definition}
  Für alle $\lambda \in K$ und $k \geq 0$ sei
  \[
      \heigenspace{V}{f}{\lambda}{k}
    = \{ v \in V \suchthat (f - \lambda \id_V)^k(v) = 0 \}
    = \ker (f - \id_V)^k.
  \]
  Der Untervektorraum
  \[
              \geigenspace{V}{f}{\lambda}
    \coloneqq \bigcup_{k=0}^\infty \heigenspace{V}{f}{\lambda}{k}
    =         \left\{
                v \in V
              \suchthat
                \text{es gibt $k \geq 0$ mit $(f - \lambda \id_V)^k(v) = 0$}
              \right\}
  \]
  ist der \emph{verallgemeinerte Eigenraum} von $f$ zu $\lambda$.
  Für $A \in \matrices{n}{K}$ und alle $\lambda \in K$ und $k \geq 0$ sei
  \[
      \heigenspace{(K^n)}{A}{\lambda}{k}
    = \{ x \in K^n \suchthat (A - \lambda \Id)^k x = 0 \}
    = \ker (A - \Id)^k.
  \]
  Der Untervektorraum
  \[
              \geigenspace{(K^n)}{A}{\lambda}
    \coloneqq \bigcup_{k=0}^\infty \heigenspace{(K^n)}{A}{\lambda}{k}
    =         \left\{
                x \in K^n
              \suchthat
                \text{es gibt $k \geq 0$ mit $(A - \lambda \Id)^k x = 0$}
              \right\}
  \]
  ist der \emph{verallgemeinerte Eigenraum} von $A$ zu $\lambda$.
\end{definition}

\begin{lemma}
  \begin{enumerate}
    \item
      Es gilt genau dann $\geigenspace{V}{f}{\lambda} \neq 0$, wenn $\lambda$ ein Eigenwert von $f$ ist.
    \item
      Die Summe $\sum_{\lambda \in K} \geigenspace{V}{f}{\lambda}$ ist direkt.
  \end{enumerate}
\end{lemma}

Mithilfe der verallgemeinerten Eigenräume ergibt sich eine Charakterisierung der Existenz der Jordan-Normalform:

\begin{theorem}
  \label{theorem: existence of generalized eigespace decomposition and jordan normal form}
  Die folgenden Bedingungen äquivalent:
  \begin{enumerate}
    \item
      Das charakteristische Polynom $\charpol{f}(t)$ zerfällt in Linearfaktoren.
    \item
      Es gilt $V = \bigoplus_{\lambda \in K} \geigenspace{V}{f}{\lambda}$.
    \item
      Die Jordan-Normalform von $f$ existiert.
  \end{enumerate}
\end{theorem}

Ist $A \in \matrices{n}{K}$, so dass das charakteristische Polynom $\charpol{A}(t)$ in Linearfaktoren zerfällt, so lässt sich die Jordan-Normalform von $A$ sowie eine zugehörige Jordanbasis wie folgt berechnen:

\begin{itemize}
  \item
    Man bestimme die Eigenwerte von $A$, etwa indem man $\charpol{A}(t)$ berechnet und anschließend die Nullstellen herausfindet.
    
  \item
    Für jeden Eigenwert $\lambda$ von $A$ führe man die folgenden Schritte durch:
    \begin{itemize}
      \item
        Man berechne die iterierten Kerne $\ker (A - \lambda \Id), \ker (A - \lambda \Id)^2, \dotsc, \ker (A - \lambda \Id)^m$ bis zu dem Punkt, an dem eine der folgenden äquivalenten Bedingungen erfüllt sind:
        \begin{itemize}
          \item
            Die Dimension $\dim \ker (A - \lambda \Id)^m$ ist die algebraische Vielfachheit von $\lambda$ in $\charpol{A}(t)$.
          \item
            Es gilt $\ker (A - \lambda \Id)^m = \ker (A - \lambda \Id)^{m+1}$.
        \end{itemize}
      \item
        Man bestimme Anhand der Zahlen $d_k(\lambda) \coloneqq \dim \ker (A - \lambda \Id)^k$ die Anzahl der auftretenden Jordanblöcke zu $\lambda$ von Größe $k$ als
        \[
          b_k(\lambda) \coloneqq 2 d_k(\lambda) - d_{k-1}(\lambda) - d_{k+1}(\lambda).
        \]
    \end{itemize}
\end{itemize}

Aus den Eigenwerten $\lambda_1, \lambda_2, \dotsc$ von $A$ und den Zahlen $b_k(\lambda_i)$ erhalten wir bereits, wieviele Blöcke es zu welchen Eigenwert von welcher Größe gibt, d.h.\ wie die Jordannormalform von $A$ (bis auf Permutation der Blöcke) aussehen wird.
Inbesondere ist $d_1(\lambda)$ die Gesamtzahl der Jordanblöcke zu $\lambda$ und die entsprechende Potenz $m$ die maximal auftretende Blöckgröße zu $\lambda$.

Zur Berechnung einer Jordanbasis von $A$ geht man weiter wie folgt vor:

\begin{itemize}[resume]
  \item
    Für jeden Eigenwert $\lambda$ von $A$ gehe man weiterhin wie folgt vor:
    \begin{itemize}
      \item
        Man wähle linear unabhängige Vektoren $v_1, \dotsc, v_{b_m} \in \ker A^m$ mit
        \[
                    \ker A^m
          =         \ker A^{m-1}
            \oplus  \generated{v_1, \dotsc, v_{b_m}}.
        \]
        (Zur konkreten Berechnung ergänze man eine Basis $\ker A^{m-1}$ zu einer Basis von $\ker A^m$; dann kann man $v_1, \dotsc, v_{b_m}$ als die neu hinzugekommenen Basisvektoren wählen.)
      \item
        Hierdurch ergeben sich für $\basis{B}$ die ersten paar Basisvektoren
        \begin{align*}
          v_1,     A v_1,     &\dotsc, A^{m-1} v_1,     \\
          v_2,     A v_2,     &\dotsc, A^{m-1} v_2,     \\
                              &\dotsc,                  \\
          v_{b_m}, A v_{b_m}, &\dotsc, A^{m-1} v_{b_m}.
        \end{align*}
      \item
        Man wählt nun linear unabhängige Vektoren $v'_1, \dotsc, v'_{b_{m-1}} \in \ker A^{m-1}$, so dass
        \[
                    \ker A^{m-1}
          =         \ker A^{m-2}
            \oplus  \generated{ A v_1, \dotsc, A v_{b_m} }
            \oplus  \generated{ v'_1, \dotsc, v'_{b_{m-1}} }
        \]
        gilt.
      \item
        Hierdurch erhält man für $\basis{B}$ die weiteren Basisvektoren
        \begin{align*}
          v'_1,         A v'_1,         &\dotsc, A^{m-2} v'_1,          \\
          v'_2,         A v'_2,         &\dotsc, A^{m-2} v'_2,          \\
                                        &\dotsc,                        \\
          v'_{b_{m-1}}, A v'_{b_{m-1}}, &\dotsc, A^{m-2} v'_{b_{m-1}}.
        \end{align*}
      \item
        Man wähle nun $v''_1, \dotsc, v''_{b_{m-2}} \in \ker A^{m-2}$, so dass
        \begin{align*}
           &\,      \ker A^{m-1}  \\
          =&\,      \ker A^{m-2}
            \oplus  \generated{ A^2 v_1, \dotsc, A^2 v_{b_m} }
            \oplus  \generated{ A v'_1, \dotsc, A v'_{b_{m-1}} }
            \oplus  \generated{ v''_1, \dotsc, v''_{b_{m-2}} }
        \end{align*}
        gilt.
      \item
        Hiermit ergeben sich für $\basis{B}$ die Basisvektoren
        \begin{align*}
          v''_1,         A v''_1,         &\dotsc, A^{m-2} v''_1,         \\
          v''_2,         A v''_2,         &\dotsc, A^{m-2} v''_2,         \\
                                          &\dotsc,                        \\
          v''_{b_{m-2}}, A v''_{b_{m-2}}, &\dotsc, A^{m-2} v''_{b_{m-2}}.
        \end{align*}
    \end{itemize}
    Durch Weiterführen der obigen Schritte erhält man schließlich eine Basis $\basis{B}_\lambda$ von $\geigenspace{(K^n)}{A}{\lambda}$.
    
  \item
    Sind $\lambda_1, \dotsc, \lambda_n$ die paarweise verschiedenen Eigenwerte von $K^n$, so ergibt sich Zusammenfügen der Basen $\basis{B}_{\lambda_1}, \dotsc, \basis{B}_{\lambda_t}$ eine Basis $\basis{B}$ von $K^n$.
 
  \item
    Die Basis $\basis{B}$ ist eine Jordanbasis von $A$:
    Indem man die (in der obigen Reihenfolge entstandenen) Basisvektoren als Spalten in eine Matrix $C$ einträgt, erhält man schließlich $C \in \GL{n}{K}$, so dass $C^{-1} A C$ in Jordan-Normalform ist.
    Dabei sind die Blöcke zunächst nach den Eigenwerten $\lambda_1, \dotsc, \lambda_t$ (in dieser Reihenfolge) sortiert;
    die Blöcke zum gleichen Eigenwert sind nach absteigender Größe sortiert.
\end{itemize}

% TODO: Adding examples.



\subsection{Implizite Bestimmen der Jordan-Normaform}

Es sei $J \in \matrices{n}{K}$ eine Jordan-Normalform von $f$.
\begin{itemize}
  \item
    Für das charakteristische Polynom $\charpol{f}(t) = \charpol{J}(t) = (t - \lambda_1) \dotsm (t - \lambda_n)$ gelten
    \[
        \tr f
      = \tr J
      = \lambda_1 + \dotsb + \lambda_n
    \quad\text{und}\quad
        \det f
      = \det J
      = \lambda_1 \dotsm \lambda_n.
    \]
  \item
    Für das charakteristische Polynom $\charpol{f}(t) = \charpol{J}(t) = (t - \lambda_1)^{n_1} \dotsm (t - \lambda_s)^{n_s}$ mit $\lambda_i \neq \lambda_j$ für $i \neq j$ gilt
    \[
        n_i
      = \dim \geigenspace{V}{f}{\lambda}
      = \text{wie oft $\lambda_i$ auf der Diagonalen von $J$ steht}
    \]
  \item
    Es gilt
    \[
        \text{Anzahl der Jordanblöcke zu $0$ in $J$}
      = \dim \ker f
      = n - \rank f.
    \]
  \item
    Für das Minimalpolynom $\minpol{f}(t) = \minpol{J}(t) = (t - \lambda_1)^{m_1} \dotsm (t - \lambda_s)^{n_s}$ mit $\lambda_i \neq \lambda_j$ für $i \neq j$ gilt
    \[
        m_i
      = \text{maximale auftrettende Größe eines Jordanblockes zu $\lambda_i$ in $J$}
    \]
  \item
    Ist allgemeiner $q(t) \in K[t]$ ein Polymom mit $q(t) = (t - \lambda_1)^{m'_1} \dotsm (t - \lambda_s)^{m'_s}$, wobei $\lambda_i \neq \lambda_j$ für $i \neq j$, und $q(f) = 0$, so ergeben sich die folgenden beiden Restriktionen an $J$:
    \begin{enumerate}
      \item
        Jeder Eigenwert von $f$ kommt in $\lambda_1, \dotsc, \lambda_s$ vor (siehe Lemma~\ref{lemma: polynomial equations give restriction for the eigenvalues}),
      \item
        Es gilt $\minpol{f} \divides q$.
        Deshalb ist $\minpol{f} = (t - \lambda_1)^{m_1} \dotsm (t - \lambda_s)^{m_s}$ mit $m_i \leq m'_i$ für alle $i$.
        Also sind die Jordanblöcke zu $\lambda_i$ in $J$ jeweils höchstens $m'_i$ groß.
    \end{enumerate}
\end{itemize}

Für kleine Matrizen kann man hierdurch bereits Aussagen über die Jordan-Nor\-mal\-form treffen, ohne die Matrix selbst zu kennen.

% TODO: Beispiele hinzufügen.

% \subsubsection{Nilpotente Endomorphismen}
% \label{subsection: jordan normal form for nilpotent endomorphisms}
% 
% \begin{definition}
%   Eine Matrix $A \in \matrices{n}{K}$ ist nilpotent falls es $k \geq 0$ mit $A^k = 0$ gibt.
%   Der Endomorphismus $f$ ist nilpotent, falls es $k \geq 0$ mit $f^k = 0$ gibt.
% \end{definition}
% 
% \begin{lemma}
%   Die folgenden Bedingungen sind äquivalent:
%   \begin{enumerate}
%     \item
%       Der Endomorphismus $f$ ist nilpotent.
%     \item
%       Es gibt eine Basis $\basis{B}$ von $V$, so dass die darstellende Matrix $\repmatrixendo{f}{\basis{B}}$ nilpotent ist.
%     \item
%       Für jede Basis $\basis{B}$ von $V$ ist die darstellende Matrix $\repmatrixendo{f}{\basis{B}}$ nilpotent.
%   \end{enumerate}
% \end{lemma}
% 
% Eine Besonderheit nilpotente Endomorphismen und Matrizen besteht darin, dass diese immer eine Jordan-Normalform besitzen.
% Dies ergibt sich durch Angabe eines konkreten Algorithmus zur Berechnung einer Jordan-Basis einer nilpotenten Matrix $A$:
% 
% \begin{itemize}
%   \item
%     Man berechne die iterierten Kerne $\ker A, \ker A^2, \dotsc, \ker A^m$ bis zu dem Punkt, an dem $\ker A^m = K^n$ gilt.
%   \item
%     Man bestimme Anhand der Zahlen $d_k \coloneqq \dim \ker A^k$ die Anzahl der auftretenden Jordanblöcke von Größe $k$ als $b_k \coloneqq 2 d_k - d_{k-1} - d_{k+1}$.
% \end{itemize}
% 
% Aus den Zahlen $b_1, b_2, \dotsc$ erkennen wir bereits, wieviele Blöcke es zx welcher Größe gibt, d.h.\ wie die Jordannormalform von $A$ (bis auf Permutation der Blöcke) aussehen wird.
% Inbesondere ist $d_1$ die Gesamtzahl der Jordanblöcke und $m$ die maximal auftretende Blöckgröße.
% 
% Zur Berechnung einer Jordanbasis von $A$ geht man weiter wie folgt vor:
% 
% \begin{itemize}[resume]
%   \item
%     Man wähle linear unabhängige Vektoren $v_1, \dotsc, v_{b_m} \in \ker A^m$ mit
%     \[
%                 \ker A^m
%       =         \ker A^{m-1}
%         \oplus  \generated{v_1, \dotsc, v_{b_m}}.
%     \]
%     (Ergänzt man eine Basis von $\ker A^{m-1}$ zu einer Basis von $\ker A^m$, so sind $v_1, \dotsc, v_{b_m}$ die neu hinzugekommenen Basisvektoren.)
%   \item
%     Hierdurch ergeben sich für $\basis{B}$ die ersten paar Basisvektoren
%     \begin{align*}
%       v_1,     A v_1,     &\dotsc, A^{m-1} v_1,     \\
%       v_2,     A v_2,     &\dotsc, A^{m-1} v_2,     \\
%                           &\dotsc,                  \\
%       v_{b_m}, A v_{b_m}, &\dotsc, A^{m-1} v_{b_m}.
%     \end{align*}
%   \item
%     Man wählt nun linear unabhängige Vektoren $v'_1, \dotsc, v'_{b_{m-1}} \in \ker A^{m-1}$, so dass
%     \[
%                 \ker A^{m-1}
%       =         \ker A^{m-2}
%         \oplus  \generated{ A v_1, \dotsc, A v_{b_m} }
%         \oplus  \generated{ v'_1, \dotsc, v'_{b_{m-1}} }
%     \]
%     gilt.
%   \item
%     Hierdurch erhält man für $\basis{B}$ die weiteren Basisvektoren
%     \begin{align*}
%       v'_1,         A v'_1,         &\dotsc, A^{m-2} v'_1,          \\
%       v'_2,         A v'_2,         &\dotsc, A^{m-2} v'_2,          \\
%                                     &\dotsc,                        \\
%       v'_{b_{m-1}}, A v'_{b_{m-1}}, &\dotsc, A^{m-2} v'_{b_{m-1}}.
%     \end{align*}
%   \item
%     Man wähle nun $v''_1, \dotsc, v''_{b_{m-2}} \in \ker A^{m-2}$, so dass
%     \begin{align*}
%        &\,      \ker A^{m-1}  \\
%       =&\,      \ker A^{m-2}
%         \oplus  \generated{ A^2 v_1, \dotsc, A^2 v_{b_m} }
%         \oplus  \generated{ A v'_1, \dotsc, A v'_{b_{m-1}} }
%         \oplus  \generated{ v''_1, \dotsc, v''_{b_{m-2}} }
%     \end{align*}
%     gilt.
%   \item
%     Hiermit ergeben sich für $\basis{B}$ die Basisvektoren
%     \begin{align*}
%       v''_1,         A v''_1,         &\dotsc, A^{m-2} v''_1,         \\
%       v''_2,         A v''_2,         &\dotsc, A^{m-2} v''_2,         \\
%                                       &\dotsc,                        \\
%       v''_{b_{m-2}}, A v''_{b_{m-2}}, &\dotsc, A^{m-2} v''_{b_{m-2}}.
%     \end{align*}
% \end{itemize}
% 
% Indem man die obigen Schritte iteriert, erhält man schließlich die gewünschte Basis $\basis{B}$ von $K^n$.
% Indem man die (in der obigen Reihenfolge entstandenen) Basisvektoren als Spalten in eine Matrix $C$ einträgt, erhält man schließlich $C \in \GL{n}{K}$, so dass $C^{-1} A C$ in Jordan-Normalform ist;
% dabei sind die Jordanblöcke in absteigender Größe sortiert.
% 
% 
% 
% \subsubsection{Allgemeine Endomorphismen}
% 
% \begin{definition}
%   Für alle $\lambda \in K$ und $k \geq 0$ sei
%   \[
%       \heigenspace{V}{f}{\lambda}{k}
%     = \{ v \in V \suchthat (f - \lambda \id_V)^k(v) = 0 \}
%     = \ker (f - \id_V)^k.
%   \]
%   Der Untervektorraum
%   \[
%               \geigenspace{V}{f}{\lambda}
%     \coloneqq \bigcup_{k=0}^\infty \heigenspace{V}{f}{\lambda}{k}
%     =         \left\{
%                 v \in V
%               \suchthat
%                 \text{es gibt $k \geq 0$ mit $(f - \lambda \id_V)^k(v) = 0$}
%               \right\}
%   \]
%   ist der \emph{verallgemeinerte Eigenraum} von $f$ zu $\lambda$.
%   Für alle $A \in \matrices{n}{K}$ und $k \geq 0$ sei
%   \[
%       \heigenspace{(K^n)}{A}{\lambda}{k}
%     = \{ x \in K^n \suchthat (A - \lambda \Id)^k x = 0 \}
%     = \ker (A - \Id)^k.
%   \]
%   Der Untervektorraum
%   \[
%               \geigenspace{(K^n)}{A}{\lambda}
%     \coloneqq \bigcup_{k=0}^\infty \heigenspace{(K^n)}{A}{\lambda}{k}
%     =         \left\{
%                 x \in K^n
%               \suchthat
%                 \text{es gibt $k \geq 0$ mit $(A - \lambda \Id)^k x = 0$}
%               \right\}
%   \]
%   ist der \emph{verallgemeinerte Eigenraum} von $A$ zu $\lambda$.
% \end{definition}
% 
% \begin{lemma}
%   \begin{enumerate}
%     \item
%       Für alle $\lambda \in K$ ist $\geigenspace{V}{f}{\lambda}$ ein $f$-invarianter Untervektorraum von $V$.
%     \item
%       Es gilt genau dann $\geigenspace{V}{f}{\lambda} \neq 0$, wenn $\lambda$ ein Eigenwert von $f$ ist.
%     \item
%       Für alle $\lambda \in K$ gibt es ein $m \in \Natural$ mit $\geigenspace{V}{f}{\lambda} = \heigenspace{V}{f}{\lambda}{m}$.
%       Insbesondere ist die Einschränkung $(f - \lambda \id_V)|_{\geigenspace{V}{f}{\lambda}}$ nilpotent.
%     \item
%       Die Summe $\sum_{\lambda \in K} \geigenspace{V}{f}{\lambda}$ ist direkt.
%   \end{enumerate}
% \end{lemma}
% 
% Mithilfe der verallgemeinerten Eigenräume ergibt sich eine Charakterisierung der Existenz der Jordan-Normalform:
% 
% \begin{theorem}
%   \label{theorem: existence of generalized eigespace decomposition and jordan normal form}
%   Die folgenden Bedingungen äquivalent:
%   \begin{enumerate}
%     \item
%       Das charakteristische Polynom $\charpol{f}(t)$ zerfällt in Linearfaktoren.
%     \item
%       Es gilt $V = \bigoplus_{\lambda \in K} \geigenspace{V}{f}{\lambda}$.
%     \item
%       Die Jordan-Normalform von $f$ existiert.
%   \end{enumerate}
% \end{theorem}
% 
% Die Zerlegung $V = \bigoplus_{\lambda \in K} \geigenspace{V}{f}{\lambda}$ aus Satz~\ref{theorem: existence of generalized eigespace decomposition and jordan normal form} ist die \emph{verallgemeinerte Eigenraumzerlegung}.
% Die Existenz der Jordan-Normalform ergibt sich aus der verallgemeinerten Eigenraumzerlegung wie folgt:
% 
% \begin{itemize}
%   \item
%     Ist $f$ nilpotent, so gibt es nach dem Verfahren aus Abschnitt~\ref{subsection: jordan normal form for nilpotent endomorphisms} eine Jordan-Normalform von $f$.
%   \item
%     Gibt es ein $\lambda \in K$, so dass $f - \id_V$ nilpotent ist, so gibt es nach dem vorherigen Fall eine Jordanbasis $\basis{B}$ von $V$, so dass
%     \[
%         \repmatrixendo{f - \lambda \id_V}{\basis{B}}
%       = \begin{pmatrix}
%           J_{n_1}(0)  &         &             \\
%                       & \ddots  &             \\
%                       &         & J_{n_t}(0)
%         \end{pmatrix}.
%     \]
%     gilt.
%     Dann gilt auch
%     \begin{align*}
%           \repmatrixendo{f}{\basis{B}}
%       &=  \repmatrixendo{f - \lambda \id_V}{\basis{B}}
%           +
%           \repmatrixendo{\lambda \id_V}{\basis{B}}
%       \\
%       &= \begin{pmatrix}
%           J_{n_1}(0)  &         &             \\
%                       & \ddots  &             \\
%                       &         & J_{n_t}(0)
%         \end{pmatrix}
%         +
%         \begin{pmatrix}
%           \lambda &         &         \\
%                   & \ddots  &         \\
%                   &         & \lambda
%         \end{pmatrix}
%       = \begin{pmatrix}
%           J_{n_1}(\lambda)  &         &                 \\
%                             & \ddots  &                 \\
%                             &         & J_{n_t}(\lambda)
%         \end{pmatrix}
%     \end{align*}
%   \item
%     Existiert die verallgemeinerte Eigenraumzerlegung $V = \bigoplus_{\lambda \in K} \geigenspace{V}{f}{\lambda}$, so gilt
%     \[
%       V = \geigenspace{V}{f}{\lambda_1} \oplus \dotsb \oplus \geigenspace{V}{f}{\lambda_t}
%     \]
%     wobei $\lambda_1, \dotsc, \lambda_t \in K$ die paarweise verschiedenen Eigenwerte von $f$ bezeichnet.
%     Ist $\basis{B}_i = (v_{i,1}, \dotsc, v_{i,n_i})$ eine Basis von $\geigenspace{V}{f}{\lambda_i}$, so ist
%     \[
%                 \basis{B}
%       \coloneqq ( v_{1, 1}, \dotsc, v_{1, n_1}, \dotsc, v_{t, 1}, \dotsc, v_{t, n_t} )
%     \]
%     eine Basis von $V$, und es gilt
%     \[
%         \repmatrixendo{f}{\basis{B}}
%       = \begin{pmatrix}
%             \repmatrixendo{ f|_{\geigenspace{V}{f}{\lambda_1}} }{\basis{B}_1}
%           & {}
%           & {}
%           \\
%             {}
%           & \ddots
%           & {}
%           \\
%             {}
%           & {}
%           & \repmatrixendo{ f|_{\geigenspace{V}{f}{\lambda_t}} }{\basis{B}_t}
%         \end{pmatrix}.
%     \]
%     Da die Endomorphismen $f|_{\geigenspace{V}{f}{\lambda_i}} - \lambda \id_{\geigenspace{V}{f}{\lambda_i}} = (f - \lambda \id_V)|_{\geigenspace{V}{f}{\lambda_i}}$ nilpotent ist, lassen sich die Basen $\basis{B}_i$ nach dem vorherigen Fall so wählen, dass die darstellenden Matrizen $\repmatrixendo{ f|_{\geigenspace{V}{f}{\lambda_i}} }{\basis{B}_i}$ in Jordan-Normalform sind.
%     Dann ist auch $\repmatrixendo{f}{\basis{B}}$ in Jordan-Normalform.
% \end{itemize}



\subsection{Lösung des Ähnlichkeitsproblems}

\begin{theorem}
  Zwei Endomorphismen $f, g \colon V \to V$ sind genau dann ähnlich, wenn sie (bis auf Permutation der Blöcke) die gleiche Jordan-Normalform besitzen.
\end{theorem}













