\chapter{Skalarprodukte}

Im Folgenden sei $K$ ein Körper.
Im Folgenden sei $\Korper \in \{ \Real, \Complex \}$.





\section{Allgemeine Definitionen}

\begin{definition}
  Es seien $U$, $V$ und $W$ $K$-Vektorräume.
  \begin{itemize}
    \item
      Eine Abbildung $\beta \colon V \times W \to U$ heißt \emph{$K$-bilinear}, falls
      \begin{gather*}
          \beta(v_1 + v_2, w)
        = \beta(v_1, w) + \beta(v_2, w),
        \\
          \beta(v, w_1 + w_2)
        = \beta(v, w_1) + \beta(v, w_2),
        \\
          \beta(\lambda v, w)
        = \lambda \beta(v, w)
        \quad\text{und}\quad
          \beta(v, \lambda w)
        = \lambda \beta(v, w)
      \end{gather*}
      für alle $v, v_1, v_2 \in V$, $w, w_1, w_2 \in W$ und $\lambda \in K$ gilt.
      Gilt zusätzlich $U = K$, so ist $\beta$ eine \emph{Bilinearform}.
      Es ist
      \[
                  \BilForm(V, W)
        \coloneqq \{ \beta \colon V \times W \to K \suchthat \text{$\beta$ ist eine Bilinearform} \}.
      \]
    
    \item
      Gilt zusätzlich $V = W$ und
      \[
          \beta(v_1, v_2)
        = \beta(v_2, v_1)
      \]
      für alle $v_1, v_2 \in V$, so heißt $\beta$ \emph{symmetrisch}.
      Es sind
      \begin{gather*}
        \BilForm(V) \coloneqq \BilForm(V, V)
      \shortintertext{sowie}
                  \SymForm(V)
        \coloneqq \{
                    \beta \in \BilForm(V)
                  \suchthat
                    \text{$\beta$ ist symmetrisch}
                  \}.
      \end{gather*}
  \end{itemize}
\end{definition}

\begin{definition}
  Es seien $U$, $V$ und $W$ $\Complex$-Vektorräume.
  \begin{itemize}
    \item
      Eine Abbildung $\beta \colon V \times W \to U$ heißt \emph{sesquilinear}, falls
      \begin{gather*}
          \beta(v_1 + v_2, w)
        = \beta(v_1, w) + \beta(v_2, w),
        \\
          \beta(v, w_1 + w_2)
        = \beta(v, w_1) + \beta(v, w_2),
        \\
          \beta(\lambda v, w)
        = \lambda \beta(v, w)
        \quad\text{und}\quad
          \beta(v, \lambda w)
        = \conjugate{\lambda} \beta(v,w)
      \end{gather*}
      für alle $v, v_1, v_2 \in V$, $w, w_1, w_2 \in W$ und $\lambda \in \Complex$ gilt.
      Gilt zusätzlich $U = \Complex$, so ist $\beta$ eine \emph{Sesquilinearform}.
      Es ist
      \[
                  \SesForm(V, W)
        \coloneqq \{
                    \beta \colon V \times W \to \Complex
                   \suchthat
                    \text{$\beta$ ist eine Sesquilinearform}
                  \}.
      \]
    
    \item
      Gilt zusätzlich $V = W$ und
      \[
          \beta(v_1, v_2)
        = \conjugate{\beta(v_2, v_1)}
      \]
      für alle $v_1, v_2 \in V$, so heißt $\beta$ \emph{hermitesch}.
      (Für alle $v \in V$ gilt dann $\beta(v,v) \in \Real$.)
      Es sind
      \begin{gather*}
        \SesForm(V) \coloneqq \SesForm(V, V)
      \shortintertext{sowie}
                  \HerForm(V)
        \coloneqq \{ \beta \in \BilForm(V) \suchthat \text{$\beta$ ist hermitesch} \}.
      \end{gather*}
  \end{itemize}
\end{definition}

\begin{definition}
  Eine Abbildung $f \colon V \to W$ zwischen $\Korper$-Vektorräumen $V$ und $W$ ist \emph{halblinear} falls
  \[
    f(v_1 + v_2) = f(v_1) + f(v_2)
    \quad\text{und}\quad
    f(\lambda v) = \conjugate{\lambda} f(v)
  \]
  für alle $v, v_1, v_2 \in V$ und $\lambda \in \Korper$ gilt.
\end{definition}

Eine Abbildung $\beta \colon V \times W \to U$ ist genau dann $K$-bilinear, wenn die Abbildungen
\[
          \beta(v,-)
  \colon  W
  \to     U,
  \quad   w
  \mapsto \beta(v,w)
  \quad\text{und}\quad
  \beta(-,w)
  \colon  V
  \to     U,
  \quad   v
  \mapsto \beta(v,w)
\]
für alle $v \in V$ und $w \in W$ linear sind.
Eine Abbildung $\beta \colon V \times W \to U$ ist genau dann sesquilinear, wenn die Abbildung $\beta(v,-) \colon W \to U$ für jedes $v \in V$ halblinear ist, und die Abbildung $\beta(-,w) \colon V \to U$ für jedes $w \in W$ linear ist.





\section{Orthogonalität}

Es sei $V$ ein $K$-Vektorraum und $\beta \in \SymForm(V)$ eine symmetrische Bilinearform, bzw.\ $K = \Complex$ und $\beta \in \HerForm(V)$ eine hermitsche Sesquilinearform.

\begin{definition}
  \begin{itemize}
    \item
      Zwei Vektoren $v_1, v_2 \in V$ sind \emph{orthogonal zueinander}, im Zeichen $v_1 \orth v_2$, falls $\beta(v_1, v_2) = 0$ gilt.
    \item
      Für jeden Untervektorraum $U \subseteq V$ ist $U^\perp = \{v \in V \suchthat \text{$\beta(v,u) = 0$ für alle $u \in U$}\}$ das \emph{orthogonale Komplement} von $U$ bezüglich $\beta$.
    \item
      Eine Familie $(v_i)_{i \in I}$ von Vektoren $v_i \in V$ heißt \emph{orthogonal}, falls $v_i \orth v_j$ für alle $i \neq j$ gilt.
    \item
      Eine \emph{Orthogonalbasis} ist eine orthogonale Basis.
    \item
      Ein Vektor $v \in V$ heißt \emph{normiert} falls $\beta(v,v) = 1$.
    \item
      Eine Familie $(v_i)_{i \in I}$ von Vektoren $v_i \in V$ heißt \emph{normiert}, falls $v_i$ für jedes $i \in I$ normiert ist.
    \item
      Eine \emph{Orthonormalbasis} ist eine orthogonale, normierte Basis.
  \end{itemize}
\end{definition}




\section{Definitheit \& Skalarprodukte}

Für eine hermitsche Sesquilinearform $\beta \in \HerForm(V)$ gilt $\beta(v,v) \in \Real$ für alle $v \in V$.
Daher ergibt die folgende Definition Sinn:

\begin{definition}
  Es sei $V$ ein $\Real$-Vektorraum und $\beta \in \SymForm(V)$, oder $V$ ein $\Complex$-Vektorraum und $\beta \in \HerForm(V)$.
  Dann ist $\beta$
  \begin{itemize}
    \item
      \emph{positiv definit}, falls $\beta(v,v) > 0$ für alle $v \neq 0$ gilt,
    \item
      \emph{positiv semidefinit}, falls $\beta(v,v) \geq 0$ für alle $v \neq 0$ gilt,
    \item
      \emph{negativ definit}, falls $\beta(v,v) < 0$ für alle $v \neq 0$ gilt,
    \item
      \emph{negativ semidefinit}, falls $\beta(v,v) \leq 0$ für alle $v \neq 0$ gilt,
    \item
      \emph{indefinit}, falls keiner der obigen Fälle eintritt.
  \end{itemize}
\end{definition}

\begin{definition}
  \begin{itemize}
    \item
      Ein \emph{Skalarprodukt} ist eine positiv definite symmetrische Bilinearform, bzw.\ hermitsche Sesquilinearform.
    \item
      Ein \emph{euklidischer Vektorraum} ein ein reeller Vektorraum $V$ zusammen mit einem Skalarprodukt auf $V$.
    \item
      Ein \emph{unitärer Vektorraum} ist ein komplexer Vektorraum $V$ zusammen mit einem Skalarprodukt auf $V$.
  \end{itemize}
  Wir bezeichnen euklidische Vektorräume und unitäre Vektorräume auch als \emph{Skalarprodukträume}.
  Das Skalarprodukt schreiben wir häufig nur als $\bil{-}{-}$.
\end{definition}

% TODO: Adding examples, in particular the standard inner product.

\begin{definition}
  Es sei $V$ ein Skalarproduktraum.
  Die \emph{Norm} eines Vektors $v \in V$ ist definiert als $\norm{v} \coloneqq \sqrt{ \bil{v}{v} }$.
\end{definition}

\begin{remark}
  Es handelt sich hierbei um eine Norm im Sinne der Analysis.
\end{remark}


% TODO: The norm for the previous examples.




\section{Das Gram-Schmidt-Verfahren}

Es sei $V$ ein Skalarproduktraum.
Es seien $v_1, \dotsc, v_n \in V$ linear unabhängig.
Induktiv konstruiere man Vektoren $w_1, \dotsc, w_n \in V$ wie folgt:
\begin{itemize}
  \item
    Man beginnt mit $w_1 \coloneqq w_1 / \norm{w_1}$.
  \item
    Falls $w_1, \dotsc, w_i$ definiert sind, so konstruiert man $w_{i+1}$ in zwei Schritten:
    \begin{itemize}
      \item
        Man berechne den Vektor $\tilde{w}_{i+1} \coloneqq v_{i+1} - \sum_{j=1}^i \bil{v_{i+1}}{w_j} w_j$.
        Der Vektor $\tilde{w}_{i+1}$ ist orthogonal zu $w_1, \dotsc, w_i$.
      \item
        Aus der linearen Unabhängigkeit von $v_1, \dotsc, v_n$ folgt, dass $\tilde{w}_{i+1} \neq 0$ gilt.
        Somit lässt sich $\tilde{w}_{i+1}$ normieren, und man erhält $w_{i+1} \coloneqq \tilde{w}_{i+1} / \norm{\tilde{w}_{i+1}}$.
    \end{itemize}
\end{itemize}
Die entstehende Familie $(w_1, \dotsc, w_n)$ ist orthonormal mit
\[
    \generated{v_1, \dotsc, v_i}
  = \generated{w_1, \dotsc, w_i}
  \qquad
  \text{für alle $i = 1, \dotsc, n$}.
\]

Hieraus folgen die Existenzaussagen über Orthonormalbasen:

\begin{theorem}
  Ist $U \subseteq V$ ein Untervektorraum, so lässt sich jede Orthonormalbasis von $U$ zu einer Orthonormalbasis von $V$ ergänzen.
  Inbesondere existiert eine Orthonormalbasis für $V$.
\end{theorem}

\begin{corollary}
  Für jeden Untervektorraum $U \subseteq V$ gilt $V = U \oplus U^\perp$ und $(U^\perp)^\perp = U$.
\end{corollary}

\begin{remark}
  Wendet man das Gram-Schmidt Verfahren auf linear abhängige Vektoren $v_1, \dotsc, v_n \in V$ an, so ergeben sich für das minimale $i$ mit $v_i \in \generated{v_1, \dotsc, v_{i-1}}$ zwar noch orthonormale Vektoren $w_1, \dotsc, w_{i-1}$, aber dann $\tilde{w}_i = 0$.
\end{remark}






\section{Die Adjungierte Abbildung}
\label{section: adjoint map for scalar products}

Es sei $f \colon V \to W$ eine lineare Abbildung zwischen Skalarprodukträumen $V$ und $W$.

\begin{proposition}
  Es gibt eine eindeutige Abbildung $\adj{f} \colon W \to V$ mit
  \begin{equation}
    \label{equation: definition of the adjoint map by elements for scalar products}
      \bil{f(v)}{w}
    = \bil{v}{\adj{f}(w)}
    \qquad
    \text{für alle $v \in V$, $w \in W$},
  \end{equation}
  und $\adj{f}$ ist linear.
\end{proposition}

\begin{lemma}
  \begin{itemize}
    \item
      Es gilt $\adj{(\adj{f})} = f$.
    \item
      Es gilt $\adj{\id_V} = \id_V$.
    \item
      Es gilt $\adj{(f \circ g)} = \adj{g} \circ \adj{f}$.
    \item
      Die Abbildung $\Hom{V}{W} \to \Hom{W}{V}$, $f \mapsto \adj{f}$ ist antilinear.
  \end{itemize}
\end{lemma}

In orthonormalen Koordinaten entspricht das Adjungieren einer linearen Abbildung dem Transponieren-Konjugieren.

\begin{notation}
  Für alle $\matrices{n}{\Korper}$ schreiben wir $\madj{A} \coloneqq \transpose{\conjugate{A}}$.
\end{notation}

\begin{lemma}
  Ist $\basis{B}$ eine Orthonormalbasis von $V$ und $\basis{C}$ eine Orthonormalbasis von $W$, so gilt
  \[
      \repmatrixhom{\adj{f}}{\basis{C}}{\basis{B}}
    = \madj{ \repmatrixhom{f}{\basis{B}}{\basis{C}} }.
  \]
\end{lemma}

Die adjungierte Abbildung $\adj{f} \colon W \to V$ hängt eng mit der dualen Abbildung $\dual{f} \colon \dual{W} \to \dual{V}$, $\varphi \mapsto \varphi \circ f$ zusammen:

Das Skalarprodukt $\bil{-}{-}$ auf $V$ liefert eine Abbildung
\[
          \Phi_{\basis{B}}
  \colon  V
  \to     V^*,
  \quad   v
  \mapsto \bil{-}{v}.
\]
Diese Abbildung ist wohldefiniert, da $\bil{-}{-}$ im ersten Argument linear ist, und halblinear, da $\Phi_{\basis{B}}$ linear im zweiten Argument ist.
Die Abbildung $\Phi_{\basis{B}}$ ist bereits ein Isomorphismus, denn es gilt $\dim \dual{V} = \dim V$ und $\Phi_{\basis{B}}$ ist injektiv\footnote{Hier nutzen wir, dass die aus der Linearen~Algebra~I bekannten Aussagen über linearen Abbildungen auch für halblineare Abbildungen gelten.}, denn für $v \in \ker \Phi_{\basis{B}}$ gilt $\bil{-}{v} = \Phi_{\basis{B}}(v) = 0$, also $\|v\|^2 = \bil{v}{v} = 0$ und somit $v = 0$.
Gleiches gilt für $\Phi_{\basis{C}} \colon W \to W^*$, $w \mapsto \bil{-}{w}$.

Die Bedingung \eqref{equation: definition of the adjoint map by elements for scalar products} ist äquivalent dazu, dass das folgende Diagramm kommutiert:
\[
  \begin{tikzcd}
      V^*
    & W^*
      \arrow[swap]{l}{\dual{f}}
    \\
      V
      \arrow{u}{\Phi_{\basis{B}}}
    & W
      \arrow[swap]{u}{\Phi_{\basis{C}}}
      \arrow{l}{\adj{f}}
  \end{tikzcd}
\]
Somit ist $\adj{f}$ eindeutig bestimmt als $\adj{f} = \Phi_{\basis{B}}^{-1} \circ \dual{f} \circ \Phi_{\basis{C}}$.

























