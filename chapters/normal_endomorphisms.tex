\section{Normalenformen für normale Endomorphismen}

Es seien $V$ und $W$ endlichdimensionale Skalarprodaukträume.





\subsection{Anfängliche Definitionen}

\begin{definition}
  \begin{itemize}
    \item
      Eine Matrix $A \in \matrices{n}{\Korper}$ heißt \emph{unitär} im Fall $\Korper = \Complex$, bzw.\ \emph{orthogonal} im Fall $\Korper = \Real$, falls sie die folgenden äquivalenten Bedingungen erfüllt:
      \begin{enumerate}
        \item
          Die Matrix $A$ ist invertierbar mit $A^{-1} = \madj{A}$.
        \item
          Es gilt $A \madj{A} = \Id$.
        \item
          Es gilt $\madj{A} A = \Id$.
        \item
          Die Spalten von $A$ bilden eine Orthonormalbasis von $\Korper^n$.
        \item
          Die Zeilen von $A$ bilden eine Orthonormalbasis von $\Korper^n$ (betrachtet als Zeilenvektoren).
      \end{enumerate}
    \item
      Es seien
      \begin{align*}
                    \Unitary{n}
        &\coloneqq  \{ U \in \matrices{n}{\Complex} \suchthat \text{$U$ ist unitär} \}
      \shortintertext{und}
                    \Orthogonal{n}
        &\coloneqq  \{ U \in \matrices{n}{\Real} \suchthat \text{$U$ ist orthogonal} \}.
      \end{align*}
  \end{itemize}
\end{definition}

Für alle $\varphi \in \real$ sei
\[
            D(\varphi)
  \coloneqq \begin{pmatrix*}[r]
              \cos \varphi  & -\sin \varphi \\
              \sin \varphi  &  \cos \varphi
            \end{pmatrix*}
  \in       \matrices{2}{\Real}
\]
die Drehmatrix um den Winkel $\varphi$.





\subsection{Normale Endomorphismen}

Es sei $f \colon V \to V$ ein Endomorphismus eines endlichdimensionalen Skalarproduktraums $V$.

\begin{definition}
  Der Endomorphismus $f$ heißt \emph{normal} falls $f$ und $\adj{f}$ kommutieren, d.h.\ falls $f \circ \adj{f} = \adj{f} \circ f$ gilt.
  Eine Matrix $A$ heißt normal, falls $A$ und $\madj{A}$ kommutieren, d.h.\ falls $A \madj{A} = \madj{A} A$ gilt.
\end{definition}

\begin{lemma}
  Die folgenden Bedingungen sind äquivalent:
  \begin{enumerate}
    \item
      Der Endomorphismus $f$ ist normal.
    \item
      Es gibt eine Orthonormalbasis $\basis{B}$ von $V$, so dass die darstellende Matrix $\repmatrixendo{f}{\basis{B}}$ normal ist.
    \item
      Für jede Orthonormalbasis $\basis{B}$ von $V$ ist die darstellende Matrix $\repmatrixendo{f}{\basis{B}}$ normal.
  \end{enumerate}
\end{lemma}



\subsubsection{Der komplexe Fall}

\begin{theorem}
  Für $\Korper = \Complex$ sind die folgenden Bedingungen sind äquivalent:
  \begin{enumerate}
    \item
      Der Endomorphismus $f$ ist unitär.
    \item
      Es gibt eine Orthonormalbasis von $V$ aus Eigenvektoren von $f$.
    \item
      Der Endomorphismus von $f$ ist diagonalisierbar und die Eigenräume sind orthogonal zueinander.
  \end{enumerate}
\end{theorem}

\begin{corollary}
  Für eine Matrix $A \in \matrices{n}{\Complex}$ sind die folgenden Bedingungen äquivalent:
  \begin{enumerate}
    \item
      Die Matrix $A$ ist normal.
    \item
      Es gibt eine Orthonormalbasis von $V$ aus Eigenvektoren von $A$.
    \item
      Die Matrix $A$ ist diagonalisierbar und die Eigenräume sind orthogonal zueinander.
    \item
      Es gibt eine unitäre Matrix $U \in \Unitary{n}$, so dass $U^{-1} A U$ eine Diagonalmatrix ist.
  \end{enumerate}
\end{corollary}



\subsubsection{Der reelle Fall}

\begin{theorem}
  Für $\Korper = \Real$ sind die folgenden Bedingungen sind äquivalent:
  \begin{enumerate}
    \item
      Der Endomorphismus $f$ ist normal.
    \item
      Es gibt eine Orthonormalbasis $\basis{B}$ von $V$, so dass $\repmatrixendo{f}{\basis{B}}$ von der Form
      \begin{equation}
      \label{equation: normal form for real normal endomorphisms}
          \repmatrixendo{f}{\basis{B}}
        = \begin{pmatrix}
            \lambda_1 &         &           &                   &         &                   \\
                      & \ddots  &           &                   &         &                   \\
                      &         & \lambda_t &                   &         &                   \\
                      &         &           & r_1 D(\varphi_1)  &         &                   \\
                      &         &           &                   & \ddots  &                   \\
                      &         &           &                   &         & r_s D(\varphi_s)
          \end{pmatrix}
      \end{equation}
      mit $\lambda_1, \dotsc, \lambda_t \in \Real$, $r_1, \dotsc, r_s > 0$ und $\varphi_1, \dotsc, \varphi_s \in (0, \pi)$ ist.
      Diese Form ist bis auf Permutation der Diagonalblöcke eindeutig.
  \end{enumerate}
\end{theorem}

\begin{corollary}
  Für $A \in \matrices{n}{\Real}$ sind die folgenden Bedingungen äquivalent:
  \begin{enumerate}
    \item
      Die Matrix $A$ ist normal.
    \item
      Es gibt eine orthonormale Matrix $U \in \Orthogonal{n}$, so dass $U^{-1} A U$ in der Form \eqref{equation: normal form for real normal endomorphisms} ist.
      Diese Form ist eindeutig bis auf Permutation der Diagonalblöcke.
  \end{enumerate}
\end{corollary}





\subsection{Unitäre \& Orthogonale Endomorphismen}

\begin{definition}
  Eine lineare Abbildung $f \colon V \to W$ heißt \emph{unitär} im Fall $\Korper = \Complex$, bzw.\ \emph{orthogonal} im Fall $\Korper = \Real$, falls
  \[
      \bil{f(v_1)}{f(v_2)}
    = \bil{v_1}{v_2}
    \qquad
    \text {für alle $v_1, v_2 \in V$}.
  \]
  gilt.
  Im Fall $\Korper = \Complex$ sei
  \begin{align*}
                \Unitary{V}
    &\coloneqq  \{ f \in \End{V} \suchthat \text{$f$ ist unitär} \}
  \intertext{und im Fall $\Korper = \Real$ sei}
                \Orthogonal{V}
    &\coloneqq  \{ f \in \End{V} \suchthat \text{$f$ ist orthogonal} \}
  \end{align*}
\end{definition}

\begin{lemma}
  \label{lemma: properties of unitary and orthogonal transformations}
  \begin{itemize}
    \item
      Unitäre und orthogonale Abbildungen sind injektiv.
    \item
      Eine lineare Abbildung $f \colon V \to W$ ist genau dann unitär, bzw.\ orthogonal, falls $f$ eine Isometrie ist, d.h.\ falls
      \[
        \norm{ f(v) } = \norm{v}
        \qquad
        \text{für alle $v \in V$ gilt.}
      \]
    \item
      Es ist $\Unitary{V} \subseteq \GL{}{V}$, bzw.\ $\Orthogonal{V} \subseteq \GL{}{V}$ eine Untergruppe.
  \end{itemize}
\end{lemma}

Für den zweiten Punkt aus Lemma~\ref{lemma: properties of unitary and orthogonal transformations} nutzt man die \emph{Polarisierungsformeln}:
Im Fall $\Korper = \Real$ gilt
\[
    \bil{v_1}{v_2}
  = \frac{ \norm{v_1 + v_2}^2 - \norm{v_1}^2 - \norm{v_2}^2 }{2}
  = \frac{ \norm{v_1 + v_2}^2 - \norm{v_1 - v_2}^2 }{4},
\]
und im Fall $\Korper = \Complex$ gilt
\[
    \bil{v_1}{v_2}
  = \frac{ \norm{v_1 + v_2}^2 - \norm{v_1 - v_2}^2 + i \norm{v_1 + i v_2}^2 - i \norm{v_1 - i v_2}^2 }{4}.
\]

Es sei nun $f \colon V \to V$ ein Endomorphismus.

\begin{lemma}
  Die folgenden Bedingungen sind äquivalent:
  \begin{enumerate}
    \item
      Der Endomorphismus $f$ ist unitär, bzw.\ orthogonal.
    \item
      Der Endomorphismus $f$ ist ein Isomorphismus mit $f^{-1} = \adj{f}$.
    \item
      Es gilt $f \circ \adj{f} = \id_V$.
    \item
      Es gilt $\adj{f} \circ f = \id_V$.
  \end{enumerate}
\end{lemma}

\begin{lemma}
  Die folgenden Bedingungen äquivalent:
  \begin{enumerate}
    \item
      Der Endomorphismus $f$ ist unitär, bzw.\ orthogonal.
    \item
      Es gibt eine Orthonormalbasis $\basis{B}$ von $V$, so dass die darstellende Matrix $\repmatrixendo{f}{\basis{B}}$ unitär ist.
    \item
      Für jede Orthonormalbasis $\basis{B}$ von $V$ ist die darstellende Matrix $\repmatrixendo{f}{\basis{B}}$ unitär.
  \end{enumerate}
  Im Fall $\Korper = \Real$ sind diese unitären Matrizen dabei bereits orthogonal.
\end{lemma}



\subsubsection{Der komplexe Fall}

\begin{theorem}
  Für $\Korper = \Complex$ sind die folgenden Bedingungen sind äquivalent:
  \begin{enumerate}
    \item
      Der Endomorphismus $f$ ist unitär.
    \item
      Der Endomorphismus $f$ ist normal,und für jeden Eigenwert $\lambda \in \Complex$ von $f$ gilt $\abs{\lambda} = 1$.
  \end{enumerate}
\end{theorem}

\begin{corollary}
  Für eine Matrix $A \in \matrices{n}{\Complex}$ sind die folgenden Bedingungen äquivalent:
  \begin{enumerate}
    \item
      Die Matrix $A$ ist unitär.
    \item
      Die Matrix $A$ ist normal, und für jeden Eigenwert $\lambda \in \Complex$ von $A$ gilt $\abs{\lambda} = 1$.
  \end{enumerate}
\end{corollary}



\subsubsection{Der reelle Fall}

\begin{theorem}
  Für $\Korper = \Real$ sind die folgenden Bedingungen sind äquivalent:
  \begin{enumerate}
    \item
      Der Endomorphismus $f$ ist orthogonal.
    \item
      Es gibt eine Orthonormalbasis $\basis{B}$ von $V$, so dass $\repmatrixendo{f}{\basis{B}}$ von der Form
      \begin{equation}
      \label{equation: normal form for orthogonal endomorphisms}
          \repmatrixendo{f}{\basis{B}}
        = \begin{pmatrix}
            \pm 1 &         &       &               &         &               \\
                  & \ddots  &       &               &         &               \\
                  &         & \pm 1 &               &         &               \\
                  &         &       & D(\varphi_1)  &         &               \\
                  &         &       &               & \ddots  &               \\
                  &         &       &               &         & D(\varphi_s)
          \end{pmatrix}
      \end{equation}
      mit $\varphi_1, \dotsc, \varphi_s \in (0, \pi)$ ist.
      Diese Form ist bis auf Permutation der Diagonalblöcke eindeutig.
      (Inbesondere ist die Anzahl der Einsen und Minus-Einsen jeweils eindeutig.)
  \end{enumerate}
\end{theorem}

\begin{corollary}
  Für $A \in \matrices{n}{\Real}$ sind die folgenden Bedingungen äquivalent:
  \begin{enumerate}
    \item
      Die Matrix $A$ ist normal.
    \item
      Es gibt eine orthonormale Matrix $U \in \Orthogonal{n}$, so dass $U^{-1} A U$ in der Form \eqref{equation: normal form for orthogonal endomorphisms} ist.
      Diese Form ist eindeutig bis auf Permutation der Diagonalblöcke.
  \end{enumerate}
\end{corollary}





\subsection{Selbstadjungierte Endomorphismen}

Es sei nun $f \colon V \to V$ ein Endomorphismus.

\begin{definition}
  Der Endomorphismus $f$ heißt \emph{selbstadjungiert}, falls $\adj{f} = f$ gilt.
  Eine Matrix $A \in \matrices{n}{\Complex}$ heißt \emph{hermitesch}, falls $\madj{A} = A$ gilt.
\end{definition}

Inbesondere ist eine reelle Matrix genau dann hermitesch, wenn sie symmetrisch ist.

\begin{lemma}
  Die folgenden Bedingungen sind äquivalent:
  \begin{enumerate}
    \item
      Der Endomorphismus $f$ ist selbstadjungiert.
    \item
      Es gibt eine Orthonormalbasis $\basis{B}$ von $V$, so dass $\repmatrixendo{f}{\basis{B}}$ hermitesch ist.
    \item
      Für jede Orthonormalbasis $\basis{B}$ von $V$ ist $\repmatrixendo{f}{\basis{B}}$ hermitesch.
  \end{enumerate}
\end{lemma}

\begin{theorem}
  Die folgenden Bedingungen sind äquivalent:
  \begin{enumerate}
    \item
      Der Endomorphismus $f$ ist selbstadjungiert.
    \item
      Es gibt eine Orthonormalbasis von $V$ aus Eigenvektoren von $f$, und alle Eigenwerte von $f$ sind reell.
    \item
      Der Endomorphismus $f$ ist diagonalisierbar, und die Eigenräume sind orthogonal zueinander.
  \end{enumerate}
\end{theorem}

\begin{corollary}
  \label{corollary: hermitian matrices are diagonalizable}
  Für eine Matrix $A \in \matrices{n}{\Korper}$ sind die folgenden Bedingungen äquivalent:
  \begin{enumerate}
    \item
      Die Matrix $A$ ist hermitesch.
    \item
      Es gibt eine Orthonormalbasis von $\Korper^n$ aus Eigenvektoren von $A$, und alle Eigenwerte von $A$ sind reell.
    \item
      Es gibt eine unitäre Matrix $U \in \Unitary{n}$, so dass $U^{-1} A U$ eine Diagonalmatrix mit reellen Diagonaleinträgen ist.
    \item
      Die Matrix $A$ ist diagonalisierbar, und die Eigenräume sind orthogonal zueinander.
  \end{enumerate}
  Im Fall $\Korper = \Real$ lässt sich die Matrix $U$ reell wählen, und somit als $U \in \Orthogonal{n}$.
\end{corollary}

\begin{definition}
  Ein selbstadjungierter Endomorphismus $f \colon V \to V$ \textup(bzw.\ eine hermitesche Matrix $A \in \matrices{n}{\Korper}$\textup) heißt \emph{positiv definit} falls die symmetrische Bilinearform $\beta \in \SymForm(V)$ \textup(bzw.\ $\beta \in \SymForm(\Real^n)$\textup) mit $\beta(v_1, v_2) \coloneqq \bil{f(v_1)}{v_2}$ positiv definit ist.
  
  Analog sind für $f$ \textup(bzw.\ $A$\textup) die Eigenschaften \emph{positiv semidefinit}, \emph{negativ definit}, \emph{negativ semidefinit} und \emph{indefinit} definiert.
\end{definition}

\begin{corollary}
  Ein selbstadjungierter Endomorphismus $f \colon V \to V$, \textup(bzw.\ eine hermitesche Matrix $A \in \matrices{n}{\Korper}$\textup) ist genau dann positiv definit, wenn alle Eigenwerte von $f$ \textup(bzw.\ $A$\textup) positiv sind.
  
  Für positiv semidefinit, negativ definit, negativ semidefinit und indefinit definiert gelten die analogen Aussagen.
\end{corollary}





\subsection{Übersicht}

Die verschiedenen Normalenformen lassen sich wie folgt zusammenfassen:

\begin{center}
  \tabcolsep=0pt
  \begin{tabular}[4]{lccc}
      {}
    & \begin{tabular}[1]{c}
        Charakterisierung \\
        durch $\adj{f}$
      \end{tabular}
    & $\Korper = \Complex$
    & $\Korper = \Real$
    \\
    \hline
    \hline
    \\
      normal
    & $f \adj{f} = \adj{f} f$
    & \begingroup
        \renewcommand*{\arraystretch}{1.5}
        \begin{tabular}[1]{@{}c}
        
        $ \begin{psmallmatrix}
            \lambda_1 &         &           \\
                      & \ddots  &           \\
                      &         & \lambda_n
          \end{psmallmatrix}$
        \\
        $\lambda_i \in \Complex$
        \end{tabular}
      \endgroup
    & \begingroup
        \renewcommand*{\arraystretch}{1.5}
        \begin{tabular}[1]{@{}c}
          $ \begin{psmallmatrix}
              \lambda_1 &         &           &                   &         &                   \\
                        & \ddots  &           &                   &         &                   \\
                        &         & \lambda_t &                   &         &                   \\
                        &         &           & r_1 D(\varphi_1)  &         &                   \\
                        &         &           &                   & \ddots  &                   \\
                        &         &           &                   &         & r_s D(\varphi_s)
            \end{psmallmatrix} $
        \\
        $\lambda_i \in \Real$, $r_i > 0$,  $\varphi_i \in (0, \pi)$
        \end{tabular}
      \endgroup
    \\
    \hline
    \\
      unitär,
      orthogonal
    & $\adj{f} = f^{-1}$
    & \begingroup
        \renewcommand*{\arraystretch}{1.5}
        \begin{tabular}[1]{@{}c}
        
        $ \begin{psmallmatrix}
            \lambda_1 &         &           \\
                      & \ddots  &           \\
                      &         & \lambda_n
          \end{psmallmatrix}$
        \\
        $\abs{\lambda_i} = 1$
        \end{tabular}
      \endgroup
    & \begingroup
        \renewcommand*{\arraystretch}{1.5}
        \begin{tabular}[1]{@{}c}
          $ \begin{psmallmatrix}
              \pm 1 &         &       &               &         &               \\
                    & \ddots  &       &               &         &               \\
                    &         & \pm 1 &               &         &               \\
                    &         &       & D(\varphi_1)  &         &               \\
                    &         &       &               & \ddots  &               \\
                    &         &       &               &         & D(\varphi_s)
            \end{psmallmatrix} $
        \\
        $\varphi_i \in (0, \pi)$
        \end{tabular}
      \endgroup
    \\
    \hline
    \\
      selbstadjungiert
    & $\adj{f} = f$
    & \multicolumn{2}{@{}c}
      {
       \begingroup
        \renewcommand*{\arraystretch}{1.5}
        \begin{tabular}[1]{@{}c}
        
        $ \begin{pmatrix}
            \lambda_1 &         &           \\
                      & \ddots  &           \\
                      &         & \lambda_n
          \end{pmatrix}$
        \\
        $\lambda_i \in \Real$
        \end{tabular}
        \endgroup
      }
  \end{tabular}
\end{center}



\subsection{Polar- und Iwasawa-Zerlegung}

\begin{lemma}
  \label{lemma: existence of square roots}
  \begin{itemize}
    \item
      Für jeden positiv semidefiniten selbstadjungierten Endomorphismus $f \colon V \to V$ gibt es eine eindeutigen positiv semidefinitinen selbstadjungierten Endomorphismus $g \colon V \to V$ mit $f = g^2$.
    \item
      Für jede positiv semidefinite hermitesche Matrix $A \in \matrices{n}{\Korper}$ gibt es eine eindeutige positiv semidefinite hermitesche Matrix $B \in \matrices{n}{\Korper}$ mit $A = B^2$.
  \end{itemize}
\end{lemma}

In der Situation von Lemma~\ref{lemma: existence of square roots} schreiben wir $g = \sqrt{f}$ (bzw.\ $B = \sqrt{A}$).

\begin{theorem}[Polarzerlegung]
  \begin{itemize}
    \item
      Für jeden Endomorphismus $f \colon V \to V$ gibt es einen selbstadjungierten Endomorphismus $s \colon V \to V$ und einen unitären, bzw.\ orthogonalen Endomorphismus $u \colon V \to V$ mit $f = s \circ u$.
      Der Endomorphismus $s$ ist eindeutig bestimmt durch $s = \sqrt{f \circ \adj{f}}$.
      Ist $f$ invertierbar, so ist auch $u$ eindeutig.
    
    \item
      Für jede Matrix $A \in \matrices{n}{\Korper}$ gibt eine hermitesche Matrix $S \in \matrices{n}{\Korper}$ und eine unitäre, bzw.\ orthogonale Matrix $U \in \matrices{n}{\Korper}$ mit $A = SU$.
      Die Matrix $S$ ist eindeutig bestimmt durch $S = \sqrt{A \madj{A}}$.
      Ist $A$ invertierbar, so ist auch $U$ eindeutig.
  \end{itemize}
\end{theorem}

\begin{theorem}[Iwasawa-Zerlegung]
  Es seien $K, A, N \subseteq \GL{n}{\Korper}$ die folgenden Untergruppen:
  \begin{itemize}
    \item
      Es ist $K = \Unitary{n}$, bzw.\ $K = \Orthogonal{n}$.
    \item
      $A$ besteht aus den Diagonalmatrizen mit reellen, positiven Diagonaleinträgen.
    \item
      $N$ besteht aus den obenen Dreiecksmatrizen mit Einsen auf der Diagonalen.
  \end{itemize}
  Dann gibt es für jede Matrix $s \in \GL{n}{\Korper}$ eindeutige Matrizen $k \in K$, $a \in A$ und $n \in N$ mit $s = kan$, d.h.\ die Abbildung
  \[
            K \times A \times N
    \to     \GL{n}{\Korper},
    \quad   (k, a, n)
    \mapsto kan
  \]
  ist eine Bijektion.
\end{theorem}

\begin{warning}
  Diese Bijektion ist i.A.\ kein Gruppenhomomorphismus.
  Sie ist aber ein Homöomorphismus (und sogar ein Diffeomorphismus).
\end{warning}







