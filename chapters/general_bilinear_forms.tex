\chapter{Mehr zu Bilinearformen}

Im Folgenden seien $V$ und $W$ zwei endlichdimensionale $K$-Vektorräume.





\section{Darstellende Matrizen \& Kongruenz}

\begin{definition}
  \label{definition: representing matrix for a bilinear form}
  \leavevmode
  \begin{itemize}
    \item
      Die \emph{darstellende Matrix} einer Bilinearform $\beta \in \BilForm(V,W)$ bezüglich einer Basis $\basis{B} = (v_1, \dotsc, v_m)$ von $V$ und einer Basis $\basis{C} = (w_1, \dotsc, w_n)$ von $W$ ist die Matrix
      \[
                  \repmatrixbildiff{\beta}{\basis{B}}{\basis{C}}
        \coloneqq \begin{pmatrix}
                    \beta(v_1, w_1) & \beta(v_1, w_2) & \cdots  & \beta(v_1, w_n) \\
                    \beta(v_2, w_1) & \beta(v_2, w_2) & \cdots  & \beta(v_2, w_n) \\
                    \vdots          & \vdots          & \ddots  & \vdots          \\
                    \beta(v_m, w_1) & \beta(v_m, w_2) & \cdots  & \beta(v_m, w_n)
                  \end{pmatrix}
                  \in \mnatrices{m}{n}{K}.
      \]
    \item
      Die \emph{darstellende Matrix} einer Bilinearform $\beta \in \BilForm(V)$ bezüglich einer Basis $\basis{B}$ von $V$ ist $\repmatrixbilsame{\beta}{\basis{B}} \coloneqq \repmatrixbildiff{\beta}{\basis{B}}{\basis{B}}$.
  \end{itemize}
\end{definition}

\begin{proposition}
  Es sei $\basis{B} = (v_1, \dotsc, v_m)$ eine Basis von $V$ und $\basis{C} = (w_1, \dotsc, w_n)$ eine Basis von $W$, und es seien
  \begin{gather*}
              V
    \to       K^m,
    \quad     v
    =         \sum_{i=1}^m x_i v_i
    \mapsto   \transpose{ (x_1, \dotsc, x_m) }
    \eqqcolon \coord{v}{\basis{B}}
  \shortintertext{und}
              W
    \to       K^n,
    \quad     w
    =         \sum_{i=1}^n y_i w_i
    \mapsto   \transpose{ (y_1, \dotsc, y_n) }
    \eqqcolon \coord{w}{\basis{C}}
  \end{gather*}
  die zugehörige Isomorphismen $V \to K^n$ und $W \to K^n$.
  \begin{enumerate}
    \item
      Für $\beta \in \BilForm(V,W)$ ist die darstellende Matrix $\repmatrixbildiff{\beta}{\basis{B}}{\basis{C}}$ eindeutig dadurch bestimmt, dass
      $\beta(v, w) = \transpose{\coord{v}{\basis{B}}} \, \repmatrixbildiff{\beta}{\basis{B}}{\basis{C}} \, \coord{w}{\basis{C}}$
      für alle $v \in V$, $w \in W$ gilt.
    \item
      Die Abbildung
      \[
                \BilForm(V,W)
        \to     \mnatrices{m}{n}{K},
        \quad   \beta
        \mapsto \repmatrixbildiff{\beta}{\basis{B}}{\basis{C}}
      \]
      ist ein Isomorphismus von $K$-Vektorräumen.
  \end{enumerate}
\end{proposition}

\begin{lemma}
  Für $\beta \in \BilForm(V)$ sind die folgenden Bedingungen äquivalent:
  \begin{enumerate}
    \item
      Die Bilinearform $\beta$ ist symmetrisch.
    \item
      Es gibt eine Basis $\basis{B}$ von $V$, so dass $\repmatrixbilsame{\beta}{\basis{B}}$ symmetrisch ist.
    \item
      Für jede Basis $\basis{B}$ von $V$ ist $\repmatrixbilsame{\beta}{\basis{B}}$ symmetrisch.
  \end{enumerate}
\end{lemma}

\begin{lemma}
  Sind $\basis{B}$, $\basis{B}'$ zwei Basen von $V$ und $\basis{C}$, $\basis{C}'$ zwei Basen von $W$, so gilt für jede Bilinearform $\beta \in \BilForm(V,W)$, dass
  \[
      \repmatrixbildiff{\beta}{\basis{B}}{\basis{C}}
    = \repmatrixhom{\id_V}{\basis{B}}{\basis{B}'}^T
      \repmatrixbildiff{\beta}{\basis{B}'}{\basis{C}'}
      \repmatrixhom{\id_W}{\basis{C}}{\basis{C}'}.
  \]
\end{lemma}

\begin{definition}
  Zwei Matrizen $A, B \in \matrices{n}{K}$ heißen \emph{kongruent} zueinander, falls es $S \in \GL{n}{K}$ mit $A = \transpose{S} B S$ gibt.
\end{definition}

Kongruente Matrizen stellen die gleiche Bilinearform bezüglich verschiedener Basen dar, d.h.\ für $n = \dim V$ sind zwei Matrizen $A, B \in \matrices{n}{K}$ genau dann kongruent, falls es eine Bilinearform $\beta \in \BilForm(V)$ sowie Basen $\basis{A}$ und $\basis{B}$ von $V$ gibt, so dass $A = \repmatrixbilsame{\beta}{\basis{A}}$ und $B = \repmatrixbilsame{\beta}{\basis{B}}$ gelten.

\begin{corollary}
  Konkruenz ist eine Äquivalenzrelation auf $\matrices{n}{K}$.
\end{corollary}

\begin{definition}
  Zwei Bilinearformen $\beta_1, \beta_2 \in \BilForm(V)$ sind \emph{kongruent}, wenn sie die folgenden äquivalenten Bedingungen erfüllen:
  \begin{enumerate}
    \item
      Es gibt eine Basis $\basis{B}$ von $V$, so dass $\repmatrixbilsame{\beta_1}{\basis{B}}$ und $\repmatrixbilsame{\beta_2}{\basis{B}}$ kongruent sind.
    \item
      Für jede Basis $\basis{B}$ von $V$ sind $\repmatrixbilsame{\beta_1}{\basis{B}}$ und $\repmatrixbilsame{\beta_2}{\basis{B}}$ kongruent.
    \item
      Es gibt $f \in \GL{}{V}$ mit $\beta_1( f(v_1), f(v_2) ) = \beta_2( v_1, v_2 )$ für alle $v_1, v_2 \in V$.
  \end{enumerate}
\end{definition}

\begin{definition}
  Der \emph{Rang} einer Bilinearform $\beta \in \BilForm(V)$ ist $\rank \beta \coloneqq \rank \repmatrixbilsame{\beta}{\basis{B}}$, wobei $\basis{B}$ eine Basis von $V$ ist.
\end{definition}





\section{Existenz von Orthogonalbasen}

Es sei $\ringchar{K} \neq 2$ und $\beta \in \BilForm(V)$ eine Bilinearform.

Falls eine eine Orthogonalbasis $\basis{B} = (v_1, \dotsc, v_n)$ von $V$ bezüglich $\beta$ gibt, so ist $\beta$ bereits symmetrisch, denn die Matrix
\[
    \repmatrixbilsame{\beta}{\basis{B}}
  = \begin{pmatrix}
      \beta(v_1, v_1) &         &                 \\
                      & \ddots  &                 \\
                      &         & \beta(v_n, v_n)
    \end{pmatrix}
\]
ist symmetrisch.
Es gilt auch die Umkehrung (da $\ringchar{K} \neq 2$):

\begin{theorem}
  \label{theorem: exstience of an orthogonal basis}
  Es gibt eine Orthogonalbasis von $V$ bezüglich $\beta$.
\end{theorem}

\begin{corollary}
  \label{corollary: every symmetric matrix is congruent to a diagonal matrix}
  Für jede symmetrische Matrix $A \in \matrices{n}{K}$ gibt es eine Basiswechselmatrix $S \in \GL{n}{K}$, so dass $\transpose{S} A S$ eine Diagonalmatrix ist.
\end{corollary}

Ist $A \in \matrices{n}{K}$ eine symmetrische Matrix, so lässt sich eine Diagonalform wie in Korollar~\ref{corollary: every symmetric matrix is congruent to a diagonal matrix} sowie eine zugehörige Basiswechselmatrix $S \in \GL{n}{K}$ mithilfe von \emph{simultanen Zeilen- und Spaltenumformungen} berechnen:
\begin{itemize}
  \item
    Man wende elementare Zeilenumformungen auf die Matrix an.
  \item
    Nach jeder elementaren Zeilenumformung führt die man die analoge elementare Spaltenumformung durch.
    Hierdurch bleibt die Matrix symmetrisch.
\end{itemize}
Wie beim Gauß-Vefahren bringt man die Matrix hierdurch in Zeilenstufenform.
Als Ergebnis erhält man eine symmetrische Matrix $B$ in oberer Dreiecksform, also eine Diagonalmatrix.
Eine Matrix $S \in \GL{n}{K}$ mit $\transpose{S} A S = B$ erhält man, indem man die genutzen elementaren Spaltenumformungen in gleicher Reihenfolge auf die Einheitsmatrix $\Id$ anwendet.
(Nutzt man anstelle der Spaltenumformungen die Zeilenumformungen, so erhält man stattdessen $\transpose{S}$.)

\begin{example}
  \label{example: simultaneous row and column operations}
  Es sei
  \[
              A
    \coloneqq \begin{pmatrix*}[r]
                2 &  1 & 3  \\
                1 & -1 & 0  \\
                3 &  0 & 2
              \end{pmatrix*}.
  \]
  Durch simultane Zeilen- und Spaltenumfomungen erhalten wir Folgendes:
  \begin{alignat*}{4}
    \begin{pmatrix*}[r]
      2 &  1 & 3  \\
      1 & -1 & 0  \\
      3 &  0 & 2
    \end{pmatrix*}
    &
    \xlongrightarrow{\text{I $\leftrightarrow$ II}}
    &&
    \begin{pmatrix*}[r]
      1 & -1 & 0  \\
      2 &  1 & 3  \\
      3 &  0 & 2
    \end{pmatrix*}
    &&
    \to
    &&
    \begin{pmatrix*}[r]
      -1 & 1 & 0  \\
       1 & 2 & 3  \\
       0 & 3 & 2
    \end{pmatrix*}
    \\
    &
    \xlongrightarrow{\text{II} + \text{I}}
    &&
    \begin{pmatrix*}[r]
      -1 & 1 & 0  \\
       0 & 3 & 3  \\
       0 & 3 & 2
    \end{pmatrix*}
    &&
    \to
    &&
    \begin{pmatrix*}[r]
      -1 & 0 & 0  \\
       0 & 3 & 3  \\
       0 & 3 & 2
    \end{pmatrix*}
    \\
    &
    \xlongrightarrow{\text{III} - \text{II}}
    &&
    \begin{pmatrix*}[r]
      -1 & 0 &  0 \\
       0 & 3 &  3 \\
       0 & 0 & -1
    \end{pmatrix*}
    &&
    \to
    &&
    \begin{pmatrix*}[r]
      -1 & 0 &  0 \\
       0 & 3 &  0 \\
       0 & 0 & -1
    \end{pmatrix*}
    \eqqcolon D.
  \end{alignat*}
  Indem wir die entsprechenden Spaltenumformungen auf die Einheitsmatrix anwenden, erhalten wir eine Basiswechselmatrix $S \in \GL{3}{K}$.
  \[
    \begin{pmatrix*}[r]
      1 & 0 & 0 \\
      0 & 1 & 0 \\
      0 & 0 & 1
    \end{pmatrix*}
    \xlongrightarrow{\text{I $\leftrightarrow$ II}}
    \begin{pmatrix*}[r]
      0 & 1 & 0 \\
      1 & 0 & 0 \\
      0 & 0 & 1
    \end{pmatrix*}
    \xlongrightarrow{\text{II} + \text{I}}
    \begin{pmatrix*}[r]
      0 & 1 & 0 \\
      1 & 1 & 0 \\
      0 & 0 & 1
    \end{pmatrix*}
    \xlongrightarrow{\text{III} - \text{II}}
    \begin{pmatrix*}[r]
      0 & 1 & -1  \\
      1 & 1 & -1  \\
      0 & 0 &  1
    \end{pmatrix*}
    \eqqcolon
    S.
  \]
  Für die Matrizen $D$ und $S$ gilt nun, dass
  \[
      \transpose{S} A S
    = D.
  \]

\end{example}





\section{Reelle symmetrische Bilinearformen}

Für diesen Unterabschnitt sei $K = \Real$



\subsection{Sylvesterscher Trägheitssatz}

Es sei $\beta \in \SymForm(V)$ eine symmetrischen Bilinearform.

Ist $\basis{B} = (v_1, \dotsc, v_n)$ eine Orthogonalbasis von $V$ bezüglich $\beta$, so ist die darstellende Matrix $\repmatrixbilsame{\beta}{\basis{B}}$ von der Form
\begin{equation}
  \label{equation: bilinear form in diagonal form}
    \repmatrixbilsame{\beta}{\basis{B}}
  = \begin{pmatrix}
      d_1 &         &     \\
          & \ddots  &     \\
          &         & d_n
    \end{pmatrix},
\end{equation}
wobei $d_i = \beta(v_i, v_i)$ gilt.
Sind $\mu_1, \dotsc, \mu_n \in K$ mit $\mu_i \neq 0$, so ist auch $\basis{B}' = (w_1, \dotsc, w_n)$ mit $w_i = \mu_i v_i$ eine Basis von $V$, und es gilt
\[
  \repmatrixbilsame{\beta}{\basis{B'}}
  = \begin{pmatrix}
      \mu_1^2 d_1 &         &             \\
                  & \ddots  &             \\
                  &         & \mu_n^2 d_n
    \end{pmatrix}
\]
Somit lassen sich in \eqref{equation: bilinear form in diagonal form} die Diagonaleinträge um Quadratzahlen aus $\Real$ abändern.
Hiermit erhält man aus Satz~\ref{theorem: exstience of an orthogonal basis} den Sylvesterschen Trägheitssatz:

\begin{corollary}[Sylvesterscher Trägheitssatz]
  \label{corollary: Sylvester}
  Es sei $K = \Real$.
  Dann existiert eine Basis $\basis{B}$ von $V$, so dass
  \[
      \repmatrixbilsame{\beta}{\basis{B}}
    = \begin{pmatrix}
        \Id_p &         &     \\
              & -\Id_q  &     \\
              &         & 0_r
      \end{pmatrix}
  \]
  gilt.
  Die Zahlen $p$, $q$ und $r$ sind eindeutig bestimmt durch
  \begin{align*}
        p
    &=  \max  \{
                \dim U
              \suchthat
                \text{$U \subseteq V$ ist ein UVR, so dass $\beta|_{U \times U}$ positiv definit ist}
              \}
  \shortintertext{und}
        q
    &=  \max  \{
                \dim U
              \suchthat
                \text{$U \subseteq V$ ist ein UVR, so dass $\beta|_{U \times U}$ negativ definit ist}
              \},
  \end{align*}
  sowie durch $p + q + r = n$.
  Zudem gilt $\rank \beta = p + q$.
\end{corollary}

\begin{definition}
  In der Situation von Korollar~\ref{corollary: Sylvester} heißt die Zahl $p - q$ die \emph{Signatur} von $\beta$.
\end{definition}

Man bemerke, dass sich die Zahlen $p$, $q$ und $r$ aus dem Sylvesterschen Trägheitssatz bereits aus der Form \eqref{equation: bilinear form in diagonal form} ablesen lassen:
Die Zahl $p$ ist die Anzahl der positiven Diagonaleinträge, die Zahl $q$ ist die Anzahl der negativen Diagonaleinträge, und $r$ ist die Häufigkeit von $0$ als Diagonaleintrag.

\begin{example}
  Es gibt $10$ Kongruenzklassen von reellen symmetrischen $(3 \times 3)$-Matrizen.
  Ein Repräsentantensystem ist durch die folgenden Matrizen gegeben:
  \[
    \begin{matrix}
      \begin{pmatrix}
        1 &   &   \\
          & 1 &   \\
          &   & 1
      \end{pmatrix}
    &
      \begin{pmatrix}
        1 &   &     \\
          & 1 &     \\
          &   & -1
      \end{pmatrix}
    &
      \begin{pmatrix}
        1 &     &     \\
          & -1  &     \\
          &     & -1
      \end{pmatrix}
    &
      \begin{pmatrix}
        -1  &     &     \\
            & -1  &     \\
            &     & -1
      \end{pmatrix}
    &
      \begin{pmatrix}
        1 &   &   \\
          & 1 &   \\
          &   & 0
      \end{pmatrix}
    \\[20pt]
      \begin{pmatrix}
        1 &    &   \\
          & -1 &   \\
          &    & 0
      \end{pmatrix}
    &
      \begin{pmatrix}
        -1  &    &   \\
            & -1 &   \\
            &    & 0
      \end{pmatrix}
    &
      \begin{pmatrix}
        1 &   &   \\
          & 0 &   \\
          &   & 0
      \end{pmatrix}
    &
      \begin{pmatrix}
        -1  &   &   \\
            & 0 &   \\
            &   & 0
      \end{pmatrix}
    &
      \begin{pmatrix}
        0 &   &   \\
          & 0 &   \\
          &   & 0
      \end{pmatrix}
    \end{matrix}
  \]
  Allgemeiner gibt es für alle $n \geq 0$ genau $\binom{n+2}{2}$ Kongruenzklassen reeller symmetrischer $(n \times n)$-Matrizen.
\end{example}

Ist $\basis{B} = (v_1, \dotsc, v_n)$ eine Basis von $V$, so ist die darstellende $A \coloneqq \repmatrixbilsame{\beta}{\basis{B}}$ symmetrisch.
Nach Korollar~\ref{corollary: hermitian matrices are diagonalizable} gibt es eine Orthonormalbasis $\basis{D} = (v_1, \dotsc, v_n)$ von $\Real^n$ bestehend aus Eigenvektoren von $A$.
Es sei $\lambda_i$ der zu $v_i$ gehörige Eigenwert, und $S \coloneqq (v_1 \dotsc v_n)$.
Die Matrix $S$ ist orthogonal, und für die Basis $\basis{C}$ von $V$ mit $\repmatrixhom{\id_V}{\basis{C}}{\basis{B}} = S$ gilt
\[
    \repmatrixbilsame{\beta}{\basis{C}}
  = \transpose{ \repmatrixhom{\id_V}{\basis{C}}{\basis{B}} }
    \repmatrixbilsame{\beta}{\basis{B}}
    \repmatrixhom{\id_V}{\basis{C}}{\basis{B}}
  = \transpose{S} A S
  = S^{-1} A S
  = \begin{pmatrix}
      \lambda_1 &         &           \\
                & \ddots  &           \\
                &         & \lambda_n
    \end{pmatrix}.
\]

\begin{proposition}
  Es seien $p$, $q$ und $r$ die Zahlen aus dem Sylvesterscher Trägheitssatz für $\beta$.
  Dann ist $p$ die Anzahl der positiven Eigenwerte von $A$, $q$ die Anzahl der negativen Eigenwerte, und $r$ die Vielfachheit des Eigenwerts $0$.
\end{proposition}

\begin{lemma}[Hauptminorenkriterium]
  Eine symmetrische Matrix $A \in \matrices{n}{\Real}$ ist genau dann positiv, wenn alle führenden Hauptminoren positiv sind.
\end{lemma}





\section{Dualität}

Es sei $W$ ein weiterer endlichdimensionaler $K$-Vektorraum



\subsection{Nicht-ausgeartetet Bilinearformen}

Es sei $\beta \in \BilForm(V, W)$ eine Bilinearform.
Die Bilinearform $\beta$ entspricht einer linearen Abbildung
\[
          \beta_1
  \colon  W
  \to     \dual{V},
  \quad
          w
  \mapsto \beta(-,w),
\]
sowie einer linearen Abbildung
\[
          \beta_2
  \colon  V
  \to     \dual{W},
  \quad
          v
  \mapsto \beta(v,-).
\]
Ist $\basis{B} = (v_1, \dotsc, v_n)$ eine Basis von $V$ und $\basis{C} = (w_1, \dotsc, w_m)$ eine Basis von $W$, so gilt bezüglich der dualen Basen $\dual{\basis{B}}$ von $\dual{V}$ und $\dual{\basis{C}}$ on $\dual{W}$, dass
\[
    \repmatrixhom{\beta_1}{\basis{C}}{\dual{\basis{B}}}
  = \repmatrixbildiff{\beta}{\basis{B}}{\basis{C}}
  \quad\text{und}\quad
    \repmatrixhom{\beta_2}{\basis{B}}{\dual{\basis{C}}}
  = \transpose{ \repmatrixbildiff{\beta}{\basis{B}}{\basis{C}} }
\]
Insbesondere gelten die Äquivalenzen
\begin{align*}
      &\, \text{$\beta_1$ ist ein Isomorphismus}                                                  \\
  \iff&\, \text{$\repmatrixhom{\beta_1}{\basis{C}}{\dual{\basis{B}}}$ ist invertierbar}           \\
  \iff&\, \text{$\repmatrixbildiff{\beta}{\basis{B}}{\basis{C}}$ ist invertierbar}                \\
  \iff&\, \text{$\transpose{ \repmatrixbildiff{\beta}{\basis{B}}{\basis{C}} }$ ist invertierbar}  \\
  \iff&\, \text{$\repmatrixhom{\beta_2}{\basis{B}}{\dual{\basis{C}}}$ ist invertierbar}           \\
  \iff&\, \text{$\beta_2$ ist ein Isomorphismus}.
\end{align*}

\begin{definition}
  Eine Bilinearform $\beta \in \BilForm(V,W)$ heißt \emph{nicht-ausgeartet}, falls Sie eine \textup(und damit alle\textup) der folgenden äquivalenten Bedingungen erfüllt:
  \begin{enumerate}
    \item
      Die lineare Abbildung $\beta_1$ ist ein Isomorphismus.
    \item
      Die lineare Abbildung $\beta_2$ ist ein Isomorphismus.
    \item
      Es gibt eine Basis $\basis{B}$ von $V$ und Basis $\basis{C}$ von $W$, so dass die darstellende Matrix $\repmatrixbildiff{\beta}{\basis{B}}{\basis{C}}$ invertierbar ist.
    \item
      Für jede Basis $\basis{B}$ von $V$ und Basis $\basis{C}$ von $W$ ist die darstellende Matrix $\repmatrixbildiff{\beta}{\basis{B}}{\basis{C}}$ invertierbar.
    \item
      Es gelten \textup(mindestes\textup) zwei \textup(und damit bereits alle drei\textup) der folgenden Bedingungen:
      \begin{enumerate}
        \item
          Es gilt $\dim V = \dim W$.
        \item
          Für jedes $v \in V$ mit $v \neq 0$ gibt es ein $w \in W$ mit $\beta(v,w) \neq 0$ \textup(d.h.\ $\beta_2$ ist injektiv\textup).
        \item
          Für jedes $w \in W$ mit $w \neq 0$ gibt es ein $v \in V$ mit $\beta(v,w) \neq 0$ \textup(d.h.\ $\beta_1$ ist injektiv\textup).
      \end{enumerate}
  \end{enumerate}
\end{definition}

\begin{example}
  \leavevmode
  \begin{enumerate}
    \item
      Es sei $\beta \in \BilForm(V, \dual{V})$ die Bilinearform mit $\beta(v, \varphi) = \varphi(v)$.
      Die lineare Abbildung $\beta_1 \colon \dual{V} \to \dual{V}$ ist die Identität $\id_{\dual{V}}$, also ist $\beta$ nicht-ausgeartet.
      Die lineare Abbildung $\beta_2 \colon V \to \ddual{V}$ ist der natürliche Isomorphismus aus Lineare~Algebra~I.
    \item
      Ist $V$ ein euklidscher Vektorraum, so ist $\bil{-}{-}$ eine nicht-ausgeartete, symmetrische Bilinearform auf $V$, da die Abbildung $V \to \dual{V}$, $v \mapsto \bil{-}{v}$ nach dem Rieszschen Darstellungssatz ein Isomorphismus ist.
    \item
      Die symmetrische Bilinearform $\beta \in \BilForm(\matrices{n}{K})$ mit $\beta(A, B) = \tr(AB)$ ist nicht-ausgeartet:
      Es sei $(E_{ij})_{i,j=1}^n$ die Standardbasis von $\matrices{n}{K}$.
      Für alle $i,j,k,l$ gilt
      \[
          \beta(E_{ij}, E_{kl})
        = \tr(E_{ij} E_{kl})
        = \tr( \delta_{jk} E_{il} )
        = \delta_{jk} \tr E_{il}
        = \delta_{il} \delta_{jk}
        = \delta_{(i,l), (j,k)}.
      \]
      Für $A \in \matrices{n}{K}$ gilt deshalb
      \begin{align*}
          \beta(A, E_{ij})
        = \beta\left( \sum_{k,l=1}^n A_{kl} E_{kl}, E_{ij} \right)
        = \sum_{k,l=1}^n A_{kl} \underbrace{ \beta(E_{kl}, E_{ij}) }_{= \delta_{(i,l), (j,k)}}
        =  A_{ji}.
      \end{align*}
      Für $A \in \matrices{n}{K}$ mit $A \neq 0$ gilt deshalb $\beta(A, E_{ji}) \neq 0$.
  \end{enumerate}
\end{example}



% \subsection{Duale Basen}
% 
% Es sei $\beta \in \BilForm(V, W)$ eine nicht-ausgeartete Bilinearform.
% 
% \begin{proposition}
%   Für jede Basis $\basis{B} = (v_1, \dotsc, v_n)$ von $V$ gibt es eine eindeutige Basis $\basis{C} = (w_1, \dotsc, w_n)$ von $W$ mit $\beta(v_i, w_j) = \delta_{ij}$ für alle $i, j$, also so dass $\repmatrixbildiff{\beta}{\basis{B}}{\basis{C}} = \Id$.
%   Dies ist die zu $\basis{B}$ bezüglich $\beta$ duale Basis.
% \end{proposition}
% 
% \begin{example}
%   Es sei $\beta \in \BilForm(V, \dual{V})$ die nicht-ausgeartete Bilinearform aus Beispiel~\ref{example: non-degenerate bilinear form}, also mit $\beta(v, \varphi) = \varphi(v)$.
%   Ist $\basis{B} = (v_1, \dotsc, v_n)$ eine Basis von $V$, so ist $\dual{\basis{B}} = (\dual{v_1}, \dotsc, \dual{v_n})$ die bezüglich $\beta$ duale Basis zu $\basis{B}$.
% \end{example}



\subsection{Die Adjungierte Abbildung}

Es seien $\beta_V \in \BilForm(V,V')$ und $\beta_W \in \BilForm(W,W')$ zwei nicht-ausgeartete Bilinearformen.
Dann gibt es für jede lineare Abbildung $f \colon V \to W$ eine eindeutige lineare Abbildung $\adj{f} \colon W' \to V'$ mit
\begin{equation}
  \label{equation: definition of the adjoint map by elements for non degenarete bilinear forms}
    \beta_W(f(v), w')
  = \beta_V(v, \adj{f}(w'))
  \qquad
  \text{für alle $v \in V$, $w' \in W'$}
\end{equation}
gilt.
Dies ergibt sich wie bereits in Abschnitt \ref{section: adjoint map for scalar products} dadurch, dass \eqref{equation: definition of the adjoint map by elements for non degenarete bilinear forms} äquivalent zur Kommutativität des Diagramms
\[
  \begin{tikzcd}
      \dual{V}
    & \dual{W}
      \arrow[swap]{l}{\dual{f}}
    \\
      V'
      \arrow{u}{(\beta_V)_1}
    & W'
      \arrow[swap]{u}{(\beta_W)_1}
      \arrow{l}{\adj{f}}
  \end{tikzcd}
\]
ist.
Also ist $\adj{f}$ eindeutig als $\adj{f} = (\beta_V)_1^{-1} \circ \dual{f} \circ (\beta_W)_1$ bestimmt.

\begin{example}
  \leavevmode
  \begin{enumerate}
    \item
      Es seien $\beta_V \in \BilForm(V, \dual{V})$ und $\beta_W \in \BilForm(W, \dual{W})$ die nicht-ausgearteten Bilinearformen mit $\beta_V(v, \varphi) = \varphi(v)$ und $\beta_W(w, \psi) = \psi(w)$.
      Dann ist $\dual{f} \colon \dual{W} \to \dual{V}$ die zu $f \colon V \to W$ bezüglich $\beta_V$ und $\beta_W$ duale Abbildung, denn
      \[
          \beta_W(f(v), \psi)
        = \psi(f(v))
        = \dual{f}(\psi)(v)
        = \beta_V(v, \dual{f}(\psi)).
      \]
    \item
      Es seien $V$ und $W$ euklidische Vektorräume.
      Dann sind die zugehörigen Skalarprodukte $\bil{-}{-}_V$ und $\bil{-}{-}_W$ nicht-ausgeartete symmetrische Bilinearformen.
      Dann ist die zu $f \colon V \to W$ bezüglich $\beta_V$ und $\beta_W$ adjungierte Abbildung die adjungierte Abbildung $\adj{f} \colon W \to V$ aus Abschnitt~\ref{section: adjoint map for scalar products}.
  \end{enumerate}
\end{example}



\subsection{Nicht-ausgeartete symmetrische Bilinearformen}

Wir betrachten nun den Fall, dass $\beta \in \SymForm(V)$ eine symmetrische Bilinearform auf $V$ ist.
Für die beiden zugehörigen linearen Abbildungen $\beta_1, \beta_2 \colon V \to V^*$ gilt dann $\beta_1 = \beta_2$.

\begin{definition}
  Das \emph{Radikal} von $\beta$ ist der Untervektorraum
  \[
              \rad{\beta}
    \coloneqq \{v \in V \suchthat \text{$\beta(v,v') = 0$ für alle $v' \in V$}.
  \]
\end{definition}

Es gilt $\rad{\beta} = \ker \beta_1 = \ker \beta_2$, weshalb $\beta$ genau dann nicht ausgeartet ist, wenn $\rad{\beta} = 0$ gilt.

\begin{example}
  Es sei $\ringchar{K} \neq 2$.
  Dann gibt es eine Orthogonalbasis $\basis{B} = (v_1, \dotsc, v_n)$ von $V$ bezüglich $\beta$, so dass die darstellende Matrix $\repmatrixbilsame{\beta}{\basis{B}}$ von der Form
  \[
      \repmatrixbilsame{\beta}{\basis{B}}
    = \begin{pmatrix}
        \lambda_1 &         &           &   &         &   \\
                  & \ddots  &           &   &         &   \\
                  &         & \lambda_r &   &         &   \\
                  &         &           & 0 &         &   \\
                  &         &           &   & \ddots  &   \\
                  &         &           &   &         & 0
      \end{pmatrix}
  \]
  ist, wobei $\lambda_1, \dotsc, \lambda_r \neq 0$ gilt.
  Dann ist $\rank \beta = r$ und somit $\repmatrixbilsame{\beta}{\basis{B}}$ genau dann invertierbar, wenn $r = n$ gilt.
  
  Zusammen mit dem Sylvesterschen Trägheitssatz ergibt sich damit, dass es auf einem $n$-dimensionalen reellen Vektorraum bis auf Kongruenz genau $n+1$ nicht-ausgeartete symmetrische Bilinearformen gibt.
  
  Außerdem gilt in der obigen Situation, dass $(v_{r+1}, \dotsc, v_n)$ eine Basis von $\rad \beta$ ist.
\end{example}

\begin{lemma}
  Für jeden Untervektorraum $U \subseteq V$ gilt
  \begin{enumerate}
    \item
      $\dim U^\perp = \dim V - \dim U$ und
    \item
      $(U^\perp)^\perp = U$,
  \end{enumerate}
\end{lemma}

\begin{warning}
  Es gilt im Allgemeinen nicht, dass $V = U \oplus U^\perp$.
\end{warning}




