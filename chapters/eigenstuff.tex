\chapter{Diagonalisierbarkeit \& Verwandte}

Es sei $K$ ein Körper, $V$ ein endlichdimensionaler $K$-Vektorraum, und $f \colon V \to V$ ein Endomorphismus.





\section{Eigenwerte und Eigenvektoren}

\begin{definition}
  \leavevmode
  \begin{itemize}
    \item
      Sind $\lambda \in K$ und $v \in V$ mit $v \neq 0$ und $f(v) = \lambda v$, so ist $v$ ein \emph{Eigenvektor} von $f$ zum \emph{Eigenwert} $\lambda$.
      Für alle $\lambda \in K$ ist der Untervektorraum
      \[
                  \eigenspace{V}{f}{\lambda}
        \coloneqq \{ v \in V \suchthat f(v) = \lambda f \}
        =         \ker (f - \lambda \id_V)
      \]
      der \emph{Eigenraum} von $f$ zu $\lambda$.
      Die Zahl $\dim \eigenspace{V}{f}{\lambda}$ ist die \emph{geometrische Vielfachheit} von $\lambda$.
    \item
      Es sei $n \in \Natural$ und $A \in \matrices{n}{K}$.
      Sind $\lambda \in K$ und $x \in K^n$ mit $x \neq 0$ und $Ax = \lambda x$, so ist $x$ ein \emph{Eigenvektor} von $A$ zum \emph{Eigenwert} $\lambda$.
      Für alle $\lambda \in K$ ist der Untervektorraum
      \[
                  \eigenspace{(K^n)}{A}{\lambda}
        \coloneqq \{ x \in K^n \suchthat Ax = \lambda x \}
        =         \ker (A - \lambda \Id)
      \]
      der \emph{Eigenraum} von $A$ zu $\lambda$.
      Die Zahl $\dim \eigenspace{(K^n)}{A}{\lambda}$ ist die \emph{geometrische Vielfachheit} von $\lambda$.
  \end{itemize}
\end{definition}

\begin{remark}
  \leavevmode
  \begin{enumerate}
    \item
      Ist $A \in \matrices{n}{K}$ und $f \colon K^n \to K^n$, $x \mapsto Ax$ der zu $A$ (bezüglich der Standardbasis) gehörige Endomorphismus, so stimmen die Eigenvektoren, Eigenwerte und Eigenräume von $A$ mit denen von $f$ überein.
      
      Es genügt daher im Folgenden, Definitionen und Aussagen für Endomorphismen zu anzugeben -- für Matrizen gelten diese dann ebenfalls.
    \item
      Es sei $f \colon V \to V$ ein Endomorphismus eines $K$-Vektorraums $V$, $\basis{B} = (v_1, \dotsc, v_n)$ eine Basis von $V$ und $A \coloneqq \repmatrixendo{f}{\basis{B}}$ die entsprechende darstellende Matrix.
      Bezüglich des zu $\basis{B}$ zugehörigen Isomorphismus
      \begin{gather*}
                  \Phi_{\basis{B}}
        \colon    V
        \to       K^n,
        \quad       v
                  = \sum_{i=1}^n x_i v_i
        \mapsto   \transpose{(x_1, \dotsc, x_n)}
        \eqqcolon [v]_{\basis{B}}
      \shortintertext{gilt}
          \Phi_{\basis{B}}\left( \eigenspace{V}{f}{\lambda} \right)
        = \eigenspace{(K^n)}{A}{\lambda}.
      \end{gather*}
      Es ist also $v \in V$ genau dann ein Eigenvektor von $f$ zum Eigenwert $\lambda \in K$, wenn $[v]_{\basis{B}}$ ein Eigenvektor von $A$ zum Eigenwert $\lambda$ ist.
      
      Berechnungen lassen sich deshalb in Matrizenform durchführen.
  \end{enumerate}
\end{remark}

Für theoretische Aussagen nutzen wir also Endomorphismen, und für konkrete Rechnungen nutzen wir Matrizen.





\section{Das charakterische Polynom}

Es sei $\basis{B}$ eine Basis von $V$ und $A \coloneqq \repmatrixendo{f}{\basis{B}}$ die zugehörige darstellende Matrix von $f$.
Dann gilt
\begin{align*}
      &\, \text{$\lambda$ ist ein Eigenwert von $f$}      \\
  \iff&\, \text{$\lambda$ ist ein Eigenwert von $A$}      \\
  \iff&\, \eigenspace{(K^n)}{A}{\lambda} \neq 0           \\
  \iff&\, \ker (A - \lambda \Id) \neq 0                   \\
  \iff&\, \text{$A - \lambda \Id$ ist nicht invertierbar} \\
  \iff&\, \det (A - \lambda \Id) = 0.
\end{align*}

\begin{definition}
  Das \emph{charakterische Polynom} von $A$ ist definiert als
  \[
              \charpol{A}(t)
    \coloneqq \det ( A - t \Id )
    \in       K[t].
  \]
  Das charakteristische Polynom von $f$ ist definiert als $\charpol{f}(t) \coloneqq \charpol{A}(t)$.
\end{definition}

Dass das charakteristische Polynom $\charpol{f}(t)$ wohldefiniert ist, also nicht von der Wahl der Basis $\basis{B}$ abhängt, folgt aus dem folgenden Lemma:

\begin{lemma}
  Ähnliche Endomorphismen, bzw.\ Matrizen haben das gleiche charakterische Polynom.
\end{lemma}

Aus unserer anfänglichen Beobachtung erhalten wir den folgenden Zusammenhang zwischen den Eigenwerten und dem charakteristischen Polynom von $f$:

\begin{proposition}
  Die Eigenwerte von $f$ sind genau die Nullstellen des charakteristischen Polynoms $\charpol{f}(t)$.
  Die Vielfachheit des Linearfaktors $t - \lambda$ in $\charpol{f}(t)$ ist die \emph{algebraische Vielfachheit} von $\lambda$.
\end{proposition}

\begin{example}
  \leavevmode
  \begin{enumerate}
    \item
      Für die Matrix
      \[
          A
        = \begin{pmatrix}
            1 & 2 & 3 \\
            3 & 1 & 2 \\
            2 & 3 & 1
          \end{pmatrix}
        \in \matrices{3}{\Real}
      \]
      gilt
      \[
          \charpol{A}(t)
        = \det
          \begin{pmatrix}
            1-t & 2   & 3   \\
            3   & 1-t & 2   \\
            2   & 3   & 1-t
          \end{pmatrix}
        = -t^3 + 3t^2 + 15t + 18.
      \]
      Die einzige reelle Nullstelle von $\charpol{A}(t)$, und somit der einzige reelle Eigenwert von $A$, ist $6$.
      Der entsprechende Eigenraum ist
      \[
          \eigenspace{(\Real^3)}{A}{6}
        = \ker (A - 6 \Id)
        = \ker \begin{pmatrix*}[r]
                -5  &  2  &  3 \\
                 3  & -5  &  2 \\
                 2  &  3  & -5
              \end{pmatrix*}
        = \generated{ \vect{1 \\ 1 \\ 1} }.
      \]
    \item
      Es gilt
      $
          \charpol{A}(t)
        = \det (A - t \Id)
        = \det \transpose{(A - t \Id)}
        = \det (\transpose{A} - t \Id)
        = \charpol{\transpose{A}}(t)
      $.
    \item
      Ist $A$ eine obere Dreiecksmatrix
      \[
          A
        = \begin{pmatrix}
            \lambda_1 & \cdots  & *         \\
                      & \ddots  & \vdots    \\
                      &         & \lambda_n
          \end{pmatrix},
      \]
      so gilt
      \begin{align*}
            \charpol{A}(t)
        &=  \det
            \begin{pmatrix}
              \lambda_1 - t & \cdots  & *             \\
                            & \ddots  & \vdots        \\
                            &         & \lambda_n - t
            \end{pmatrix}
        \\
        &=  (\lambda_1 - t) \dotsm (\lambda_n - t)
         =  (-1)^n (t - \lambda_1) \dotsm (t - \lambda_n).
      \end{align*}
      Die Eigenwerte von $A$ sind also genau die Diagonaleinträge $\lambda_1, \dotsc, \lambda_n$.
    \item
      Ist allgemeiner $A$ eine obere Blockdreiecksmatrix
      \[
          A
        = \begin{pmatrix}
            A_1 & \cdots  & *       \\
                & \ddots  & \vdots  \\
                &         & A_r
          \end{pmatrix}
      \]
      mit $A_i \in \matrices{n_i}{K}$, so gilt
      \begin{align*}
            \charpol{A}(t)
        &=  \det
            \begin{pmatrix}
              A_1 - t \Id_{n_1} & \cdots  & *                 \\
                                & \ddots  & \vdots            \\
                                &         & A_r - t \Id_{n_r}
            \end{pmatrix}
        \\
        &=  \det (A_1 - t \Id_{n_1}) \dotsm \det(A_r - t \Id_{n_r})
         =  \charpol{A_1}(t) \dotsm \charpol{A_r}(t).
      \end{align*}
  \end{enumerate}
\end{example}

Allgemein ist das charakteristische Polynom von $A \in \matrices{n}{K}$ von der Form
\[
    \charpol{A}(t)
  = (-1)^n t^n + (-1)^{n-1} (\tr A) t^{n-1} + \dotsb + \det A.
\]
\begin{example}
  Das charakteristische Polynom einer ($2 \times 2$)-Matrix $A = \begin{psmallmatrix} a & b \\ c & d \end{psmallmatrix} \in \matrices{2}{K}$ ist gegeben durch
  \[
      \charpol{A}(t)
    = t^2 - (\tr A) t + \det A
    = t^2 - (a + d) t + (ad - bc).
  \]
\end{example}





\section{Das Minimalpolynom}

\begin{lemma}
  \label{lemma: polynomial equations give restriction for the eigenvalues}
  Ist $v \in V$ ein Eigenvektor von $f$ zum Eigenwert $\lambda \in K$, so ist $v$ für jedes Polynom $p \in K[t]$ ein Eigenvektor von $p(f)$ zum Eigenwert $p(\lambda)$.
  Inbesondere sind die Eigenwerte von $f$ Nullstellen von $p$, falls $p(f) = 0$ gilt.
\end{lemma}

\begin{example}
  \leavevmode
  \begin{enumerate}
    \item
      Ist $f$ nilpotent mit $f^k = 0$, so gilt $q(f) = 0$ für das Polynom $q(t) \coloneqq t^k$.
      Deshalb ist $0$ der einzige mögliche Eigenwert von $f$.
    \item
      Gilt $f^3 = 3f - 2 \id_V$, so gilt $q(f) = 0$ für das Polynom $q(t) \coloneqq t^3 - 3t + 2$.
      Es gilt $q(t) = (t-1)^2 (t+2)$, also sind $1$ und $-2$ die einzigen möglichen Eigenwerte für $f$.
    \item
      Gilt $f^2 = -\id_V$, so hat $f$ im Fall $K = \Real$ keine Eigenwerte.
  \end{enumerate}
\end{example}

\begin{definition}
  \leavevmode
  \begin{itemize}
    \item
      Es sei
      \[
                  \vanishing{f}
        \coloneqq \{ p \in K[t] \suchthat p(f) = 0 \}.
      \]
      Das eindeutige normierte, von $0$ verschiedene Polynom minimalen Grades aus $\vanishing{f}$ ist das \emph{Minimalpolynom} von $f$, und wird mit $\minpol{f}(t) \in K[t]$ notiert.
    \item
      Für $A \in \matrices{n}{K}$ sei
      \[
                  \vanishing{A}
        \coloneqq \{ p \in K[t] \suchthat p(A) = 0 \}.
      \]
      Das eindeutige normierte, von $0$ verschiedene Polynom minimalen Grades aus $\vanishing{A}$ ist das \emph{Minimalpolynom} von $A$, und wird mit $\minpol{A}(t) \in K[t]$ notiert.
  \end{itemize}
\end{definition}

\begin{remark}
  Die Wohldefiniertheit von $\vanishing{f}$ nutzt die Endlichdimensionalität von $V$.
  Hierdurch wird sichergestellt, dass $\vanishing{f} \neq 0$ gilt.
\end{remark}

\begin{lemma}
  Ist $\basis{B} = (v_1, \dotsc, v_n)$ eine Basis von $V$, so gilt für $A = \repmatrixendo{f}{\basis{B}}$, dass $\vanishing{f} = \vanishing{A}$.
  Somit gilt $\minpol{f} = \minpol{A}$.
\end{lemma}

\begin{corollary}
  Ähnliche Endomorphismen, bzw.\ Matrizen haben das gleiche charakteristische Polynom.
\end{corollary}

Die Definition des Minimalpolynoms lässt sich bis auf Normiertheit wie folgt umschreiben:

\begin{lemma}
  \label{lemma: properties of the minimal polynomial}
  Es gilt
  \[
      \vanishing{f}
    = \{ p \cdot \minpol{f} \suchthat p \in K[t] \}.
  \]
  Für $p \in K[t]$ gilt also genau dann $p(f) = 0$, wenn $\minpol{f} \divides p$.
  Inbesondere gilt $\minpol{f}(f) = 0$.
\end{lemma}

\begin{theorem}[Cayley--Hamilton]
  Es gilt $\charpol{f}(f) = 0$, also $\minpol{f} \divides \charpol{f}$.
\end{theorem}

Nach dem Satz von Cayley--Hamilton ist jede Nullstelle von $\minpol{f}(t)$ auch eine Nullstelle von $\charpol{f}(t)$, also ein Eigenwert von $f$.
Andererseits ist jeder Eigenwert von $f$ nach Lemma~\ref{lemma: properties of the minimal polynomial} und Lemma~\ref{lemma: polynomial equations give restriction for the eigenvalues} auch eine Nullstelle von $\minpol{f}(t)$.
Somit sind die Nullstelle non $\minpol{f}(t)$ genau die Eigenwerte von $f$.
Also haben $\charpol{f}(t)$ und $\minpol{f}(t)$ die gleichen Nullstellen.

Ist $K$ algebraisch abgeschlossen, so zerfallen $\charpol{f}(t)$ und $\minpol{f}(t)$ somit in die gleichen Linearfaktoren.
Die Vielfachheit im Minimalpolynom ist dabei nach dem Satz von Cayley--Hamilton jeweils kleiner ist als die Vielfachheit im charakteristischen Polynom.





\section{Diagonalisierbarkeit}

\begin{lemma}
  \label{lemma: linear independence of eigenvectors}
  Es seien $v_1, \dotsc, v_n \in V$ Eigenvektoren von $f$ zu paarweise verschiedenen Eigenwerten, d.h.\ es gelte $f(v_i) = \lambda_i v_i$ mit $\lambda_i \neq \lambda_j$ für $i \neq j$.
  Dann sind $v_1, \dotsc, v_n$ linear unabhängig.
  Inbesondere ist die Summe $\sum_{\lambda \in K} \eigenspace{V}{f}{\lambda}$ direkt.
\end{lemma}

\begin{definition}
  Der Endomorphismus $f \colon V \to V$ heiß \emph{diagonalisierbar} falls er die folgenden, äquivalenten Bedingungen erfüllt:
  \begin{enumerate}[label = \arabic*.]
    \item
      Es gilt $V = \bigoplus_{\lambda \in K} \eigenspace{V}{f}{\lambda}$.
    \item
      Es gilt $V = \sum_{\lambda \in K} \eigenspace{V}{f}{\lambda}$.
    \item
      Es gilt $\dim V = \sum_{\lambda \in K} \dim \eigenspace{V}{f}{\lambda}$.
    \item
      Es gibt eine Basis von $V$ bestehend aus Eigenvektoren von $f$.
    \item
      Es gibt ein Erzeugendensystem von $V$ bestehend aus Eigenvektoren von $f$.
    \item
      Es gibt eine Basis $\basis{B}$ von $V$, so dass die darstellende Matrix $\repmatrixendo{f}{\basis{B}}$ eine Diagonalmatrix ist.
  \end{enumerate}
  Eine Matrix $A \in \matrices{n}{K}$ heißt \emph{diagonalisierbar}, falls sie eine der folgenden, äquivalenten Bedingungen erfüllt:
  \begin{enumerate}[label = \arabic*.]
    \item
      Es gilt $K^n = \bigoplus_{\lambda \in K} \eigenspace{(K^n)}{A}{\lambda}$.
    \item
      Es gilt $K^n = \sum_{\lambda \in K} \eigenspace{(K^n)}{A}{\lambda}$.
    \item
      Es gilt $n = \sum_{\lambda \in K} \dim \left(\eigenspace{(K^n)}{A}{\lambda} \right)$.
    \item
      Es gibt eine Basis von $K^n$ bestehend aus Eigenvektoren von $A$.
    \item
      Es gibt ein Erzeugendensystem von $K^n$ bestehend aus Eigenvektoren von $A$.
    \item
      Die Matrix $A$ ist ähnlich zu einer Diagonalmatrix ist, d.h.\ es gibt $S \in \GL{n}{K}$, so dass $S A S^{-1}$ eine Diagonalmatrix ist.
  \end{enumerate}
\end{definition}

\begin{example}
  \leavevmode
  \begin{enumerate}
    \item
      Falls das charakteristische Polynom $\charpol{f}(t)$ in paarweise verschieden Linearfaktoren $\charpol{f}(t) = (t - \lambda_1) \dotsm (t - \lambda_n)$ zerfällt, so ist $f$ diagonalisierbar:
      
      Dann gibt es zu jedem Eigenwert $\lambda_i$ einen Eigenvektor $v_i \in \eigenspace{V}{f}{\lambda_i}$, und die Familie $\basis{B} = (v_1, \dotsc, v_n)$ ist nach Lemma~\ref{lemma: linear independence of eigenvectors} linear unabhängig.
      Da $n = \dim V$ gilt, ist $\basis{B}$ eine Basis von $V$.
      
    \item
      Es sei $\ringchar{K} \neq 2$, $V$ ein zweidimensionaler $K$-Vektorraum mit Basis $\basis{B} = (v_1, v_2)$ und $f \colon V \to V$ der eindeutige Endomorphismus mit $f(v_1) = v_2$ und $f(v_2) = v_1$.
      Dann ist $v_1 + v_2$ ein Eigenvektor zum Eigenwert $1$ und $v_1 - v_2$ ein Eigenvektor zum Eigenwert $-1$, und somit $\basis{B} = (v_1 + v_2, v_1 - v_2)$  eine Basis von $V$ aus Eigenvektoren von $f$.
      Also ist $f$ diagonalisierbar.
      
    \item
      Jede reelle, symmetrische Matrix ist diagonalisierbar.
  \end{enumerate}
\end{example}

\begin{lemma}
  Die folgenden Bedingungen sind äquivalent:
  \begin{enumerate}
    \item
      Der Endomorphismus $f$ ist diagonlisierbar.
    \item
      Es gibt eine Basis $\basis{B}$ von $V$ gibt, so dass $\repmatrixendo{f}{\basis{B}}$ diagonalisierbar ist.
    \item
      Für jede Basis $\basis{B}$ von $V$ ist $\repmatrixendo{f}{\basis{B}}$ diagonalisierbar.
  \end{enumerate}
\end{lemma}

\begin{example}
  Das charakteristische Polynom der Matrix
  \[
              A
    \coloneqq \begin{pmatrix*}[r]
                3 & -2  & 0 \\
                1 &  0  & 0 \\
                0 &  0  & 1
              \end{pmatrix*}
    \in       \matrices{3}{\Real}
  \]
  ist
  \[
      \charpol{A}(t)
    = t^3 - 4 t^2 + 5t - 2
    = - (t^2 - 3t + 2)(t-1)
    = - (t-1)^2 (t-2).
  \]
  Die Eigenwerte von $A$ sind also $1$ und $2$.
  Die zugehörigen Eigenräume sind
  \begin{gather*}
      \eigenspace{(\Real^3)}{A}{1}
    = \ker (A - \Id)
    = \ker \begin{pmatrix*}[r]
            2 & -2  & 0 \\
            1 & -1  & 0 \\
            0 &  0  & 0
          \end{pmatrix*}
    = \generated{ \vect{1 \\ 1 \\ 0}, \vect{0 \\ 0 \\ 1} }
  \shortintertext{und}
      \eigenspace{(\Real^3)}{A}{2}
    = \ker (A - \Id)
    = \ker \begin{pmatrix*}[r]
            1 & -2  &  0 \\
            1 & -2  &  0 \\
            0 &  0  & -1
          \end{pmatrix*}
    = \generated{ \vect{2 \\ 1 \\ 0} }.
  \end{gather*}
  Also ist
  \[
      \basis{B}
    = \left(
        \vect{1 \\ 1 \\ 0},
        \vect{0 \\ 0 \\ 1},
        \vect{2 \\ 1 \\ 0}
      \right)
  \]
  eine Basis von $\Real^3$ aus Eigenvektoren von $A$.
  Somit ist $A$ diagonalisierbar.
  Für die entsprechende Basiswechselmatrix
  \[
              S
    \coloneqq \begin{pmatrix}
                1 & 0 & 2 \\
                1 & 0 & 1 \\
                0 & 1 & 0
              \end{pmatrix}
    \in       \GL{3}{\Real}
    \quad\text{gilt}\quad
    S^{-1} A S
    = \begin{pmatrix}
        1 &   &   \\
          & 1 &   \\
          &   & 2
      \end{pmatrix}.
  \]
\end{example}

Ob der Endomorphismus $f \colon V \to V$ diagonalisierbar ist, hängt nur vom Minimalpolynom $\minpol{f}(t)$ ab:

\begin{proposition}
  Der Endomorphismus $f$ ist genau dann diagonalisierbar, falls das Minimalpolynom $\minpol{f}$ in paarweise verschiedene Linearfaktoren zerfällt.
\end{proposition}

Die Nullstellen von $\minpol{f}$ sind dann, wie bereits oben gesehen, genau die Eigenwerte von $f$.

\begin{example}
  \leavevmode
  \begin{enumerate}
    \item
      Gilt $f^2 = f$, so gilt $q(f) = 0$ für das Polynom $q(t) \coloneqq t^2 - t = t(t-1)$.
      Da $\minpol{f} \divides q$ gilt, zerällt somit $\minpol{f}$ in die möglichen Linearfaktoren $t$ und $t-1$.
      Somit ist $f$ diagonalisierbar mit möglichen Eigenwerten $0$ und $1$.
    \item
      Gilt $f^2 = \id_V$, so gilt $q(f) = 0$ für das Polynom $q(t) = t^2 - 1 = (t+1)(t-1)$.
      Gilt $\ringchar{K} \neq 2$, so ist $f$ somit diagonalisierbar mit möglichen Eigenwerten $1$ und $-1$.
  \end{enumerate}
\end{example}

Ist $U \subseteq V$ ein $f$-invarianter Untervektorraum, so gilt für die den eingeschränkten Endomorphismus $f|_U \colon U \to U$, dass $\minpol{f}(f|_U) = \minpol{f}(f)|_U = 0$, und somit dass $\minpol{f|_U} \divides \minpol{f}$.

\begin{corollary}
  Ist $f$ diagonalisierbar, so ist auch $f|_U$ diagonalisierbar.
\end{corollary}





\section{Trigonalisierbarkeit}



\subsection{Definition von Trigonalisierbarkeit}

\begin{definition}
  Der Endomorphismus $f \colon V \to V$ heißt trigonalisierbar, falls es eine Basis $\basis{B}$ von $V$ gibt, so dass $\repmatrixendo{f}{\basis{B}}$ eine obere Dreiecksmatrix ist.
  
  Eine Matrix $A \in \matrices{n}{K}$ heißt trigonalisierbar, falls $A$ ähnlich zu einer oberen Dreiecksmatrix ist, d.h.\ falls es $S \in \GL{n}{K}$ gibt, so dass $S A S^{-1}$ eine obere Dreiecksmatrix ist.
\end{definition}

\begin{lemma}
  Die folgenden Bedingungen äquivalent:
  \begin{enumerate}
    \item
      Der Endomorphismus $f$ ist trigonalisierbar.
    \item
      Es gibt eine Basis $\basis{B}$ von $V$, so dass die Matrix $\repmatrixendo{f}{\basis{B}}$ trigonalisierbar ist.
    \item
      Für jede Basis $\basis{B}$ von $V$ ist die Matrix $\repmatrixendo{f}{\basis{B}}$ trigonalisierbar.
  \end{enumerate}
\end{lemma}



\subsection{Äquivalente Charakterisierungen}

\begin{definition}
  Eine \emph{Flagge} von $V$ ist eine aufsteigende Kette von Untervektorräumen
  \[
                0
    =           V_0
    \subsetneq  V_1
    \subsetneq  V_2
    \subsetneq  \dotsb
    \subsetneq  V_n
    =           V
  \]
  mit $\dim V_k = k$ für alle $k$.
  \textup(Inbesondere ist $n = \dim V$\textup).
  Die Flagge heißt \emph{$f$-stabil} wenn $f(V_k) \subseteq V_k$ für alle $k$ gilt.
\end{definition}

Ist $\basis{B} = (v_1, \dotsc, v_n)$ eine Basis von $V$ mit $\repmatrixendo{f}{\basis{B}}$ in oberer Dreiecksform, so ist
\[
              0
  \subsetneq  \generated{v_1}
  \subsetneq  \generated{v_1, v_2}
  \subsetneq  \dotsb
  \subsetneq  \generated{v_1, \dotsc, v_n}
  =           V
\]
eine $f$-invariante Flagge von $V$.
Ist andererseits
\begin{equation}
  \label{equation: invariant flag}
              0
  =           V_0
  \subsetneq  V_1
  \subsetneq  V_2
  \subsetneq  \dotsb
  \subsetneq  V_n
  =           V
\end{equation}
eine Flagge von $V$, so ergibt sich durch Basisergänzung eine Basis $\basis{B} = (v_1, \dotsc, v_n)$ von $V$, so dass $V_k = \generated{v_1, \dotsc, v_k}$ für alle $k$ gilt.
Ist die Flagge \eqref{equation: invariant flag} $f$-invariant, so ist die darstellende Matrix $\repmatrixendo{f}{\basis{B}}$ eine obere Dreiecksmatrix.

Hierdurch ergibt sich die folgende Charakterisierung triagonalisierbarer Endomorphismen:

\begin{theorem}
  \label{theorem: characterization of triagonalizable endomorphisms}
  Die folgenden Bedingungen sind äquivalent:
  \begin{enumerate}
    \item
      Der Endomorphismus $f$ ist trigonalisierbar.
    \item
      Es gibt eine $f$-invariante Flagge von $V$.
    \item
      Das charakteristische Polynom $\charpol{f}(t)$ zerfällt in Linearfaktoren.
  \end{enumerate}
\end{theorem}

Inbesondere hängt die Trigonalisierbarkeit von $f$ nur vom charakteristischen Polynom $\charpol{f}(t)$ ab.
Außerdem ist jeder Endomorphismus von $V$ trigonalisierbar, falls $K$ algebraisch abgeschlossen ist.



\section{Spur und Determinante durch Eigenwerte}
\label{section: trace and determinant via eigenvalues}

Zerfällt $\charpol{f}(t)$ in Linearfaktoren $\charpol{f}(t) = (t - \lambda_1) \dotsm (t - \lambda_n)$, so ist $f$ nach Satz~\ref{theorem: characterization of triagonalizable endomorphisms} trigonalisierbar.
Dann gilt
\[
    \tr f
  = \lambda_1 + \dotsb + \lambda_n
\quad\text{und}\quad
    \det f
  = \lambda_1 \dotsm \lambda_n.
\]
Außerdem ist für jedes Polynom $q \in K[t]$ auch der Endomorphismus $q(f)$ trigonalisierbar, und es gilt
\[
    \charpol{q(f)}(t)
  = (t - q(\lambda_1)) \dotsm (t - q(\lambda_n)).
\]





\section{Simultane Diagonalisierbarkeit}

\begin{definition}
  Eine Kollektion $f_i$, $i \in I$ von Endomorphismen $f_i \colon V \to V$ heißt \emph{simultan diagonalisierbar}, falls es eine Basis $\basis{B}$ von $V$ gibt, so dass $\repmatrixendo{f}{\basis{B}}$ für jedes $i \in I$ eine Diagonalmatrix ist.
  
  Eine Kollektion $A_i$, $i \in I$ von Matrizen $A_i \in \matrices{n}{K}$ heißt \emph{simultan diagonalisierbar}, falls es $S \in \GL{n}{K}$ gibt, so dass $S A_i S^{-1}$ für jedes $i \in I$ eine Diagonalmatrix ist.
\end{definition}

\begin{lemma}
  Für eine Kollektion $f_i$, $i \in I$ von Endomorphismen $f_i \colon V \to V$ sind die folgenden Bedingungen äquivalent:
  \begin{enumerate}
    \item
      Die Endomorphismen $f_i$, $i \in I$ sind simultan diagonalisierbar.
    \item
      Es gibt eine Basis $\basis{B}$ von $V$, so dass die Matrizen $\repmatrixendo{f_i}{\basis{B}}$, $i \in I$ simultan diagonalisierbar sind.
    \item
      Für jede Basis $\basis{B}$ von $V$ sind die Matrizen $\repmatrixendo{f_i}{\basis{B}}$, $i \in I$ simultan diagonalisierbar.
  \end{enumerate}
\end{lemma}

\begin{example}
  Sind die Endomorphismen $f, g \colon V \to V$ simultan diagonalisierbar, so sind auch $f + g$ und $f \circ g$ diagonalisierbar.
\end{example}

\begin{proposition}
  Für Endomorphismen $f_1, \dotsc, f_n \colon V \to V$ sind die folgenden Bedingungen äquivalent:
  \begin{enumerate}
    \item
      Die Endomorphismen $f_1, \dotsc, f_n$ sind simultan diagonalisierbar.
    \item
      Die Endomorphismen $f_1, \dotsc, f_n$ sind diagonalisierbar und paarweise kommutierend, d.h.\ es gilt $f_i \circ  f_j = f_j \circ f_i$ für alle $i, j$.
  \end{enumerate}
\end{proposition}