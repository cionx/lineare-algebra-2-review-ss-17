\chapter{Die Jordan-Normalform}

Es sei $K$ ein Körper, $V$ ein endlichdimensionaler $K$-Vektorraum und $f \colon V \to V$ ein Endomorphismus.





\section{Definition}

\begin{definition}
  Für alle $n \in \Natural$ und $\lambda \in K$ ist
  \[
              J_n(\lambda)
    \coloneqq \begin{pmatrix}
                \lambda &         &         &         \\
                1       & \ddots  &         &         \\
                        & \ddots  & \ddots  &         \\
                        &         & 1       & \lambda
              \end{pmatrix}
    \in       \matrices{n}{K}
  \]
  der \emph{Jordanblock} zu $\lambda$ von Größe $n$.
\end{definition}

\begin{definition}
  Eine Matrix $J \in \matrices{n}{K}$ der Form
  \[
      J
    = \begin{pmatrix}
        J_{n_1}(\lambda_1)  &         &                     \\
                            & \ddots  &                     \\
                            &         & J_{n_t}(\lambda_t)
      \end{pmatrix}
  \]
  ist in \emph{Jordan-Normalform}.
\end{definition}

\begin{definition}
  \leavevmode
  \begin{enumerate}
    \item
      Eine \emph{Jordan-Normalform} einer Matrix $A \in \matrices{n}{K}$ ist eine zu $A$ ähnliche Matrix $J \in \matrices{n}{K}$, so dass $J$ in Jordan-Normalform ist.
    \item
      Eine Jordan-Normalform von $f$ ist eine Jordan-Normalform der darstellenden Matrix $\repmatrixendo{f}{\basis{B}}$ bezüglich einer Basis $\basis{B}$ von $V$.
  \end{enumerate}
\end{definition}

Der Endomorphismus $f$ besitzt genau dann eine Jordan-Normalform $J \in \matrices{n}{K}$, falls es eine Basis $\basis{B}$ von $V$ gibt, so dass $\repmatrixendo{f}{\basis{B}} = J$ gilt.
Wir bezeichnen eine solche Basis als \emph{Jordanbasis} von $f$.

Eine \emph{Jordanbasis} $\basis{B} = (v_1, \dotsc, v_n)$ einer Matrix $A \in \matrices{n}{K}$ ist eine Jordanbasis der zu $A$ (bezüglich der Standardbasis) gehörigen linearen Abbildung $f_A \colon K^n \to K^n$, $x \mapsto Ax$.
Dies ist äquivalent dazu, dass für die Matrix $S = (v_1 \dotsc v_n) \in \GL{n}{K}$ die Matrix $S^{-1} A S = \repmatrixendo{f_A}{\basis{B}}$ in Jordan-Normalform ist.





\section{Eindeutigkeit}

Es sei $J \in \matrices{n}{K}$ eine Matrix in Jordan-Normalform, also
\[
    J
  = \begin{pmatrix}
      J_{n_1}(\lambda_1)  &         &                     \\
                          & \ddots  &                     \\
                          &         & J_{n_t}(\lambda_t)
    \end{pmatrix}.
\]
Für alle $\lambda \in K$ gilt dann
\[
    \dim \ker (J - \lambda \Id)^k
  = \sum_{k'=1}^k \text{Anzahl der Jordanblöcke zu $\lambda$ von Größe mindestens $k'$}.
\]
Für die Zahlen $d_k(\lambda) \coloneqq \dim \ker (J - \lambda \Id)^k$ gilt deshalb
\[
    d_k(\lambda) - d_{k-1}(\lambda)
  = \text{Anzahl der Jordanblöcke zu $\lambda$ von Größe mindestens $k$}
\]
und somit
\begin{align*}
     &\,    \text{Anzahl der Jordanblöcke zu $\lambda$ von Größe $k$}                 \\
    =&\,    \text{Anzahl der Jordanblöcke zu $\lambda$ von Größe mindestens $k$}      \\
     &\,  - \text{Anzahl der Jordanblöcke zu $\lambda$ von Größe mindestens $(k+1)$}  \\
    =&\,    ( d_k(\lambda) - d_{k-1}(\lambda) )
          - ( d_{k+1}(\lambda) - d_k(\lambda) )                                       \\
    =&\,  2 d_k(\lambda) - d_{k-1}(\lambda) - d_{k+1}(\lambda).
\end{align*}

Ist $J$ Jordan-Normalform von $A \in \matrices{n}{K}$, so sind $A$ und $J$ ähnlich, weshalb für alle $\lambda \in K$ und $k \geq 0$ auch $(A - \lambda \Id)^k$ und $(J - \lambda \Id)^k$ ähnlich sind.
Für alle $\lambda \in K$ und $k \geq 0$ gilt deshalb $\dim \ker (A - \lambda)^k = \dim \ker (J - \lambda)^k$.
Aus der obigen Berechnung ergibt sich deshalb für die Zahlen $d_k(\lambda) \coloneqq \ker (A - \lambda \Id)^k$, dass
\begin{align*}
   &\, \text{Anzahl der Jordanblöcke zu $\lambda$ von Größe $k$ in einer JNF von $A$} \\
  =&\, 2 d_k(\lambda) - d_{k-1}(\lambda) - d_{k+1}(\lambda).
\end{align*}
Damit ergibt sich inbesondere die folgende Eindeutigkeit der Jordan-Normalform:

\begin{proposition}
  Je zwei Jordan-Normalformen einer Matrix, bzw.\ eines Endomorphismus stimmen bis auf Permutation der Jordanblöcke überein.
\end{proposition}

Es ergibt daher Sinn, von \emph{der} Jordan-Normalform einer Matrix, bzw.\ eines Endomorphismus zu sprechen.


% manual pagebreak
\pagebreak



\section{Existenz \& Kochrezept}



\subsection*{Existenzkriterien}

\begin{definition}
  \leavevmode
  \begin{itemize}
    \item
      Für alle $\lambda \in K$ und $k \geq 0$ sei
      \[
          \heigenspace{V}{f}{\lambda}{k}
        = \{ v \in V \suchthat (f - \lambda \id_V)^k(v) = 0 \}
        = \ker (f - \id_V)^k.
      \]
      Der Untervektorraum
      \[
                  \geigenspace{V}{f}{\lambda}
        \coloneqq \bigcup_{k=0}^\infty \heigenspace{V}{f}{\lambda}{k}
        =         \left\{
                    v \in V
                  \suchthat
                    \text{es gibt $k \geq 0$ mit $(f - \lambda \id_V)^k(v) = 0$}
                  \right\}
      \]
      ist der \emph{verallgemeinerte Eigenraum} von $f$ zu $\lambda$.
    \item
      Für $A \in \matrices{n}{K}$ und alle $\lambda \in K$ und $k \geq 0$ sei
      \[
          \heigenspace{(K^n)}{A}{\lambda}{k}
        = \{ x \in K^n \suchthat (A - \lambda \Id)^k x = 0 \}
        = \ker (A - \Id)^k.
      \]
      Der Untervektorraum
      \[
                  \geigenspace{(K^n)}{A}{\lambda}
        \coloneqq \bigcup_{k=0}^\infty \heigenspace{(K^n)}{A}{\lambda}{k}
        =         \left\{
                    x \in K^n
                  \suchthat
                    \text{es gibt $k \geq 0$ mit $(A - \lambda \Id)^k x = 0$}
                  \right\}
      \]
      ist der \emph{verallgemeinerte Eigenraum} von $A$ zu $\lambda$.
  \end{itemize}
\end{definition}

\begin{example}
  Der Endomorphismus $f$ ist genau dann nilpotent, wenn $\geigenspace{V}{f}{0} = V$ gilt.
  (Hier nutzen wir, dass $V$ endlichdimensional ist.)
\end{example}

\begin{lemma}
  \leavevmode
  \begin{enumerate}
    \item
      Es gilt genau dann $\geigenspace{V}{f}{\lambda} \neq 0$, wenn $\lambda$ ein Eigenwert von $f$ ist.
    \item
      Die Summe $\sum_{\lambda \in K} \geigenspace{V}{f}{\lambda}$ ist direkt.
  \end{enumerate}
\end{lemma}

Mithilfe der verallgemeinerten Eigenräume ergibt sich eine Charakterisierung der Existenz der Jordan-Normalform:

\begin{theorem}
  \label{theorem: existence of generalized eigespace decomposition and jordan normal form}
  Die folgenden Bedingungen äquivalent:
  \begin{enumerate}
    \item
      Das charakteristische Polynom $\charpol{f}(t)$ zerfällt in Linearfaktoren.
    \item
      Es gilt $V = \bigoplus_{\lambda \in K} \geigenspace{V}{f}{\lambda}$.
    \item
      Die Jordan-Normalform von $f$ existiert.
  \end{enumerate}
\end{theorem}

Dabei beizeichnet man die Zerlegung $V = \bigoplus_{\lambda \in K} \geigenspace{V}{f}{\lambda}$ als die \emph{verallgemeinerte Eigenraumzerlegung} bezüglich $f$.

\begin{example}
  Ist $f$ nilpotent, so gilt $V = \geigenspace{V}{f}{0}$, also $V = \sum_{\lambda \in K} \geigenspace{V}{f}{\lambda}$ und somit $V = \bigoplus_{\lambda \in K} \geigenspace{V}{f}{\lambda}$.
  Also existiert die verallgemeinerte Eigenraumzerlegung bezüglich $f$, und somit auch die Jordan-Normalform von $f$.
  
  Nilpotente Endomorphismen besitzen also immer eine Jordan-Normalform.
\end{example}



\subsection*{Kochrezept}

Ist $A \in \matrices{n}{K}$, so dass das charakteristische Polynom $\charpol{A}(t)$ in Linearfaktoren zerfällt, so lässt sich die Jordan-Normalform von $A$ sowie anschließend auch eine zugehörige Jordanbasis nach dem folgenden \emph{Kochrezept} berechnen:

\begin{itemize}
  \item
    Man bestimme die Eigenwerte von $A$, etwa indem man das charakteristische Polynom $\charpol{A}(t)$ berechnet, und anschließend die Nullstellen von $p_A(t)$ herausfindet.
    
  \item
    Für jeden Eigenwert $\lambda$ von $A$ führe man die folgenden Schritte durch:
    \begin{itemize}
      \item
        Man berechne die iterierten Kerne $\ker (A - \lambda \Id), \ker (A - \lambda \Id)^2, \dotsc, \ker (A - \lambda \Id)^m$ bis zu dem Punkt, an dem eine der folgenden äquivalenten Bedingungen erfüllt sind:
        \begin{itemize}
          \item
            Die Dimension von $\ker (A - \lambda \Id)^m$ ist gleich der algebraische Vielfachheit von $\lambda$ in $\charpol{A}(t)$.
          \item
            Es gilt $\ker (A - \lambda \Id)^m = \ker (A - \lambda \Id)^{m+1}$.
        \end{itemize}
      \item
        Man bestimme Anhand der Zahlen $d_k(\lambda) \coloneqq \dim \ker (A - \lambda \Id)^k$ die Anzahl der auftretenden Jordanblöcke zu $\lambda$ von Größe $k$ als
        \[
          b_k(\lambda) \coloneqq 2 d_k(\lambda) - d_{k-1}(\lambda) - d_{k+1}(\lambda).
        \]
    \end{itemize}
\end{itemize}

Aus den Eigenwerten $\lambda_1, \lambda_2, \dotsc$ von $A$ und den Zahlen $b_k(\lambda_i)$ erhalten wir bereits, wieviele Blöcke es zu welchen Eigenwert von welcher Größe gibt, d.h.\ wie die Jordan-Normalform von $A$ (bis auf Permutation der Blöcke) aussehen wird.
Inbesondere ist $d_1(\lambda) = \dim \ker (A - \lambda \Id)$ die Gesamtzahl der Jordanblöcke zu $\lambda$ und die entsprechende Potenz $m$ die maximale auftretende Blöckgröße zu $\lambda$.

Zur Berechnung einer Jordanbasis von $A$ kann man weiterhin wie folgt vorgehen:

\begin{itemize}[resume]
  \item
    Für jeden Eigenwert $\lambda$ von $A$ berechnet man eine Basis $\basis{B}_\lambda$ von $\geigenspace{(K^n)}{A}{\lambda}$ wie folgt:
    \begin{itemize}
      \item
        Man wähle Vektoren $v_1, \dotsc, v_{b_m} \in \ker (A - \lambda \Id)^m$ mit
        \[
                    \ker (A - \lambda \Id)^m
          =         \ker (A - \lambda \Id)^{m-1}
            \oplus  \generated{v_1, \dotsc, v_{b_m}}.
        \]
        (Zur konkreten Berechnung ergänze man eine Basis $\ker (A - \lambda \Id)^{m-1}$ zu einer Basis von $\ker (A - \lambda \Id)^m$; dann kann man $v_1, \dotsc, v_{b_m}$ als die neu hinzugekommenen Basisvektoren wählen.)
      \item
        Hierdurch ergeben sich für $\basis{B}_\lambda$ die ersten paar Basisvektoren
        \begin{align*}
          v_1,     (A - \lambda \Id) v_1,     &\dotsc, (A - \lambda \Id)^{m-1} v_1,     \\
          v_2,     (A - \lambda \Id) v_2,     &\dotsc, (A - \lambda \Id)^{m-1} v_2,     \\
                              &\dotsc,                  \\
          v_{b_m}, (A - \lambda \Id) v_{b_m}, &\dotsc, (A - \lambda \Id)^{m-1} v_{b_m}.
        \end{align*}
      \item
        Man wählt nun Vektoren $v'_1, \dotsc, v'_{b_{m-1}} \in \ker (A - \lambda \Id)^{m-1}$, so dass
        \begin{align*}
           &\,      \ker (A - \lambda \Id)^{m-1}
          \\
          =&\,      \ker (A - \lambda \Id)^{m-2}
            \oplus  \generated{ (A - \lambda \Id) v_1, \dotsc, (A - \lambda \Id) v_{b_m} }
            \oplus  \generated{ v'_1, \dotsc, v'_{b_{m-1}} }
        \end{align*}
        gilt.
      \item
        Hierdurch erhält man für $\basis{B}_\lambda$ die weitere Basisvektoren
        \begin{align*}
          v'_1,         (A - \lambda \Id) v'_1,         &\dotsc, (A - \lambda \Id)^{m-2} v'_1,          \\
          v'_2,         (A - \lambda \Id) v'_2,         &\dotsc, (A - \lambda \Id)^{m-2} v'_2,          \\
                                        &\dotsc,                        \\
          v'_{b_{m-1}}, (A - \lambda \Id) v'_{b_{m-1}}, &\dotsc, (A - \lambda \Id)^{m-2} v'_{b_{m-1}}.
        \end{align*}
      \item
        Man wähle nun $v''_1, \dotsc, v''_{b_{m-2}} \in \ker (A - \lambda \Id)^{m-2}$, so dass
        \begin{align*}
           &\,          \ker (A - \lambda \Id)^{m-1}  \\
          =&\,          \ker (A - \lambda \Id)^{m-2}
                \oplus  \generated{ (A - \lambda \Id)^2 v_1, \dotsc, (A - \lambda \Id)^2 v_{b_m} } \\
           &\,          \phantom{\ker (A - \lambda \Id)^{m-2}\,}
                \oplus  \generated{ (A - \lambda \Id) v'_1, \dotsc, (A - \lambda \Id) v'_{b_{m-1}} }
                \oplus  \generated{ v''_1, \dotsc, v''_{b_{m-2}} }
        \end{align*}
        gilt.
      \item
        Hiermit ergeben sich für $\basis{B}_\lambda$ die Basisvektoren
        \begin{align*}
          v''_1,         (A - \lambda \Id) v''_1,         &\dotsc, (A - \lambda \Id)^{m-2} v''_1,         \\
          v''_2,         (A - \lambda \Id) v''_2,         &\dotsc, (A - \lambda \Id)^{m-2} v''_2,         \\
                                          &\dotsc,                        \\
          v''_{b_{m-2}}, (A - \lambda \Id) v''_{b_{m-2}}, &\dotsc, (A - \lambda \Id)^{m-2} v''_{b_{m-2}}.
        \end{align*}
    \end{itemize}
    Durch Weiterführen der obigen Schritte erhält man schließlich eine Basis $\basis{B}_\lambda$ von $\geigenspace{(K^n)}{A}{\lambda}$.
    
  \item
    Sind $\lambda_1, \dotsc, \lambda_n$ die paarweise verschiedenen Eigenwerte von $A$, so ergibt sich durch Hintereinanderlegen der Basen $\basis{B}_{\lambda_1}, \dotsc, \basis{B}_{\lambda_t}$ eine Basis $\basis{B}$ von $K^n$.
    (Hier nutzen wir, dass $K^n = \bigoplus_{\lambda \in K} \geigenspace{(K^n)}{A}{\lambda}$ gilt.)
 
  \item
    Die Basis $\basis{B}$ ist eine Jordanbasis von $A$.
    Indem man die Basisvektoren von $\basis{B}$ als Spalten in eine Matrix $S$ einträgt, erhält man schließlich $S \in \GL{n}{K}$, so dass $S^{-1} A S$ in Jordan-Normalform ist.
    Dabei sind die Blöcke zunächst nach den Eigenwerten in der zuvor gewählten Reihenfolge sortiert, und die Blöcke zum gleichen Eigenwert sind jeweils nach absteigender Größe sortiert.
\end{itemize}

\begin{example}
  Für die Matrix
  \[
              A
    \coloneqq \begin{pmatrix*}[r]
                 3  & 0 & 1 \\
                 1  & 2 & 1 \\
                -1  & 0 & 1
              \end{pmatrix*}
    \in \matrices{3}{\Complex}
  \]
  gilt $\charpol{A}(t) = -(t-2)^3$.
  Also ist $2$ der einzige Eigenwert von $A$.
  Es gilt
  \[
      \ker (A - 2\Id)
    = \ker
      \begin{pmatrix*}[r]
         1  & 0 &  1  \\
         1  & 0 &  1  \\
        -1  & 0 & -1
      \end{pmatrix*}
    = \generated{ \rvect{1 \\ 0 \\ -1}, \vect{0 \\ 1 \\ 0} }.
  \]
  Somit gilt $\dim \ker (A - 2\Id) = 2$.
  Also gibt es zwei Jordanblöcke.
  Die Jordan-Normalform von $A$ ist somit
  \[
    \begin{pmatrix}
      2 &   &   \\
      1 & 2 &   \\
        &   & 2
    \end{pmatrix}.
  \]
  
  
  Zur Berechnung einer Jordanbasis von $A$ gehen wir wie folgt vor:
  \begin{itemize}
    \item
      Wir benötigen einen Vektor $v_1 \in \ker (A - 2 \Id)^2$ mit $\ker (A - 2 \Id)^2 = \ker (A - 2 \Id) \oplus \generated{v_1}$, also einen Vektor $v_1 \in \Complex^3$ mit $v_1 \notin \ker (A - 2 \Id)$.
      Hierfür bietet sich etwa der Vektor
      \[
                  v_1
        \coloneqq \vect{1 \\ 0 \\ 0}
      \]
      an.
      Damit erhalten wir auch den weiteren Basisvektor
      \[
          (A - 2 \Id) v_1
        = \rvect{1 \\ 1 \\ -1}.
      \]
    \item
      Nun benötigen wir noch einen Vektor $v_2 \in \ker (A - 2 \Id)$ mit
      \[
          \ker (A - 2 \Id)
        = \generated{(A - 2 \Id) v_1} \oplus v_2
        = \generated{ \rvect{1 \\ 1 \\ -1} } \oplus v_2.
      \]
      Hierfür lässt sich etwa
      \[
                  v_2
        \coloneqq \vect{0 \\ 1 \\ 0}
      \]
      wählen.
  \end{itemize}
  Insgesamt haben wir somit für $A$ die Jordanbasis
  \[
              \basis{B}
    \coloneqq \left(
                \vect{1 \\ 0 \\ 0},
                \rvect{1 \\ 1 \\ -1},
                \vect{0 \\ 1 \\ 0}
              \right)
  \]
  von $\Complex^3$.
  Für die entsprechende Matrix
  \[
              S
    \coloneqq \begin{pmatrix*}[r]
                1 &  1  & 0 \\
                0 &  1  & 1 \\
                0 & -1  & 0
              \end{pmatrix*}
    \in \GL{3}{\Complex}
  \]
  gilt dann, dass
  \[
      S^{-1} A S
    = \begin{pmatrix}
        2 &   &   \\
        1 & 2 &   \\
          &   & 2
      \end{pmatrix}.
  \]
\end{example}

\begin{example}
  Es sei $V$ ein $K$-Vektorraum mit Basis $\basis{B} = (v_1, \dotsc, v_7)$ und $f \colon V \to V$ der eindeutige Endomorphismus mit $f(v_1) = f(v_2) = v_5$, $f(v_3) = f(v_4) = v_6$, $f(v_5) = f(v_6) = v_7$ und $f(v_7) = 0$.
  Es gilt $f^3(v_i) = 0$ für alle $i$, und deshalb $f^3 = 0$.
  Also ist $f$ nilpotent, und besitzt somit eine Jordan-Normalform;
  dabei ist $0$ der einzige auftretende Eigenwert.
  
  Für die darstellende Matrix
  \begin{gather*}
              A
    \coloneqq \repmatrixendo{f}{\basis{B}}
    =         \begin{pmatrix}
                0 &   &   &   &   &   &   \\
                0 & 0 &   &   &   &   &   \\
                0 & 0 & 0 &   &   &   &   \\
                0 & 0 & 0 & 0 &   &   &   \\
                1 & 1 & 0 & 0 & 0 &   &   \\
                0 & 0 & 1 & 1 & 0 & 0 &   \\
                0 & 0 & 0 & 0 & 1 & 1 & 0
              \end{pmatrix}
  \shortintertext{gilt}
      A^2
    = \begin{pmatrix}
        0 &   &   &   &   &   &   \\
          & 0 &   &   &   &   &   \\
          &   & 0 &   &   &   &   \\
          &   &   & 0 &   &   &   \\
          &   &   &   & 0 &   &   \\
          &   &   &   &   & 0 &   \\
        1 & 1 & 1 & 1 & 0 & 0 & 0
      \end{pmatrix}
    \quad\text{und}\quad
      A^3
    = 0.
  \end{gather*}
  Somit gelten
  \begin{align*}
        \ker A\phantom{^1}
    &=  \generated{ e_1 - e_2, e_3 - e_4, e_5 - e_6, e_7 },
    \\
        \ker A^2
    &=  \generated{ e_1 - e_2, e_2 - e_3, e_3 - e_4, e_5, e_6, e_7 },
    \\
        \ker A^3
    &=  K^7.
  \end{align*}
  Mit $d_1 = \dim \ker A = 4$, $d_2 = \dim \ker A^2 = 6$ und $d_k = \ker \ker A^k = 7$ für $k \geq 3$ erhalten wir, dass $b_1 = 2 d_1 - d_2 = 2$, $b_2 = 2 d_2 - d_3 - d_1 = 1$, $b_3 = 2 d_3 - d_4 - d_2 = 1$ und $b_k = 0$ für $k \geq 4$.
  
  Die Jordan-Normalform von $A$ (und damit von $f$) besteht also aus zwei Blöcken der Größe $1$, einem Block der Größe $2$ und einem Block der Größe $3$ (jeweils zum Eigenwert $0$);
  sie ist also
  \[
    \begin{pmatrix}
      0 &   &   &   &   &   &   \\
      1 & 0 &   &   &   &   &   \\
        & 1 & 0 &   &   &   &   \\
        &   &   & 0 &   &   &   \\
        &   &   & 1 & 0 &   &   \\
        &   &   &   &   & 0 &   \\
        &   &   &   &   &   & 0
    \end{pmatrix}.
  \]

  Wir bestimmen nun eine entsprechende Jordanbasis:
  \begin{itemize}
    \item
      Wir benötigen zunächst $w_1 \in \ker A^3 = K^7$ mit $K^7 = \ker A^2 \oplus \generated{w_1}$.
      Hierfür muss nur $w_1 \notin \ker A^2$ gelten.
      Wir wählen $w_1 \coloneqq e_1$.
      Dann erhalten wir auch die weiteren Basisvektoren $A w_1 = e_5$ und $A^2 w_1 = e_7$.
    \item
      Als nächstes benötigen wir $w_2 \in \ker A^2$ mit
      \[
          \ker A^2
        =         \ker A
          \oplus  \generated{A w_1}
          \oplus  \generated{w_1}
        =         \generated{e_1 - e_2, e_3 - e_4, e_5 - e_6, e_7}
          \oplus  \generated{e_5}
          \oplus  \generated{w_2}.
      \]
      Wir müssen also die Familie $( e_1 - e_2, e_3 - e_4, e_5, e_5 - e_6, e_7 )$ zu einer Basis von $\ker A^2$ ergänzen.
      Da
      \[
          \generated{ e_1 - e_2, e_3 - e_4, e_5, e_5 - e_6, e_7 } \\
        = \generated{ e_1 - e_2, e_3 - e_4, e_5, e_6, e_7 }
      \]
      gilt, können wir $w_2 \coloneqq e_2 - e_3$ wählen.
      Damit erhalten wir außerdem den Basisvektoren $A w_2 = e_5 - e_6$.
    \item
      Schließlich brauchen wir noch $w_3, w_4 \in \ker A$ mit
      \[
          \ker A
        =         \generated{A^2 w_1}
          \oplus  \generated{A w_2}
          \oplus  \generated{w_3, w_4}
        =         \generated{e_7}
          \oplus  \generated{e_5 - e_6}
          \oplus  \generated{w_3, w_4}.
      \]
      Wir müssen also die Familie $(e_5 - e_6, e_7)$ zu einer Basis von $\ker A$ ergänzen.
      Hierfür können wir $w_3 \coloneqq e_1 - e_2$ und $w_4 \coloneqq e_3 - e_4$ wählen.
  \end{itemize}

  Ingesamt haben wir somit die Basis $\basis{C} = (w_1, \dotsc, w_n)$, bzw.\ die Basiswechselmatrix
  \[
              S
    \coloneqq \begin{pmatrix*}[r]
                1 & 0 & 0 &  0  &  0  &  1  &  0  \\
                0 & 0 & 0 &  1  &  0  & -1  &  0  \\
                0 & 0 & 0 & -1  &  0  &  0  &  1  \\
                0 & 0 & 0 &  0  &  0  &  0  & -1  \\
                0 & 1 & 0 &  0  &  1  &  0  &  0  \\
                0 & 0 & 0 &  0  & -1  &  0  &  0  \\
                0 & 0 & 1 &  0  &  0  &  0  &  0
              \end{pmatrix*},
  \]
  und es gilt
  \[
      \repmatrixendo{f}{\basis{C}}
    = S^{-1} A S
    = \begin{pmatrix}
        0 &   &   &   &   &   &   \\
        1 & 0 &   &   &   &   &   \\
          & 1 & 0 &   &   &   &   \\
          &   &   & 0 &   &   &   \\
          &   &   & 1 & 0 &   &   \\
          &   &   &   &   & 0 &   \\
          &   &   &   &   &   & 0
      \end{pmatrix}.
  \]
\end{example}



% manual pagebreak
\pagebreak


\section{Lösung des Ähnlichkeitsproblems}

Es seien $f, g \colon V \to V$ zwei Endorphismen, die jeweils eine Jordan-Normalform besitzen.

\begin{theorem}
  Die Endomorphismen $f$ und $g$ sind genau dann ähnlich, wenn sie \textup(bis auf Permutation der Blöcke\textup) die gleiche Jordan-Normalform besitzen.
\end{theorem}

Also sind $f, g \colon V \to V$ genau dann ähnlich, wenn ihre charakteristischen Polynome
\[
    \charpol{f}(t)
  = \charpol{g}(t)
  = (t - \lambda_1)^{n_1} \dotsm (t - \lambda_s)^{n_s}
\]
übereinstimmen, und wenn für jeden Eigenwert $\lambda_i$ gilt, dass
\[
    \dim \ker (f - \lambda_i \id_V)^k
  = \dim \ker (g - \lambda_i \id_V)^k
  \qquad
  \text{für alle $k = 1, \dotsc, n_i$}.
\]

\begin{example}
  \leavevmode
  \begin{enumerate}
    \item
      Für nilpotente Matrizen existiert immer eine verallgemeinerte Eigenraumzerlegung.
      Deshalb sind zwei nilpotente Matrizen $A, B \in \matrices{n}{K}$ genau dann ähnlich, wenn $\dim \ker A^k = \dim \ker B^k$ für alle $k \geq 0$ gilt.
    \item
      Es seien $A_1, A_2, A_3 \in \matrices{3}{\Complex}$ mit
      \[
          A_1
        = \begin{pmatrix}
            2 &   &   \\
              & 1 &   \\
              & 1 & 1
          \end{pmatrix},
        \quad
          A_2
        = \begin{pmatrix}
            2 &   &   \\
              & 1 & 1 \\
              &   & 1
          \end{pmatrix},
        \quad
          A_3
        = \begin{pmatrix}
            2 & 1 &   \\
              & 1 &   \\
              &   & 1
          \end{pmatrix}.
      \]
      Dann gilt $\charpol{A_1}(t) = \charpol{A_2}(t) = \charpol{A_3}(t) = - (t-1)^2 (t-2)$.
      Da $2$ jeweils eine algebraische Vielfachheit $1$ hat, gilt
      \[
          \dim \ker (A_1 - 2 \Id)
        = \dim \ker (A_2 - 2 \Id)
        = \dim \ker (A_3 - 2 \Id)
        = 1.
      \]
      Für den Eigenwerte $1$ ergeben sich die Dimensionen
      \[
          \dim \ker (A_1 - \Id)
        = \dim \ker (A_1 - \Id)
        = 1
        \quad\text{und}\quad
          \dim \ker (A_3 - \Id)
       = 2,
      \]
      also ist $A_3$ zu keiner der beiden anderen Matrizen ähnlich.
      Da ferner
      \[
          \dim \ker (A_1 - \Id)^2
        = \dim \ker (A_2 - \Id)^2
        = 2
      \]
      gilt, sind $A_1$ und $A_2$ ähnlich.
      
    \item
      Da $A_1$ und $A_2$ in unterer, bzw.\ oberer Jordan-Normalform sind, sieht man direkt, dass Sie nicht diagonalisierbar sind.
      
      Die Matrix $A_3$ ist hingegen eine Blockdiagonalmatrix, deren Blöcke beide diagonalisierbar sind:
      Der obere Block ist diagonalisierbar, da es sich um eine obere Dreicksmatrix mit paarweise verschieden Diagonaleinträgen handelt.
      Also ist auch $A_3$ diagonalisierbar.
      
      Hierdurch sieht man bereits, dass $A_3$ nicht ähnlich zu $A_1$ oder $A_2$ ist.
      
    \item
      Dass $A_1$ und $A_2$ ähnlich sind, erkennt man auch wie folgt:
      
      Besitzt $A \in \matrices{n}{K}$ eine Jordan-Normalform, so sind $A$ und $\transpose{A}$ ähnlich:
      Wir haben bereits gesehen, dass $\charpol{A}(t) = \charpol{\transpose{A}}(t)$ gilt.
      Außerdem gilt
      \[
          \dim \ker A
        = n - \rank A
        = n - \dim \transpose{A}
        = \dim \transpose{A},
      \]
      wobei die mittlere Gleichheit genau die Gleichheit von Zeilenrang und Spaltenrang ist.
      Für alle $\lambda \in K$ und $k \geq 0$ gilt damit auch
      \begin{align*}
            \dim \ker (A - \lambda \Id)^k
        &=  \dim \ker \transpose{ ( (A - \lambda \Id)^k ) }
         =  \dim \ker ( \transpose{ (A - \lambda \Id) } )^k
        \\
        &=  \dim \ker ( \transpose{A} - \lambda \transpose{\Id} )^k
         =  \dim \ker (\transpose{A} - \lambda \Id)^k.
      \end{align*}
      Zusammen zeigt dies, dass $A$ und $\transpose{A}$ ähnlich sind.
      
      Inbesondere ist damit jede Matrix in unterer Jordan-Normalform ähnlich zu der entsprechenden oberen Jordan-Normalform.
  \end{enumerate}
\end{example}





\section{Die Jordan--Chevalley-Zerlegung}

Es sei $f \colon V \to V$ ein Endomorphismus, der die Bedingungen aus Satz~\ref{theorem: existence of generalized eigespace decomposition and jordan normal form} erfüllt (d.h.\ $f$ besitzt eine Jordan-Normalform.)

\begin{proposition}[Jordan--Chevalley-Zerlegung]
  Es gibt eine eindeutige Zerlegung $f = d + n$ in einen diagonalisierbaren Endomorphismus $d$ und einen nilpotenten Endomorphismus $n$, so dass $d$ und $n$ kommutieren.
\end{proposition}

Die Jordan--Chevalley-Zerlegung ist eine „koordinatenfreie“ Version der Jordan-Normalform;
die Existenz dieser Zerlegung ist äquivalent zur Existenz der Jordan-Normalform.

\begin{example}
  Für die Matrix
  \[
              A
    \coloneqq \begin{pmatrix*}[r]
                -4  & -9  \\
                 4  &  8
              \end{pmatrix*}
    \in       \matrices{2}{\Complex}
  \]
  und die Basiswechselmatrix
  \[
      S
    = \begin{pmatrix*}[r]
         2  & -1 \\
        -3  &  2
      \end{pmatrix*}
  \]
  gilt
  \[
      S^{-1} A S
    = \begin{pmatrix}
        2 &   \\
        1 & 2
      \end{pmatrix}
    = \begin{pmatrix}
        2 &   \\
          & 2
      \end{pmatrix}
      +
      \begin{pmatrix}
        0 &   \\
        1 & 0
      \end{pmatrix}.
  \]
  Die Jordan--Chevalley-Zerlegung $A = D + N$ ist somit gegeben durch
  \begin{gather*}
      D
    = S
      \underset{= 2 \Id}{
      \begin{pmatrix}
        2 &   \\
          & 2
      \end{pmatrix}}
      S^{-1}
    = \begin{pmatrix}
        2 &   \\
          & 2
      \end{pmatrix}
  \shortintertext{und}
      N
    = S
      \begin{pmatrix}
        0 &   \\
        1 & 0
      \end{pmatrix}
      S^{-1}
    = \begin{pmatrix*}[r]
        -6  & -9  \\
         4  &  6
      \end{pmatrix*}.
  \end{gather*}
\end{example}





\section{Implizite Bestimmen der Jordan-Normaform}

Es sei $J \in \matrices{n}{K}$ eine Jordan-Normalform von $f$.
\begin{itemize}
  \item
    Für das charakteristische Polynom $\charpol{f}(t) = \charpol{J}(t) = (t - \lambda_1)^{n_1} \dotsm (t - \lambda_s)^{n_s}$ mit $\lambda_i \neq \lambda_j$ für $i \neq j$ gilt
    \[
        n_i
      = \dim \geigenspace{V}{f}{\lambda_i}
      = \text{wie oft $\lambda_i$ auf der Diagonalen von $J$ steht}.
    \]
  \item
    Es gilt
    \[
        \text{Anzahl der Jordanblöcke zu $\lambda$ in $J$}
      = \dim \ker (f - \lambda \id_V)
      = n - \rank (f - \lambda \id_V).
    \]
    Inbesondere ist $n - \rank f$ die Anzahl der Jordanblöcke zu $0$.
  \item
    Für das Minimalpolynom $\minpol{f}(t) = \minpol{J}(t) = (t - \lambda_1)^{m_1} \dotsm (t - \lambda_s)^{m_s}$ mit $\lambda_i \neq \lambda_j$ für $i \neq j$ gilt
    \[
        m_i
      = \text{maximale auftretende Größe eines Jordanblockes zu $\lambda_i$ in $J$}.
    \]
  \item
    Ist allgemeiner $q(t) \in K[t]$ ein Polymom mit $q(t) = (t - \lambda_1)^{m'_1} \dotsm (t - \lambda_s)^{m'_s}$, wobei $\lambda_i \neq \lambda_j$ für $i \neq j$ gilFt, und $q(f) = 0$, so ergeben sich die folgenden beiden Restriktionen an $J$:
    \begin{enumerate}
      \item
        Jeder Eigenwert von $f$ kommt in $\lambda_1, \dotsc, \lambda_s$ vor (siehe Lemma~\ref{lemma: polynomial equations give restriction for the eigenvalues}),
      \item
        Es gilt $\minpol{f} \divides q$.
        Deshalb ist $\minpol{f} = (t - \lambda_1)^{m_1} \dotsm (t - \lambda_s)^{m_s}$ mit $m_i \leq m'_i$ für alle $i$.
        Also sind die Jordanblöcke zu $\lambda_i$ in $J$ jeweils höchstens $m'_i$ groß.
    \end{enumerate}
\end{itemize}

Zusammen mit den Beobachtungen aus Abschnitt~\ref{section: trace and determinant via eigenvalues} kann man hierdurch für kleine Matrizen bereits Aussagen über die Jordan-Nor\-mal\-form treffen, ohne die Matrix selbst zu kennen.

\begin{example}
  Es sei $A \in \matrices{6}{\Complex}$ mit $(A - 2\Id)^2 (A - 3\Id) = 0$ und $\tr A = 14$.
  
  Dann sind $2$ und $3$ die einzigen möglichen Eigenwerte von $A$.
  Aus $\tr A = 14$ erhält man, dass $2$ mit algebraischer Vielfachheit $4$ vorkommt, und $3$ mit algebraischer Vielfachheit $2$.
  Außerdem folgt aus $(A - 2\Id)^2 (A - 3\Id) = 0$, dass die Jordanblöcke zu $2$ höchstens $2$ groß sind, und die Jordanblöcke zu $3$ alle Größe $1$ haben.
  
  Damit ergeben sich bis auf Permutation der Jordanblöcke die folgenden drei möglichen Jordan-Normalformen:
  \[
    \begin{pmatrix}
      2 &   &   &   &   &   \\
      1 & 2 &   &   &   &   \\
        &   & 2 &   &   &   \\
        &   & 1 & 2 &   &   \\
        &   &   &   & 3 &   \\
        &   &   &   &   & 3
    \end{pmatrix},
    \quad
    \begin{pmatrix}
      2 &   &   &   &   &   \\
      1 & 2 &   &   &   &   \\
        &   & 2 &   &   &   \\
        &   &   & 2 &   &   \\
        &   &   &   & 3 &   \\
        &   &   &   &   & 3
    \end{pmatrix},
    \quad
    \begin{pmatrix}
      2 &   &   &   &   &   \\
        & 2 &   &   &   &   \\
        &   & 2 &   &   &   \\
        &   &   & 2 &   &   \\
        &   &   &   & 3 &   \\
        &   &   &   &   & 3
    \end{pmatrix}
  \]
  In den ersten beiden Fällen ist $(t-2)^2 (t-1)$ das Minimalpolynom, im dritten Fall ist es $(t-2)(t-3)$.
\end{example}














