\chapter{Grundlagen zu Bilinearformen}

Im Folgenden sei $K$ ein Körper, und $U$, $V$ und $W$ seien $K$-Vektorräume.

\begin{notation}
  Für alle $\mnatrices{m}{n}{\Complex}$ schreiben wir $\madj{A} \coloneqq \transpose{\conjugate{A}}$.
\end{notation}

\begin{definition}
  Die Matrix $A \in \matrices{n}{\Complex}$ heißt \emph{hermitesch}, falls $A = \madj{A}$ gilt.
\end{definition}

\begin{example}
  \leavevmode
  \begin{enumerate}
    \item
      Eine reelle Matrix ist genau dann hermitesch, wenn sie symmetrisch ist.
    \item
      Ist allgemeiner $A = B + iC$ mit $B, C \in \matrices{n}{\Real}$, so ist $A$ genau dann hermitesch, wenn $B$ symmetrisch und $C$ schiefsymmetrisch ist.
  \end{enumerate}
\end{example}





\section{Allgemeine Definitionen}

\begin{definition}
  \leavevmode
  \begin{itemize}
    \item
      Eine Abbildung $\beta \colon V \times W \to U$ heißt \emph{$K$-bilinear}, falls
      \begin{gather*}
          \beta(v_1 + v_2, w)
        = \beta(v_1, w) + \beta(v_2, w),
        \\
          \beta(v, w_1 + w_2)
        = \beta(v, w_1) + \beta(v, w_2),
        \\
          \beta(\lambda v, w)
        = \lambda \beta(v, w)
        \quad\text{und}\quad
          \beta(v, \lambda w)
        = \lambda \beta(v, w)
      \end{gather*}
      für alle $v, v_1, v_2 \in V$, $w, w_1, w_2 \in W$ und $\lambda \in K$ gilt.
    \item
      Gilt zusätzlich $U = K$, so ist $\beta$ eine \emph{Bilinearform}.
      Es ist
      \[
                  \BilForm(V, W)
        \coloneqq \{ \beta \colon V \times W \to K \suchthat \text{$\beta$ ist eine Bilinearform} \}.
      \]
    \item
      Gilt zusätzlich $V = W$ und
      \[
          \beta(v_1, v_2)
        = \beta(v_2, v_1)
        \qquad
        \text{für alle $v_1, v_2 \in V$},
      \]
      so heißt $\beta$ \emph{symmetrisch}.
      Es sind
      \begin{gather*}
        \BilForm(V) \coloneqq \BilForm(V, V)
      \shortintertext{sowie}
                  \SymForm(V)
        \coloneqq \{
                    \beta \in \BilForm(V)
                  \suchthat
                    \text{$\beta$ ist symmetrisch}
                  \}.
      \end{gather*}
  \end{itemize}
\end{definition}

%manual page break
\pagebreak

\begin{definition}
  Es sei $K = \Complex$.
  \begin{itemize}
    \item
      Eine Abbildung $\beta \colon V \times W \to U$ heißt \emph{sesquilinear}, falls
      \begin{gather*}
          \beta(v_1 + v_2, w)
        = \beta(v_1, w) + \beta(v_2, w),
        \\
          \beta(v, w_1 + w_2)
        = \beta(v, w_1) + \beta(v, w_2),
        \\
          \beta(\lambda v, w)
        = \lambda \beta(v, w)
        \quad\text{und}\quad
          \beta(v, \lambda w)
        = \conjugate{\lambda} \beta(v,w)
      \end{gather*}
      für alle $v, v_1, v_2 \in V$, $w, w_1, w_2 \in W$ und $\lambda \in \Complex$ gilt.
    \item
      Gilt zusätzlich $U = \Complex$, so ist $\beta$ eine \emph{Sesquilinearform}.
      Es ist
      \[
                  \SesForm(V, W)
        \coloneqq \{
                    \beta \colon V \times W \to \Complex
                   \suchthat
                    \text{$\beta$ ist eine Sesquilinearform}
                  \}.
      \]
    \item
      Gilt zusätzlich $V = W$ und
      \[
          \beta(v_1, v_2)
        = \conjugate{\beta(v_2, v_1)}
        \qquad
        \text{für alle $v_1, v_2 \in V$},
      \]
      so heißt $\beta$ \emph{hermitesch}.
      Es sind
      \begin{gather*}
        \SesForm(V) \coloneqq \SesForm(V, V)
      \shortintertext{sowie}
                  \HerForm(V)
        \coloneqq \{ \beta \in \SesForm(V) \suchthat \text{$\beta$ ist hermitesch} \}.
      \end{gather*}
  \end{itemize}
\end{definition}

\begin{definition}
  \label{definition: semilinear for complex vector spaces}
  Für $K = \Complex$ heißt eine Abbildung $f \colon V \to W$ \emph{halblinear} falls
  \[
    f(v_1 + v_2) = f(v_1) + f(v_2)
    \quad\text{und}\quad
    f(\lambda v) = \conjugate{\lambda} f(v)
  \]
  für alle $v, v_1, v_2 \in V$ und $\lambda \in K$ gilt.
\end{definition}

Eine Abbildung $\beta \colon V \times W \to U$ ist genau dann $K$-bilinear, wenn für alle $v \in V$ und $w \in W$ die Abbildungen
\begin{alignat*}{3}
           \beta(v,-)
  &\colon   W
   \to      U,
  &\quad&
            w'
  &\mapsto  \beta(v, w')
\shortintertext{und}
            \beta(-,w)
  &\colon   V
   \to      U,
  &\quad&        
            v'
  &\mapsto  \beta(v', w)
\end{alignat*}
linear sind.

Im Fall $K = \Complex$ ist $\beta$ genau dann sesquilinear, wenn die Abbildung $\beta(-,w) \colon V \to U$ für jedes $w \in W$ linear ist, und die Abbildung $\beta(v,-) \colon W \to U$ für jedes $v \in V$ halblinear ist.

% manual pagebreak
\pagebreak

\begin{example}
  \leavevmode
  \begin{enumerate}
    \item
      Für jede Matrix $A \in \mnatrices{m}{n}{K}$ ist die Abbildung
      \[
                \beta_A
        \colon  K^m \times K^n
        \to     K
        \quad\text{mit}\quad
          \beta_A(x,y)
        = \transpose{x} A y
      \]
      eine Bilinearform.
      Für $n = m$ ist $\beta_A$ genau dann symmetrisch, wenn $A$ symmetrisch ist.
      
      Jede Bilinearform $\beta \in \BilForm(K^m, K^n)$ ist von dieser Form:
      Ist $A \in \mnatrices{m}{n}{K}$ die Matrix mit $A_{ij} = \beta(e_i, e_j)$ für alle $i, j$, so gilt
      \[
          \beta_A(e_i, e_j)
        = \transpose{e_i} A e_j
        = A_{ij}
        = \beta(e_i, e_j)
        \qquad
        \text{für alle $i, j$}
      \]
      und somit $\beta = \beta_A$ wegen der Bilinearität von $\beta$ und $\beta_A$.
      
      Damit ergibt sich ein Isomorphismus $\mnatrices{m}{n}{K} \to \BilForm(K^m, K^n)$, $A \mapsto \beta_A$.
      
    \item
      Analog ergibt sich aus jeder Matrix $A \in \mnatrices{m}{n}{\Complex}$ eine Sesquilinearform
      \[
                \beta_A
        \colon  \Complex^m \times \Complex^n
        \to     \Complex,
        \quad\text{mit}\quad
          \beta_A(x,y)
        = \transpose{x} A \conjugate{y},
      \]
      und somit insgesamt ein Isomorphismus $\mnatrices{m}{n}{\Complex} \to \SesForm(\Complex^m, \Complex^n)$, $A \mapsto \beta_A$.
      Die hermiteschen Sesquilinearformen entsprechen dabei genau den hermiteschen Matrizen.
  \end{enumerate}
\end{example}





\section{Quadratische Formen und Polarisation}
\label{section: quadratic forms and polarisation}

Es gelte $\ringchar{K} \neq 2$.

\begin{proposition}
  \label{proposition: characterizations of quadratic forms}
  Für $Q \colon V \to K$ sind die folgenden Bedingungen äquivalent:
  \begin{enumerate}
    \item
      Es gibt eine Bilinearform $\beta \in \BilForm(V)$ mit $Q(v) = \beta(v, v)$ für all $v \in V$.
    \item
      Es gibt eine Basis $\basis{B} = (v_1, \dotsc, v_n)$ von $V$ und Koeffizienten $c_{ij} \in K$ mit
      \begin{equation}
        \label{equation: quadratic form in coordinates}
          Q\left( \sum_{i=1}^n x_i v_i \right)
        = \sum_{i,j=1}^n c_{ij} x_i x_j
        \qquad
        \text{für alle $x_1, \dotsc, x_n \in K$}.
      \end{equation}
    \item
      Für jede Basis $\basis{B} = (v_1, \dotsc, v_n)$ von $V$ gibt es Koeffizienten $c_{ij} \in K$, so dass \eqref{equation: quadratic form in coordinates} gilt.
    \item
      Es gilt $Q(av) = a^2 Q(v)$ für alle $v \in V$ und $a \in K$, und die Abbildung
      \[
                \beta
        \colon  V \times V
        \to     K
        \quad\text{mit}\quad
          \beta(v,w)
        = \frac{ Q(v + w) - Q(v) - Q(w) }{2}
      \]
      ist bilinear.
  \end{enumerate}
\end{proposition}

\begin{definition}
  Eine Abbildung $Q \colon V \to K$, die eine, und damit alle Eigenschaften aus Proposition~\ref{proposition: characterizations of quadratic forms} erfüllt, heißt \emph{quadratische Form} auf $V$.
  Es ist
  \[
              \Quad{V}
    \coloneqq \{ Q \colon V \to K \suchthat \text{$Q$ ist eine quadratische Form} \}
  \]
  der $K$-Vektorraum der quadratischen Formen auf $K$ \textup(mit punktweiser Addition und Skalarmultiplikation\textup).
\end{definition}

Jede Bilinearform $\beta \in \BilForm(V)$ liefert eine quadratische Form
\[
          Q_\beta
  \colon  V
  \to     K,
  \quad   v
  \mapsto \beta(v,v).
\]
Andererseits liefert jede quadratische Form $Q \colon V \to K$ eine symmetrische Bilinearform $\beta_Q \in \SymForm(V)$ durch
\begin{equation}
  \label{equation: polarisation formula}
            \beta_Q(v,w)
  \coloneqq \frac{Q(v+w) - Q(v) - Q(w)}{2}
  \qquad
  \text{für alle $v, w \in V$}.
\end{equation}
Diese beiden Konstruktionen sind invers zueinander, und liefern einen Isomorphismus
\begin{align*}
                      \SymForm(V)
  \longleftrightarrow \Quad{V},
  \quad
                      \beta
  \longmapsto         Q_\beta,
  \quad
                      Q
  \longmapsfrom       \beta_Q.
\end{align*}

Wir bezeichnen \eqref{equation: polarisation formula} als eine \emph{Polarisationsformel};
sie erlaubt es uns, aus der quadratischen Form $Q$ die ursprüngliche symmetrische Bilinearform $\beta$ zurückzugewinnen.
Neben \eqref{equation: polarisation formula} gilt auch noch die Polarisationsformel
\[
    \beta(v_1, v_2)
  = \frac{Q(v_1 + v _2) - Q(v_1 - v_2)}{4}
  \qquad
  \text{für alle $v_1, v_2 \in V$}.
\]


\begin{remark}
  Proposition~\ref{proposition: characterizations of quadratic forms} gilt auch für $\ringchar{K} = 2$, sofern man den Ausdruck $(Q(v+w) - Q(v) - Q(w))/2$ durch $Q(v+w) - Q(v) - Q(w)$ ersetzt.
\end{remark}

Auch für Sesquilinearformen gibt es eine Polarisationsformel:
Ist $\beta \in \SesForm(V)$ eine Sesquilinearform (nicht notwendigerweise hermitesch!), so gilt für die Abbildung $Q \colon V \to V$, $v \mapsto Q(v,v)$, dass
\[
    \beta(v_1, v_2)
  = \frac{Q(v_1 + v_2) - Q(v_1 - v_2) + i Q(v_1 + i v_2) - i Q(v_1 - i v_2)}{4}
\]
für alle $v_1, v_2 \in V$






\section{Orthogonalität}

Es sei $\beta \in \SymForm(V)$ eine symmetrische Bilinearform, bzw.\ $K = \Complex$ und $\beta \in \HerForm(V)$ eine hermitsche Sesquilinearform.

\begin{definition}
  \leavevmode
  \begin{itemize}
    \item
      Zwei Vektoren $v_1, v_2 \in V$ sind \emph{orthogonal zueinander}, notiert mit $v_1 \orth v_2$, falls $\beta(v_1, v_2) = 0$ gilt.
    \item
      Für jeden Untervektorraum $U \subseteq V$ ist
      \[
          U^\perp
        = \{
            v \in V
          \suchthat
            \text{$\beta(v,u) = 0$ für alle $u \in U$}
          \}.
      \]
      das \emph{orthogonale Komplement} von $U$ bezüglich $\beta$.
    \item
      Eine Familie $(v_i)_{i \in I}$ von Vektoren $v_i \in V$ heißt \emph{orthogonal}, falls $v_i \orth v_j$ für alle $i \neq j$ gilt.
    \item
      Eine \emph{Orthogonalbasis} ist eine orthogonale Basis.
    \item
      Ein Vektor $v \in V$ heißt \emph{normiert} falls $\beta(v,v) = 1$ gilt.
    \item
      Eine Familie $(v_i)_{i \in I}$ von Vektoren $v_i \in V$ heißt \emph{normiert}, falls $v_i$ für jedes $i \in I$ normiert ist.
    \item
      Eine \emph{Orthonormalbasis} ist eine orthogonale, normierte Basis.
  \end{itemize}
\end{definition}




