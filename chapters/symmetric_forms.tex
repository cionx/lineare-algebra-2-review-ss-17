\chapter{Symmetrische Bilinearformen}

Im Folgenden sei $K$ ein Körper und $V$ ein endlichdimensionaler $K$-Vektorraum.





\section{Darstellende Matrizen \& Kongruenz}

\begin{definition}
  \label{definition: representing matrix for a bilinear form}
  Die \emph{darstellende Matrix} einer Bilinearform $\beta \in \BilForm(V)$ bezüglich einer Basis $\basis{B} = (v_1, \dotsc, v_n)$ von $V$ ist die Matrix
  \[
              \repmatrixbilone{\beta}{\basis{B}}
    \coloneqq \begin{pmatrix}
                \beta(v_1, v_1) & \beta(v_1, v_2) & \cdots  & \beta(v_1, v_n) \\
                \beta(v_2, v_1) & \beta(v_2, v_2) & \cdots  & \beta(v_2, v_n) \\
                \vdots          & \vdots          & \ddots  & \vdots          \\
                \beta(v_n, v_1) & \beta(v_n, v_2) & \cdots  & \beta(v_n, v_n)
              \end{pmatrix}.
  \]
\end{definition}

\begin{proposition}
  Es sei $\basis{B} = (v_1, \dotsc, v_n)$ eine Basis von $V$ und
  \[
              V
    \to       K^n,
    \quad     v
    =         \sum_{i=1}^n x_i v_i
    \mapsto   \vect{x_1 \\ \vdots \\ x_n}
    \eqqcolon [v]
  \]
  der zugehörige Isomorphismus $V \to K^n$.
  \begin{enumerate}
    \item
      Für alle $w_1, w_2 \in V$ gilt $\beta(w_1, w_2) = \transpose{[w_1]} A [w_2]$.
    \item
      Die Abbildung
      \[
                \BilForm(V)
        \to     \matrices{n}{K},
        \quad   \beta
        \mapsto \repmatrixbilone{\beta}{\basis{B}}     
      \]
      ist ein Isomorphismus von $K$-Vektorräumen.
  \end{enumerate}
\end{proposition}

\begin{lemma}
  Für $\beta \in \BilForm(V)$ sind die folgenden Bedingungen äquivalent:
  \begin{enumerate}
    \item
      Die Bilinearform $\beta$ ist symmetrisch.
    \item
      Es gibt eine Basis $\basis{B}$ von $V$, so dass $\repmatrixbilone{\beta}{\basis{B}}$ symmetrisch ist.
    \item
      Für jede Basis $\basis{B}$ von $V$ ist $\repmatrixbilone{\beta}{\basis{B}}$ symmetrisch.
  \end{enumerate}
\end{lemma}

\begin{lemma}
  Sind $\basis{B}$ und $\basis{C}$ zwei Basen von $V$, und ist $\beta \in \BilForm(V)$ eine Bilinearform, so gilt
  \[
      \repmatrixbilone{\beta}{\basis{B}}
    = \repmatrixhom{\id_V}{\basis{B}}{\basis{C}}^T
      \repmatrixbilone{\beta}{\basis{C}}
      \repmatrixhom{\id_V}{\basis{B}}{\basis{C}}.
  \]
\end{lemma}

\begin{definition}
  Zwei Matrizen $A, B \in \matrices{n}{K}$ heißen \emph{kongruent} zueinander, falls es $C \in \GL{n}{K}$ mit $A = \transpose{C} B C$ gibt.
\end{definition}

Kongruente Matrizen stellen also die gleiche Bilinearform bezüglich verschiedener Basen dar, d.h.\ für $n = \dim V$ sind zwei Matrizen $A, B \in \matrices{n}{K}$ genau dann kongruent, falls es eine Bilinearform $\beta \in \BilForm(V)$ sowie Basen $\basis{A}$ und $\basis{B}$ von $V$ gibt, so dass $A = \repmatrixbilone{\beta}{\basis{A}}$ und $B = \repmatrixbilone{\beta}{\basis{B}}$ gelten.

\begin{corollary}
  Konkruenz ist eine Äquivalenzrelation auf $\matrices{n}{K}$.
\end{corollary}

Da der Rang einer Matrix invariant unter Kongruenz ist, ergibt die folgende Definition Sinn:

\begin{definition}
  Der \emph{Rang} einer Bilinearform $\beta \in \BilForm(V)$ ist $\rank \beta \coloneqq \rank \repmatrixbilone{\beta}{\basis{B}}$, wobei $\basis{B}$ eine Basis von $V$ ist.
\end{definition}





\section{Existenz von Orthogonalbasen}

Es sei $\ringchar{K} \neq 2$ und $\beta \in \SymForm(V)$ eine symmetrische Bilinearform.

\begin{theorem}
  \label{theorem: exstience of an orthogonal basis}
  Es gibt eine Orthogonalbasis von $V$ bezüglich $\beta$.
\end{theorem}

\begin{corollary}
  \label{corollary: every symmetric matrix is congruent to a diagonal matrix}
  Für jede symmetrische Matrix $A \in \matrices{n}{K}$ gibt es eine Basiswechselmatrix $S \in \GL{n}{K}$, so dass $\transpose{S} A S$ eine Diagonalmatrix ist.
\end{corollary}

Ist $A \in \matrices{n}{K}$ eine symmetrische Matrix, so lässt sich eine Diagonalform wie in Korollar~\ref{corollary: every symmetric matrix is congruent to a diagonal matrix} sowie eine zugehörige Basiswechselmatrix $S \in \GL{n}{K}$ mithilfe von \emph{simultanen Zeilen- und Spaltenumformungen} berechnen:
\begin{itemize}
  \item
    Man bringe die Matrix durch elementare Zeilenumformungen in Zeilenstufenform, und somit insbesondere in obere Dreiecksform.
  \item
    Nach jeder elementaren Zeilenumformung führt die man die analoge elementare Spaltenumformung durch.
    Hierdurch bleibt die Matrix symmetrisch.
\end{itemize}
Als Ergebnis erhält man somit eine Matrix symmetrische Matrix $B$, die in oberer Dreiecksform ist.
Also ist $B$ eine Diagonalmatrix.
Eine Matrix $S \in \GL{n}{K}$ mit $\transpose{C} A C = B$ erhält man, indem man die genutzen elementaren Spaltenumformungen in gleicher Reihenfolge auf die Einheitsmatrix $\Id$ anwendet.
(Nutzt man anstelle der Spaltenumformungen die Zeilenumformungen, so erhält man stattdessen $\transpose{C}$.)

% TODO: Adding an example





\section{Reelle symmetrische Bilinearformen}

Für diesen Unterabschnitt sei $K = \Real$



\subsection{Sylvesterscher Trägheitssatz}

Es sei $\beta \in \SymForm(V)$ eine symmetrischen Bilinearform.

Ist $\basis{B} = (v_1, \dotsc, v_n)$ eine Orthogonalbasis von $V$ bezüglich $\beta$, so ist die darstellende Matrix $\repmatrixbilone{\beta}{\basis{B}}$ von der Form
\begin{equation}
  \label{equation: bilinear form in diagonal form}
    \repmatrixbilone{\beta}{\basis{B}}
  = \begin{pmatrix}
      d_1 &         &     \\
          & \ddots  &     \\
          &         & d_n
    \end{pmatrix},
\end{equation}
wobei $d_i = \beta(v_i, v_i)$ gilt.
Sind $\mu_1, \dotsc, \mu_n \in K$ mit $\mu_i \neq 0$, so ist auch $\basis{B}' = (w_1, \dotsc, w_n)$ mit $w_i = \mu_i v_i$ eine Basis von $V$, und es gilt
\[
  \repmatrixbilone{\beta}{\basis{B'}}
  = \begin{pmatrix}
      \mu_1^2 d_1 &         &             \\
                  & \ddots  &             \\
                  &         & \mu_n^2 d_n
    \end{pmatrix}
\]
Somit lassen sich in \eqref{equation: bilinear form in diagonal form} die Diagonaleinträge um Quadratzahlen aus $\Real$ abändern.
Hiermit erhält man aus Satz~\ref{theorem: exstience of an orthogonal basis} den Sylvesterschen Trägheitssatz:

\begin{corollary}[Sylvesterscher Trägheitssatz]
  \label{corollary: Sylvester}
  Es sei $K = \Real$.
  Dann existiert eine Basis $\basis{B}$ von $V$, so dass
  \[
      \repmatrixbilone{\beta}{\basis{B}}
    = \begin{pmatrix}
        \Id_p &         &     \\
              & -\Id_q  &     \\
              &         & 0_r
      \end{pmatrix}
  \]
  gilt.
  Die Zahlen $p$, $q$ und $r$ sind eindeutig bestimmt durch
  \begin{align*}
        p
    &=  \max  \{
                \dim U
              \suchthat
                \text{$U \subseteq V$ ist ein UVR, so dass $\beta|_{U \times U}$ positiv definit ist}
              \}
  \shortintertext{und}
        q
    &=  \max  \{
                \dim U
              \suchthat
                \text{$U \subseteq V$ ist ein UVR, so dass $\beta|_{U \times U}$ negativ definit ist}
              \},
  \end{align*}
  sowie durch $p + q + r = n$.
  Zudem gilt $\rank \beta = p + q$.
\end{corollary}

\begin{definition}
  In der Situation von Korollar~\ref{corollary: Sylvester} heißt die Zahl $p - q$ die \emph{Signatur} von $\beta$.
\end{definition}

Man bemerke, dass sich die Zahlen $p$, $q$ und $r$ aus dem Sylvesterschen Trägheitssatz bereits aus der Form \eqref{equation: bilinear form in diagonal form} ablesen lassen:
Die Zahl $p$ ist die Anzahl der positiven Diagonaleinträge, die Zahl $q$ ist die Anzahl der negativen Diagonaleinträge, und $r$ ist die Häufigkeit von $0$ als Diagonaleintrag.



\subsection{Hauptachsentransformation}

Ist $\beta \in \SymForm(\Real^n)$ eine symmetrische Bilinearform, so gilt für die symmetrische Matrix $A \in \matrices{n}{\Real}$ mit $A_{ij} = \beta(e_i, e_j)$, dass
\[
    \beta(x,y)
  = \transpose{x} A y
  \qquad
  \text{für alle $x, y \in \Real^n$}.
\]
Nach Korollar~\ref{corollary: hermitian matrices are diagonalizable} gibt es eine Orthonormalbasis $\basis{B} = (v_1, \dotsc, v_n)$ von $\Real^n$ bestehend aus Eigenvektoren von $A$.
Ist $\lambda_i$ der zu $v_i$ gehörige Eigenwert, so gilt
\[
    \beta(v_i, v_j)
  = \transpose{v_i} A v_j
  = \lambda_j \transpose{v_i} v_j
  = \lambda_j \bil{v_i}{v_j}
  = \lambda_j \delta_{ij}
  = \lambda_i \delta_{ij}.
\]
Somit ist $\basis{B}$ eine Orthogonalbasis von $\Real^n$ bezüglich $\beta$ und
\[
    \repmatrixbilone{\beta}{\basis{B}}
  = \begin{pmatrix}
      \lambda_1 &         &           \\
                & \ddots  &           \\
                &         & \lambda_n
    \end{pmatrix}.
\]

\begin{proposition}
  Es seien $p$, $q$ und $r$ die Zahlen aus dem Sylvesterscher Trägheitssatz für $\beta$.
  Dann ist $p$ die Anzahl der positiven Eigenwerte von $A$, $q$ die Anzahl der negativen Eigenwerte, und $r$ die Vielfachheit des Eigenwerts $0$.
\end{proposition}

% TODO: Adding examples.

\begin{lemma}[Hauptminorenkriterium]
  Eine symmetrische Matrix $A \in \matrices{n}{\Real}$ ist genau dann positiv, wenn alle führenden Hauptminoren positiv sind.
\end{lemma}





\section{Dualität}

Es sei $W$ ein weiterer endlichdimensionaler $K$-Vektorraum



\subsection{Nicht-ausgeartetet Bilinearformen}

Es sei $\beta \in \BilForm(V, W)$ eine Bilinearform.
Die Bilinearform $\beta$ entspricht einer linearen Abbildung
\[
          \beta_1
  \colon  W
  \to     \dual{V},
  \quad
          w
  \mapsto \beta(-,w),
\]
sowie einer linearen Abbildung
\[
          \beta_2
  \colon  V
  \to     \dual{W},
  \quad
          v
  \mapsto \beta(v,-).
\]
Ist $\basis{B} = (v_1, \dotsc, v_n)$ eine Basis von $V$ und $\basis{C} = (w_1, \dotsc, w_m)$ eine Basis von $W$, so können wir als Verallgemeinerung von Definition~\ref{definition: representing matrix for a bilinear form} die darstellende Matrix
\[
            \repmatrixbiltwo{\beta}{\basis{B}}{\basis{C}}
  \coloneqq \begin{pmatrix}
              \beta(v_1, w_1) & \beta(v_1, w_2) & \cdots  & \beta(v_1, w_m) \\
              \beta(v_2, w_1) & \beta(v_2, w_2) & \cdots  & \beta(v_2, w_m) \\
              \vdots          & \vdots          & \ddots  & \vdots          \\
              \beta(v_n, w_1) & \beta(v_n, w_2) & \cdots  & \beta(v_n, w_m)
            \end{pmatrix}.
\]
betrachten.
Dann gilt bezüglich der dualen Basen $\dual{\basis{B}}$ von $\dual{V}$ und $\dual{\basis{C}}$ on $\dual{W}$, dass
\[
    \repmatrixhom{\beta_1}{\basis{C}}{\dual{\basis{B}}}
  = \repmatrixbiltwo{\beta}{\basis{B}}{\basis{C}}
  \quad\text{und}\quad
    \repmatrixhom{\beta_2}{\basis{B}}{\dual{\basis{C}}}
  = \transpose{ \repmatrixbiltwo{\beta}{\basis{B}}{\basis{C}} }
\]
Insbesondere gelten die Äquivalenzen
\begin{align*}
      &\, \text{$\beta_1$ ist ein Isomorphismus}                                                \\
  \iff&\, \text{$\repmatrixhom{\beta_1}{\basis{C}}{\dual{\basis{B}}}$ ist invertierbar}         \\
  \iff&\, \text{$\repmatrixbiltwo{\beta}{\basis{B}}{\basis{C}}$ ist invertierbar}               \\
  \iff&\, \text{$\transpose{ \repmatrixbiltwo{\beta}{\basis{B}}{\basis{C}} }$ ist invertierbar} \\
  \iff&\, \text{$\repmatrixhom{\beta_2}{\basis{B}}{\dual{\basis{C}}}$ ist invertierbar}         \\
  \iff&\, \text{$\beta_2$ ist ein Isomorphismus}.
\end{align*}

\begin{definition}
  Eine Bilinearform $\beta \in \BilForm(V,W)$ heißt \emph{nicht-ausgeartet}, falls Sie eine (und damit alle) der folgenden äquivalenten Bedingungen erfüllt:
  \begin{enumerate}
    \item
      Die lineare Abbildung $\beta_1$ ist ein Isomorphismus.
    \item
      Die lineare Abbildung $\beta_2$ ist ein Isomorphismus.
    \item
      Es gibt eine Basis $\basis{B}$ von $V$ und $\basis{C}$ von $W$, so dass die darstellende Matrix $\repmatrixbiltwo{\beta}{\basis{B}}{\basis{C}}$ invertierbar ist.
    \item
      Für jede Basis $\basis{B}$ von $V$ und $\basis{C}$ von $W$ ist die darstellende Matrix $\repmatrixbiltwo{\beta}{\basis{B}}{\basis{C}}$ invertierbar.
    \item
      Es gelten (mindestes) zwei (und damit bereits alle drei) der folgenden Bedingungen:
      \begin{enumerate}
        \item
          Es gilt $\dim V = \dim W$.
        \item
          Für jedes $v \in V$ mit $v \neq 0$ gibt es ein $w \in W$ mit $\beta(v,w) \neq 0$ (d.h.\ $\beta_2$ ist injektiv).
        \item
          Für jedes $w \in W$ mit $w \neq 0$ gibt es ein $v \in V$ mit $\beta(v,w) \neq 0$ (d.h.\ $\beta_1$ ist injektiv).
      \end{enumerate}
  \end{enumerate}
\end{definition}

Wir betrachten nun den Fall, dass $\beta \in \SymForm(V)$ eine symmetrische Bilinearform auf $V$ ist.
Für die beiden zugehörigen linearen Abbildungen $\beta_1, \beta_2 \colon V \to V^*$ gilt dann $\beta_1 = \beta_2$.

\begin{definition}
  Das \emph{Radikal} von $\beta$ ist der Untervektorraum
  \[
              \rad{\beta}
    \coloneqq \{v \in V \suchthat \text{$\beta(v,v') = 0$ für alle $v' \in V$}.
  \]
\end{definition}

Es gilt $\rad{\beta} = \ker \beta_1 = \ker \beta_2$, weshalb $\beta$ genau dann nicht ausgeartet ist, wenn $\rad{\beta} = 0$ gilt.



\subsection{Die Adjungierte Abbildung}

Es seien $\beta \in \BilForm(V,V')$ und $\gamma \in \BilForm(W,W')$ zwei nicht-ausgeartete Bilinearformen.
Dann gibt es für jede lineare Abbildung $f \colon V \to W$ eine eindeutige lineare Abbildung $\adj{f} \colon W' \to V'$ mit
\begin{equation}
  \label{equation: definition of the adjoint map by elements for non degenarete bilinear forms}
    \gamma(f(v), w')
  = \beta(v, \adj{f}(w'))
  \qquad
  \text{für alle $v \in V$, $w' \in W'$}
\end{equation}
gilt.
Dies ergibt sich wie bereits in \ref{section: adjoint map for scalar products} dadurch, dass \eqref{equation: definition of the adjoint map by elements for non degenarete bilinear forms} äquivalent zur Kommutativität des Diagramms
\[
  \begin{tikzcd}
      \dual{V}
    & \dual{W}
      \arrow[swap]{l}{\dual{f}}
    \\
      V'
      \arrow{u}{\beta_1}
    & W'
      \arrow[swap]{u}{\gamma_1}
      \arrow{l}{\adj{f}}
  \end{tikzcd}
\]
ist.
Also ist $\adj{f}$ eindeutig als $\adj{f} = \beta_1^{-1} \circ \dual{f} \circ \beta'_1$ bestimmt.













