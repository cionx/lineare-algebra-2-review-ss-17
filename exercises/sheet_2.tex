\documentclass[a4paper, 10pt]{scrartcl}
\usepackage{../styles/generalstyle}
\usepackage{../styles/exercisestyle}

\subject{Repetitorium Lineare Algebra II, SS 2017}
\title{Tag 2}
\author{}
\date{}

\begin{document}





\begin{question}[subtitle = Bestimmtheit der Jordan-Normalform]
  Man gebe gebe jeweils die größte Zahl $n \geq 1$ an, so dass die Jordan-Normalform aller ($n \times n$)-Matrizen durch die folgenden Informationen bis auf Permutation der Jordanblöcke eindeutig bestimmt ist:
  \begin{enumerate}
    \item
      Das charakteristische Polynom $\charpol{A}(t)$.
    \item
      Das Minimalpolynom $\charpol{A}(t)$.
    \item
      Die Dimension aller Eigenräume $\dim \eigenspace{(\Complex^n)}{A}{\lambda}$, $\lambda \in \Complex$.
    \item
      Das Minimalpolynom $\minpol{A}(t)$ und die Dimension aller Eigenräume $\dim \eigenspace{(\Complex^n)}{A}{\lambda}$.
  \end{enumerate}
\end{question}





% \begin{question}
%   Es seien $A, B \in \matrices{n}{\Complex}$ mit $\dim \geigenspace{(\Complex^n)}{A}{\lambda} \leq 3$ für alle $\lambda \in \Complex$.
%   Zeigen Sie, dass $A$ und $B$ genau dann ähnlich sind, wenn $\charpol{A} = \charpol{B}$ und $\minpol{A} = \minpol{B}$ gelten.
% \end{question}





\begin{question}[subtitle = Ähnlichkeitsklassen nilpotenter Matrizen]
  Bestimmen Sie die Anzahl der Ähnlichkeitsklassen nilpotenter Matrizen in $\matrices{6}{K}$.
\end{question}





\begin{question}[subtitle = Jordan--Chevalley-Zerlegung reeller Matrizen]
  \begin{enumerate}
    \item
      Es sei $A = D + N$ mit $D, N \in \matrices{n}{\Complex}$ die Jordan--Chevalley-Zerlegung einer Matrix $A \in \matrices{n}{\Real}$.
      Zeigen Sie, dass bereits $D, N \in \matrices{n}{\Real}$ gilt.
      \\
      (\emph{Tipp}:
       Nutzen Sie komplexe Konjugation.)
    \item
      Über $\Real$ besitzt nicht jede Matrix eine Jordan-Normalform, und somit auch nicht jede Matrix eine Jordan--Chevalley-Zerlegung.
      Wieso steht dies nicht im Widerspruch zu der obigen Aussage?
  \end{enumerate}
\end{question}





\begin{question}[subtitle = Implizites Bestimmen von Jordan-Normalformen]
  Bestimmen Sie für eine Matrix $A \in \matrices{n}{\Complex}$, $n \geq 1$ mit den angegebenen Eigenschaften jeweils alle möglichen Jordan-Normalformen bis auf Permutation der Jordanblöcke.
  \begin{enumerate}
    \item
      Es gelten $\charpol{A}(t) = (t-4)^3 (t+3)^2$ und $\minpol{A}(t) = (t-4) (t+3)^2$.
    \item
      $A\in \matrices{2}{\Complex}$ ist nicht diagonalisierbar mit $\tr A = 0$.
    \item
      Es gilt $A^3 = 0$ und alle nicht-trivialen Eigenräume von $A$ sind eindimensional.
    \item
      Es gelten $\charpol{A}(t) = (t-2)(t+2)^3$ und $(A - 2 \Id)(A + 2 \Id) = 0$.
    \item
      Es gilt $\charpol{A}(t) = t^3 - t$.
    \item
      Es gilt $\charpol{A}(t) = (t^2 - 5t + 6)^2$ und alle Eigenräume von $A$ sind entweder null- oder eindimensional.
    \item
      Es gilt $A^2 = A$ und alle nicht-trivialen Eigenräume von $A$ sind zweidimensional.
%     \item
%       Es gilt $\charpol{A}(t) = t^5$, alle Eigenräume von $A$ sind entweder null- oder zweidimensional.
    \item
      Es gilt $\charpol{A}(t) = (t + 3)^3 t^2$ und $A$ hat keine zweidimensionalen Eigenräume.
    \item
      Es gilt $\charpol{A}(t) = t^5 -2 t^4$ und alle nicht-trivialen Eigenräume von $A$ haben die gleiche Dimension.
%     \item
%       Es gilt $\charpol{A}(t) = (t-3)^4 (t-5)^4$ und $(A - 3 \Id)^2 (A - 5 \Id)^2 = 0$.
    \item
      $A \in \matrices{3}{\Complex}$ mit $\tr A = \det A = 0$.
    \item
      $A \in \matrices{8}{\Complex}$ mit $(A - \Id)(A^5 - A^4) = 0$, $\tr A = 2$ und $\rank A = 6$.
%     \item
%       $A \in \matrices{7}{\Complex}$ mit $A^5 = A^4$, $\tr A = 2$ und $\rank A = \rank A^2 + 1 = 4$.
%     \item
%       $A \in \matrices{6}{\Complex}$ mit $\minpol{A}(t) = (t-1)^2 (t-2)^3$ und $\dim {\eigenspace{(\Complex^6)}{A}{2}} = 1$.
%     \item
%       Es gelten $\charpol{A} = (t-1)^2 (t-2) (t-7)^2$, $\tr A = 6$, $\det A = 4$, $\rank (A - \Id) = 3$.
%     \item
%       $A \in \matrices{5}{\Complex}$ mit $A^3 = 0$ und $\rank A = 3$.
%     \item
%       $A \in \matrices{7}{\Complex}$ mit $A^7 - 2 A^5 + A^3 = 0$, $\rank A = \rank(A + \Id) = 6$, $\tr A = 0$.
%     \item
%       Es gelten $\charpol{A}(t) = (t+2)^4 (t-1)^2$, $\rank (A + 2 \Id) > \rank (A + 2 \Id)^2 = 2$ und $\rank (A - \Id) = 5$.
  \end{enumerate}
\end{question}





\begin{question}[subtitle = Jordan-Normalform in Abhängigkeit von einem Parameter]
  Bestimmen Sie für jedes $a \in \Real$ die Jordan-Normalform und das Minimalpolynom der Matrix
  \[
              A_a
    \coloneqq \begin{pmatrix}
                2 & a+1 &  0  \\
                0 &  2  & a-2 \\
                0 &  0  &  2
              \end{pmatrix}.
  \]
\end{question}











% \begin{question}
%   Bestimmen Sie für die folgenden Matrizen jeweils die Jordan-Normalform:
%   \begin{gather*}
%               A_1
%     \coloneqq \begin{pmatrix*}[r]
%                 0 & 0 &  1  \\
%                 1 & 0 & -3  \\
%                 0 & 1 &  3
%               \end{pmatrix*}
%               \in \matrices{3}{\Real},
%     \quad
%               A_2
%     \coloneqq \begin{pmatrix*}[r]
%                 0 & 0 & 1 \\
%                 1 & 1 & 1 \\
%                 1 & 0 & 0
%               \end{pmatrix*}
%               \in \matrices{3}{\Field_2},
%     \\
%               A_3
%     \coloneqq \begin{pmatrix*}[r]
%                 1 & 0 & 1 & -1  \\
%                 0 & 1 & 1 &  0  \\
%                 0 & 0 & 1 &  1  \\
%                 0 & 0 & 0 &  1
%               \end{pmatrix*}
%               \in \matrices{4}{\Real},
%     \quad
%               A_4
%     \coloneqq \begin{pmatrix*}[r]
%                 -6  & 12  & 4 \\
%                 -2  &  2  & 1 \\
%                 -3  & 20  & 6
%               \end{pmatrix*}
%               \in \matrices{3}{\Real},
%     \\
%               A_5
%     \coloneqq \begin{pmatrix}
%                 0 & 0 & 1 & 2 \\
%                 0 & 1 & 2 & 1 \\
%                 2 & 1 & 0 & 0 \\
%                 0 & 2 & 2 & 0
%               \end{pmatrix}
%               \in \matrices{4}{\Field_3}.
%   \end{gather*}
% \end{question}





% \begin{question}
%   Es sei $V$ ein $K$-Vektorraum, und $f, g \colon V \to V$ seien zwei kommutierende Endomorphismen.
%   Zeigen Sie, dass $\eigenspace{V}{f}{\lambda}$ und $\geigenspace{V}{f}{\lambda}$ invariant unter $g$ sind.
% \end{question}




\end{document}
