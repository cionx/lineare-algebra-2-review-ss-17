\documentclass[a4paper, 10pt]{scrartcl}
\usepackage{../styles/generalstyle}
\usepackage{../styles/exercisestyle}

\subject{Repetitorium Lineare Algebra II, SS 2017}
\title{Tag 5}
\author{}
\date{}

\begin{document}





\begin{question}
  Es sei $\ringchar{K} \neq 2$ und $\beta \in \SymForm(V)$ eine symmetrische Bilinearform.
  \begin{enumerate}
    \item
      Zeigen Sie, dass es im Fall $\beta \neq 0$ ein $v \in V$ mit $\beta(v,v) \neq 0$ gibt.\\
      (\emph{Hinweis}:
       Betrachten Sie die Polarisationsformeln.)
       
      
    \item
      Zeigen Sie, dass das orthogonale Komplement $\dim \generated{v}^\perp = \dim V - 1$ gilt.\\
      (\emph{Hinweis}:
       Betrachten Sie die lineare Abbildung $\beta(-, v) \colon V \to K$.) 
    \item
      Folgern Sie, dass $V = \generated{v} \oplus \generated{v}^\perp$ gilt.
    \item
      Zeigen Sie induktiv, dass es eine Orthogonalbasis von $V$ bezüglich $\beta$ gibt.
  \end{enumerate}
\end{question}





\begin{question}
  Zeigen Sie, dass
  \[
            \beta
    \colon  K^2 \times K^2
    \to     K,
    \quad\text{mit}\quad
      \beta\left( \vect{x_1 \\ x_2}, \vect{y_1 \\ y_2} \right)
    = \det
      \begin{pmatrix}
        x_1 & y_1 \\
        x_2 & y_2
      \end{pmatrix}
  \]
  eine nicht-ausgeartete Bilinearform auf $K^2$ definiert.
\end{question}





\begin{question}
  Es sei $\beta \in \SymForm(\matrices{n}{\Real})$ die nicht-ausgeartete Bilinearform mit $\beta(A,B) = \tr(AB)$.
  \begin{enumerate}
    \item
      Zeigen Sie, dass die Untervektorräume $\Sym{n}{\Real}$ und $\Alt{n}{\Real}$ von $\matrices{n}{\Real}$ bezüglich $\beta$ orthogonal zueinander sind.
    \item
      Zeigen Sie, dass die Einschränkung $\beta|_{\Sym{n}{\Real} \times \Sym{n}{\Real}}$ positiv definit ist, und die Einschränkung $\beta|_{\Alt{n}{\Real} \times \Alt{n}{\Real}}$ negativ definit.
    \item
      Bestimmen Sie die Signatur von $\beta$.
    \item
      Bestimmen Sie eine Orthogonalbasis von $\matrices{2}{\Real}$ bezüglich $\beta$.
  \end{enumerate}
\end{question}





\begin{question}
  Es seien $V$, $W$ zwei $K$-Vektorräume und $\beta_V \in \BilForm(V)$, $\beta_W \in \BilForm(W)$ nicht-ausgeartete Bilinearformen.
  Es sei $f \colon V \to W$ ein Funktion mit
  \[
      \beta_W( f(v_1), f(v_2) )
    = \beta_V( v_1, v_2 )
    \qquad
    \text{für alle $v_1, v_2 \in V$}.
  \]
  Zeigen Sie, dass $f$ ein Isomorphismus ist.
\end{question}





\begin{question}
  Es sei $\ringchar{K} \neq 2$, $V$ ein endlichdimensionaler $K$-Vektorraum und $\beta \in \SymForm(V)$ eine nicht-ausgeartete symmetrische Bilinearform.
  Es sei $U \subseteq V$ ein total isotroper Untervektorraum, d.h.\ es gelte $\beta(u,u) = 0$ für alle $u \in U$.
  Zeigen Sie, dass $\dim U \leq \frac{1}{2} \dim V$ gilt.
  (\emph{Hinweis}:
   Zeigen Sie zunächst, dass $U \subseteq U^\perp$ gilt.)
\end{question}





\begin{question}
  Bestimmen Sie für die gegebenen quadratischen Formen auf $\Real^n$ jeweils die Signatur der zugehörigen symmetrischen Bilinearform.
  \begin{enumerate}
    \item
      $q(x_1, x_2) = 2 x_1^2 - 3 x_2^2 + 2 x_1 x_2$.
    \item
      $q(x_1, x_2) = -x_1^2 + x_2^2 + a x_1 x_2$ mit $a \in \Real$.
    \item
      $q(x_1, x_2) = x_1^2 + 15 x_2^2 + 6 x_1 x_2$.
    \item
      $q(x_1, x_2) = 2 x_1 x_2$.
    \item
      $q(x_1, x_2, x_3) = x_1^2 + 2 x_1 x_2 - 2 x_1 x_3 + x_2^2 - 2 x_2 x_3 - x_3^2$.
    \item
      $q(x_1, x_2, x_3, x_4) = x_1^2 - 7 x_2^2 - x_3^2 - x_4^2 + 2 x_1 x_2 - 6 x_2 x_3 + 6 x_2 x_4 + 2 x_3 x_4$.
  \end{enumerate}
\end{question}





\begin{question}
  Es sei
  \[
              A
    \coloneqq \begin{pmatrix}
                  &         & 1 \\
                  & \iddots &   \\
                1 &         &
              \end{pmatrix}
    \in \matrices{n}{\Real}.
  \]
  Bestimmen Sie Rang und Signatur von $\beta \in \SymForm(\Real^n)$ mit
  \[
              \beta(x,y)
    \coloneqq \transpose{x} A y
    \qquad
    \text{für alle $x, y \in \Real^n$}.
  \]
\end{question}





\begin{question}
  Für alle $n \geq 1$ seien
  \[
              A_n
    \coloneqq \begin{pmatrix}
                2 & 1 &         &         &   \phantom{\ddots}  \\
                1 & 2 & 1       &         &   \phantom{\ddots}  \\
                  & 1 & \ddots  & \ddots  &   \phantom{\ddots}  \\
                  &   & \ddots  & \ddots  & 1 \phantom{\ddots}  \\
                  &   &         &      1  & 2 \phantom{\ddots}
              \end{pmatrix}
  \]
  und $a_n \coloneqq \det A_n$.
  \begin{enumerate}
    \item
      Zeigen Sie induktiv für alle $n \geq 3$, dass $a_n = 2 a_{n-1} - a_{n-2}$ gilt.
    \item
      Zeigen Sie für alle $n \geq 2$, dass $a_{n+1} - a_n \geq a_n - a_{n-1}$ gilt.
    \item
      Folgern Sie, dass $a_n > 0$ für alle $n \geq 1$ gilt.
    \item
      Folgern Sie, dass die Matrix $A_n$ für alle $n \geq 1$ positiv definit ist.
  \end{enumerate}
\end{question}





\begin{question}
  Es seien
  \begin{gather*}
              A_1
    \coloneqq \begin{pmatrix}
                4 & 3 \\
                3 & 4
              \end{pmatrix},
    \quad    
              A_2
    \coloneqq \begin{pmatrix*}[r]
                3 &   4 \\
                4 &  -3
              \end{pmatrix*},
    \quad     A_3
    \coloneqq \begin{pmatrix*}[r]
                 3  & -2  &  0  \\
                -2  &  2  & -2  \\
                 0  & -2  &  1
              \end{pmatrix*},
    \\
              A_4
    \coloneqq \begin{pmatrix*}[r]
                 2  & -1  & 1 \\
                -1  &  2  & 1 \\
                 1  &  1  & 2
              \end{pmatrix*},
    \quad
              A_5
    \coloneqq \begin{pmatrix*}[r]
                3 & 8 &  5  \\
                8 & 3 &  0  \\
                5 & 0 & -1
              \end{pmatrix*}.
  \end{gather*}
  Bestimmen Sie jeweils den Rang und die Signatur von $A_i$, sowie eine orthogonale Matrix $U_i \in \Orthogonal{n_i}$, so dass die Matrix $\transpose{U_i} A_i U_i$ in Diagonalform ist.
\end{question}


\begin{question}
  Es sei $A \in \matrices{2}{\Real}$ symmetrisch mit $\det A < 0$.
  Bestimmen Sie den Rang und die Signatur von $A$.
\end{question}


\begin{question}
  Bestimmen Sie die Anzahl der Kongruenzklassen
  \begin{enumerate}
    \item
      symmetrischer Bilinearformen auf einem $n$-dimensionalen reellen Vektorraum,
    \item
      symmetrischer, nicht-ausgearteter Bilinearformen auf einem $n$-dimensionalen reellen Vektorraum,
    \item
      symmetrischer Bilinearformen auf einem $n$-dimensionalen komplexen Vektorraum,
    \item
      symmetrischer, nicht-ausgearteter Bilinearformen auf einem $n$-dimensionalen komplexen Vektorraum.
  \end{enumerate}
\end{question}









\end{document}
