\documentclass[a4paper, 10pt]{scrartcl}
\usepackage{../styles/generalstyle}
\usepackage{../styles/exercisestyle}

\subject{Repetitorium Lineare Algebra II, SS 2017}
\title{Tag 4}
\author{}
\date{}

\begin{document}

Im Folgenden seien alle Vektorräume endlichdimensional.





\begin{question}[subtitle = Diagonalisierbarkeit und Selbstadjungiertheit]
  \begin{enumerate}
    \item
      Es sei $V$ ein reeller Vektorraum.
      Zeigen Sie, dass es für jeden diagonalisierbaren Endomorphismus $f \colon V \to V$ ein Skalarprodukt auf $V$ gibt, bezüglich dessen $f$ selbstadjungiert ist.
    \item
      Wieso gilt die analoge Aussage für komplexe Vektorräume nicht?
  \end{enumerate}
\end{question}





\begin{question}[subtitle = Selbstadjungierte Endomorphismen]
  Es sei $V$ ein Skalarproduktraum und $f, g \colon V \to V$ seien selbstadjungierte Endomorphismen.
  \begin{enumerate}
    \item
      Zeigen Sie, dass $f = 0$ gilt, falls $f$ nilpotent ist
    \item
      Zeigen Sie, dass $f^2 = \id_V$ gilt, falls $f$ orthogonal ist.
    \item
      Zeigen Sie, dass $f = g$ gilt, falls es ein $n \geq 0$ mit $(f-g)^n = 0$ gibt.
  \end{enumerate}
\end{question}





% \begin{question}
%   Es sei $f \colon V \to V$ ein normaler Endomorphismus.
%   \begin{enumerate}
%     \item
%       Zeigen Sie, dass $\norm{f(v)} = \norm{\adj{f}(v)}$ für alle $v \in V$ gilt.
%     \item
%       Folgern Sie, dass $\ker f = \ker \adj{f}$ gilt.
%     \item
%       Zeigen Sie, dass für alle $\lambda \in \Korper$ auch $f - \lambda \id_V$ normal ist.
%     \item
%       Folgern Sie, dass für alle $\lambda \in \Korper$ die Gleichheit $\eigenspace{V}{f}{\lambda} = \eigenspace{V}{\adj{f}}{\conjugate{\lambda}}$ gilt.
%     \item
%       Folgern Sie, dass $\eigenspace{V}{f}{\lambda} \perp \eigenspace{V}{f}{\mu}$ für alle $\lambda \neq \mu$ gilt.
%   \end{enumerate}
% \end{question}





\begin{question}[subtitle = Charakterisierung antiselbstadjungierter Endomorphismen]
  Es sei $V$ ein unitärer Vektorraum und $f \colon V \to V$ ein Endomorphismus.
  Zeigen Sie, dass die folgenden Bedingungen äquivalent sind:
  \begin{enumerate}
    \item
      Der Endomorphismus $f$ ist \emph{antiselbstadjungiert}, d.h.\ es gilt $\adj{f} = -f$.
    \item
      Der Endomorphismus $f$ ist normal, und alle Eigenwerte von $f$ sind rein imaginär (d.h.\ aus $i \Real$).
  \end{enumerate}
\end{question}





% \begin{question}
%   Es sei $A \in \matrices{n}{\Real}$, so dass $\charpol{A}(t)$ in Linearfaktoren zerfällt.
%   \begin{enumerate}
%     \item
%       Zeigen Sie, dass $A$ diagonalisierbar ist, falls $A$ normal ist.
%     \item
%       Zeigen Sie, dass $A^2 = \Id$ gilt, falls $A$ orthogonal ist.
%   \end{enumerate}
% \end{question}





% \begin{question}
%   Es sei $A \in \Unitary{n}$.
%   \begin{enumerate}
%     \item
%       Zeigen Sie, dass $|\tr A| \leq n$ gilt.
%     \item
%       Zeigen Sie, dass genau dann Gleichheit gilt, wenn $A$ ein Vielfaches von $\Id$ ist.
%   \end{enumerate}
% \end{question}





\begin{question}[subtitle = Zerlegung von Matrizen]
  Es sei $A \in \matrices{n}{\Complex}$.
  Zeigen Sie:
  \begin{enumerate}
    \item
      Es gibt eindeutige hermitesche Matrizen $B, C \in \matrices{n}{\Complex}$ mit $A = B + i C$.
    \item
      $A$ ist genau dann normal, wenn $B$ und $C$ kommutieren.
    \item
      Es gibt eine eindeutige hermitesche Matrix $D \in \matrices{n}{\Complex}$ und schiefhermitesche Matrix $E \in \matrices{n}{\Complex}$ (d.h.\ $\madj{E} = -E$) mit $A = D + E$.
    \item
      $A$ genau dann normal ist, wenn $D$ und $E$ kommutieren.
    \item
      Wie hängen die beiden Zerlegungen $A = B + iC$ ud $A = D + E$ zusammen?
  \end{enumerate}
\end{question}





% \begin{question}[subtitle = Spiegelungen]
%   Es sei $V$ ein euklidischer Vektorraum.
%   Für jedes $\alpha \in V$ mit $\alpha \neq 0$ sei
%   \[
%             s_\alpha
%     \colon  V
%     \to     V,
%     \quad\text{mit}\quad
%               s_\alpha(x)
%     \coloneqq x - 2 \frac{\bil{x}{\alpha}}{\|\alpha\|^2} \alpha.
%   \]
%   Ferner seien $L_\alpha \coloneqq \Real \alpha$ und $H_\alpha \coloneqq L_\alpha^\perp$.
%   \begin{enumerate}
%     \item
%       Zeigen Sie, dass $L_\alpha = \eigenspace{V}{s_\alpha}{-1}$ und $H_\alpha = \eigenspace{V}{s_\alpha}{1}$.
%   \end{enumerate}
%   Also ist $s_\alpha$ die orthogonale Spiegelung an der Hyperebene $H_\alpha$.
%   \begin{enumerate}[resume]
%     \item
%       Zeigen Sie, dass $s_\alpha$ orthogonal ist.
%     \item
%       Zeigen Sie, dass $t s_\alpha t^{-1} = s_{t(\alpha)}$ für alle $t \in \Orthogonal{V}$ gilt.
%   \end{enumerate}
% \end{question}





% \begin{question}
%   Es sei $V$ ein euklidischer Vektorraum und $f \colon V \to V$ ein Endomorphismus.
%   Entscheiden Sie, weche Implikationen zwischen den folgenden Aussagen gelten:
%   \begin{enumerate}
%     \item
%       Der Endomorphismus $f$ ist selbstadjungiert mit positiven Eigenwerten.
%     \item
%       Der Endomorphismus $f$ ist orthogonal mit positiven Eigenwerten.
%     \item
%       Der Endomorphismus $f$ ist normal mit $\det f > 0$.
%     \item
%       Der Endomorphismus $f$ ist selbstadjungiert und orthogonal.
%   \end{enumerate}
% \end{question}





% \begin{question}
%   Bestimmen Sie alle Matrizen $A \in \Orthogonal{n}{\Real}$, die obere Dreiecksmatrizen sind.
% \end{question}





\begin{question}[subtitle = Wurzeln aus negativ semidefiniten Endomorphismen]
  Es sei $V$ ein unitärer Vektorraum und $f \colon V \to V$ ein selbstadjungierter, negativ semidefiniter Endomorphismus.
  \begin{enumerate}
    \item
      Zeigen Sie, dass es einen antiselbstadjungierten Endomorphismus $g \colon V \to V$ mit $g^2 = f$ gibt.
    \item
      Entscheiden Sie, ob $g$ eindeutig ist.
  \end{enumerate}
\end{question}





\end{document}
