\documentclass[a4paper, 10pt]{scrartcl}
\usepackage{../styles/generalstyle}
\usepackage{../styles/exercisestyle}

\subject{Repetitorium Lineare Algebra II, SS 2017}
\title{Tag 3}
\author{}
\date{}

\begin{document}

Im Folgenden seien alle Vektorräume endlich-dimensional.



\begin{question}
  Es seien $V$ ein Skalarproduktraum.
  \begin{enumerate}
    \item
      Es sei $\basis{B} = (v_1, \dotsc, v_n)$ eine Orthonormalbasis von $V$.
      Zeigen Sie für alle $v, w \in V$ mit $v = \sum_{i=1}^n x_i v_i$ und $w = \sum_{i=1}^n y_i v_i$ die Gleichungen
      \[
          \bil{v}{w}
        = \sum_{i=1}^n x_i \overline{y_i}
        \quad\text{und}\quad
          \norm{v}
        = \sqrt{ \sum_{i=1}^n \abs{x_i}^2 }.
      \]
    \item
      Zeigen Sie die Parallelogrammgleichung:
      \[
          \norm{v + w}^2 + \norm{v - w}^2
        = 2( \norm{v}^2 + \norm{w}^2 )
        \qquad
        \text{für alle $v, w \in V$}.
      \]
  \end{enumerate}
\end{question}



\begin{question}[subtitle = Existenz von Skalarprodukten]
  Es sei $\basis{B}$ eine Basis eines $\Korper$-Vektorraums $V$.
  Zeigen Sie, dass es ein eindeutiges Skalarprodukt auf $V$ gibt, bezüglich dessen $\basis{B}$ orthonormal ist.
\end{question}





% \begin{question}
%   Es sei $V$ ein endlichdimensionaler $\Korper$-Vektorraum und $\bil{-}{-}_1$ und $\bil{-}{-}_2$ seien zwei Skalarprodukte auf $V$ mit
%   \[
%           \bil{v_1}{v_2}_1 = 0
%     \iff  \bil{v_1}{v_2}_2 = 0
%     \qquad
%     \text{für alle $v_1, v_2 \in V$}.
%   \]
%   Zeigen Sie, dass es eine Konstante $c > 0$ gibt, so dass $\bil{-}{-}_2 = c \bil{-}{-}_1$ gilt.
% \end{question}





\begin{question}
  Es seien $V$ und $W$ Skalarprodukträume und es sei $f \colon V \to W$ linear.
  Zeigen Sie, dass $\ker f = (\im \adj{f})^\perp$ und $\im f = (\ker \adj{f})^\perp$ gelten.
\end{question}





\begin{question}[subtitle = Endomorphismen mit {$\bil{f(v)}{v} = 0$}]
  \begin{enumerate}
    \item
      Es sei $V$ ein Skalarproduktraum und $f \colon V \to V$ ein diagonalisierbarer Endomorphismus mit $\bil{f(v)}{v} = 0$ für alle $v \in V$.
      Zeigen Sie, dass bereits $f = 0$ gilt.
      \\
      (\emph{Tipp}:
      Betrachten Sie die Eigenwerte von $f$.)
    \item
      Es sei nun $V$ ein unitärer Vektorraum und $f \colon V \to V$ ein Endomorphismus mit $\bil{f(v)}{v} = 0$ für alle $v \in V$.
      Zeigen Sie, dass bereits $f = 0$ gilt.
      \\
      (\emph{Tipp}:
       Betrachten Sie die Sesquilinearform $\beta \in \SesForm(V)$ mit $\beta(v_1, v_2) = \bil{f(v_1)}{v_2}$.)
  \end{enumerate}
\end{question}





\begin{question}
  Es seien $V$ und $W$ Skalarprodukträume und es sei $f \colon V \to W$ linear.
  Zeigen Sie, dass die folgenden Bedingungen äquivalent sind:
  \begin{enumerate}
    \item
      Für alle $v_1, v_2 \in V$ gilt $\bil{f(v_1)}{f(v_2)} = \bil{v_1}{v_2}$.
    \item
      Für alle $v \in V$ gilt $\norm{f(v)} = \norm{v}$.
  \end{enumerate}
\end{question}





\begin{question}
  Es sei $V$ ein unitärer Vektorraum.
  \begin{enumerate}
    \item
      Zeigen Sie, dass es für jede Flagge
      \[
                    0
        =           V_0
        \subsetneq  V_1
        \subsetneq  V_2
        \subsetneq  \dotsb
        \subsetneq  V_n
        =           V
      \]
      eine Orthonormalbasis $\basis{B} = (v_1, \dotsc, v_n)$ von $V$ gibt, so dass $V_i = \generated{v_1, \dotsc, v_i}$ für alle $i$ gilt.
    \item
      Zeigen Sie, dass es für jeden Endomorphismus $f \colon V \to V$ eine Orthonormalbasis $\basis{B}$ von $V$ gibt, so dass $\repmatrixendo{f}{\basis{B}}$ in oberer Dreiecksform ist.
  \end{enumerate}
\end{question}





\begin{question}
  Es sei $A \in \matrices{n}{\Korper}$ orthogonal, unitär, bzw.\ hermitesch.
  Bestimmen Sie jeweils alle möglichen Werte von $\det A$.
\end{question}





% \begin{question}
%   Es seien $v_1, v_2 \in \Real^3$ mit
%   \[
%               v_1
%     \coloneqq \vect{1 \\ 1 \\ 0}
%     \quad\text{und}\quad
%               v_2
%     \coloneqq \vect{0 \\ 1 \\ 1}.
%   \]
%   \begin{enumerate}
%     \item
%       Zeigen Sie, dass es ein Skalarprodukt auf $\Real^3$ gibt, bezüglich dessen die Familie $(v_1, v_2)$ orthonormal ist.
%     \item
%       Geben Sie eine Matrix $B \in \matrices{3}{\Real}$ an, so dass die Bilinearform $\bil{-}{-}_B \in \BilForm(\Real^3)$ mit
%       \[
%                   \bil{x}{y}_B
%         \coloneqq \transpose{x} B y
%         \qquad
%         \text{für alle $x, y \in \Real^3$}
%       \]
%       ein solches Skalarprodukt ist.
%   \end{enumerate}
% \end{question}




\end{document}
