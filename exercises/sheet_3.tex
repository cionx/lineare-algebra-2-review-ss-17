\documentclass[a4paper, 10pt]{scrartcl}
\usepackage{../styles/generalstyle}
\usepackage{../styles/exercisestyle}

\subject{Repetitorium Lineare Algebra II, SS 2017}
\title{Tag 3}
\author{}
\date{}

\begin{document}





\begin{question}
  Es sei $\basis{B}$ eine Basis von $V$.
  Zeigen Sie, dass es ein eindeutiges Skalarprodukt auf $V$ gibt, so dass $\basis{B}$ orthonormal ist.
\end{question}





\begin{question}
  Es sei $V$ ein endlichdimensionaler $\Korper$-Vektorraum und $\bil{-}{-}_1$ und $\bil{-}{-}_2$ seien zwei Skalarprodukte auf $V$ mit
  \[
          \bil{v_1}{v_2}_1 = 0
    \iff  \bil{v_1}{v_2}_2 = 0
    \qquad
    \text{für alle $v_1, v_2 \in V$}.
  \]
  Zeigen Sie, dass es eine Konstante $c > 0$ gibt, so dass $\bil{-}{-}_2 = c \bil{-}{-}_1$ gilt.
\end{question}





\begin{question}
  Es seien $v_1, v_2 \in \Real^3$ mit
  \[
              v_1
    \coloneqq \vect{1 \\ 1 \\ 0}
    \quad\text{und}\quad
              v_2
    \coloneqq \vect{0 \\ 1 \\ 1}.
  \]
  \begin{enumerate}
    \item
      Zeigen Sie, dass es ein Skalarprodukt auf $\Real^3$ gibt, bezüglich dessen die Familie $(v_1, v_2)$ orthonormal ist.
    \item
      Geben Sie eine Matrix $B \in \matrices{3}{\Real}$ an, so dass die Bilinearform $\bil{-}{-}_B \in \BilForm(\Real^3)$ mit
      \[
                  \bil{x}{y}_B
        \coloneqq \transpose{x} B y
        \qquad
        \text{für alle $x, y \in \Real^3$}
      \]
      ein solches Skalarprodukt ist.
  \end{enumerate}
\end{question}





\begin{question}
  Es seien $V$ und $W$ endlichdimensionale Skalarprodukträume und es sei $f \colon V \to W$ linear.
  Zeigen Sie, dass $\ker f = (\im \adj{f})^\perp$ und $\im f = (\ker \adj{f})^\perp$ gelten.
\end{question}





\begin{question}
  Es sei $V$ ein endlichdimensionaler unitärer Vektorraum.
  \begin{enumerate}
    \item
      Zeigen Sie, dass es für jede Flagge
      \[
                    0
        =           V_0
        \subsetneq  V_1
        \subsetneq  V_2
        \subsetneq  \dotsb
        \subsetneq  V_n
        =           V
      \]
      eine Orthonormalbasis $\basis{B} = (v_1, \dotsc, v_n)$ von $V$ gibt, so dass $V_i = \generated{v_1, \dotsc, v_i}$ für alle $i$ gilt.
    \item
      Zeigen Sie, dass es für jeden Endomorphismus $f \colon V \to V$ eine Orthonormalbasis $\basis{B}$ von $V$ gibt, so dass $\repmatrixendo{f}{\basis{B}}$ in oberer Dreiecksform ist.
  \end{enumerate}
\end{question}





\begin{question}
  Es sei $A \in \matrices{n}{\Korper}$
  \begin{enumerate}
    \item
      orthogonal,
    \item
      unitär,
    \item
      hermitesch.
  \end{enumerate}
  Bestimmen Sie jeweils alle möglichen Werte von $\det A$.
\end{question}






\end{document}
