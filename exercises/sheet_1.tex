\documentclass[a4paper, 10pt]{scrartcl}
\usepackage{../styles/generalstyle}
\usepackage{../styles/exercisestyle}

\subject{Repetitorium Lineare Algebra II, SS 2017}
\title{Tag 1}
\author{}
\date{}

\begin{document}





\begin{question}[subtitle = Eigenwerte und Diagonalisierbarkeit]
  \begin{enumerate}
    \item
      Es sei $A \in \GL{2}{\Complex}$ mit $\tr A = 0$.
      Zeigen Sie, dass $A$ diagonalisierbar ist.
    \item
      Zeigen Sie, dass jede Matrix $A \in \matrices{3}{\Real}$ einen reellen Eigenwert hat.
    \item
      Folgern Sie, dass jede nicht-triagonalisierbare Matrix $A \in \matrices{3}{\Real}$ über $\Complex$ diagonalisierbar ist.
      \\
      (\emph{Tipp}:
       Zeigen Sie, dass für jeden Eigenwert $\lambda$ von $A$ auch $\conjugate{\lambda}$ ein Eigenwert ist.)
    \item
      Es sei $A \in \matrices{n}{\Complex}$ und $k \geq 0$ mit $A^k = \Id$.
      Zeigen Sie, dass $A$ diagonalisierbar ist, und bestimmen Sie alle möglichen Eigenwerte für $A$.
    \item
      Es sei $A \in \matrices{2}{\Complex}$ mit $\tr A = 0$ und $\tr A^2 = -2$.
      Bestimmen Sie $\det A$.
      Entscheiden Sie auch, ob $A$ diagonalisierbar ist.
    \item
      Es sei $A \in \matrices{2}{\Complex}$ mit $\tr A = 2$ und $\tr A^2 = 4$.
      Zeigen Sie, dass $A$ diagonalisierbar ist, und bestimmen Sie die Eigenwerte von $A$.
    \item
      Es sei $A \in \matrices{n}{\Complex}$ mit $A^2 + A = 6 \Id$ und $\det A = 144$.
      Bestimmen Sie $n$.
    \item
      Es sei $A \in \matrices{n}{\Complex}$ mit $A^3 = 3A - 2$ und $A^3 + A^2 = A + \Id$.
      Zeigen Sie, dass $A = \Id$.
  \end{enumerate}
\end{question}





\begin{question}[subtitle = Determinante und Potenzen der Spur]
  Zeigen Sie für alle $A \in \matrices{3}{\Complex}$ die Gleichheit
  \[
      \det A
    =   \frac{1}{6} (\tr A)^3
      - \frac{1}{2} (\tr A^2)(\tr A)
      + \frac{1}{3} (\tr A^3).
  \]
\end{question}





\begin{question}[subtitle = Diagonalisieren]
  \begin{enumerate}
    \item
      Es sei
      \[
                  A
        \coloneqq \begin{pmatrix}
                    2       & 1       & \cdots  & 1       \phantom{\ddots}  \\
                    1       & \ddots  & \ddots  & \vdots  \phantom{\ddots}  \\
                    \vdots  & \ddots  & \ddots  & 1       \phantom{\ddots}  \\
                    1       & \cdots  & 1       & 2       \phantom{\ddots}
                  \end{pmatrix}
        \in       \matrices{n}{\Real}.
      \]
      Bestimmen Sie die Eigenwerte von $A$ und zeigen Sie, dass $A$ diagonalisierbar ist.
    \item
      Es sei
      \[
                  A
        \coloneqq \begin{pmatrix}
                    0     & \Id_n \\
                    \Id_n & 0
                  \end{pmatrix}
        \in       \matrices{2n}{\Real}.
      \]
      Geben Sie eine Basis von $K^{2n}$ aus Eigenvektoren von $A$ an.
      Bestimmen Sie anschließend $\charpol{A}(t)$ sowie $\det A$.
      
      (\emph{Tipp}: $A$ vertauscht die Basisvektoren $e_i$ und $e_{n+i}$.)
  \end{enumerate}
\end{question}





\begin{question}[subtitle = Wurzeln und Potenzen]
  Es seien
  \[
              A
    \coloneqq \begin{pmatrix*}[r]
                7 & -12 \\
                4 &  -7
              \end{pmatrix*}
              \in \matrices{2}{\Complex}
    \quad\text{und}\quad
              B
    \coloneqq \frac{1}{\sqrt{2}}
              \begin{pmatrix}
                1 & i \\
                i & 1
              \end{pmatrix}
              \in \matrices{2}{\Complex}.
  \]
  \begin{enumerate}
    \item
      Geben Sie eine Matrix $C \in \matrices{2}{\Complex}$ mit $A = C^2$ an.
    \item
      Berechnen Sie $B^{2017}$.
      \\
      (\emph{Tipp}:
       Ignorieren Sie ggf.\ zunächst den Vorfaktor $1/\sqrt{2}$.)
  \end{enumerate}
\end{question}





\begin{question}[subtitle = Cayley-Hamilton]
  Es sei $K$ ein Körper.
  \begin{enumerate}
    \item
      Zeigen Sie für $A \in \matrices{n}{K}$, dass die Potenzen $\Id, A, A^2, \dotsc, A^n$ linear abhängig sind.
    \item
      Es sei $A \in \GL{n}{K}$.
      Zeigen Sie, dass es ein Polynom $p \in K[t]$ mit $p(A) = A^{-1}$ gibt.
      Bestimmen Sie ein solches Polynom für die Matrix
      \[
                  A
        \coloneqq \begin{pmatrix}
                    1 & 0 & 1 \\
                    1 & 1 & 1 \\
                    0 & 0 & 1
                  \end{pmatrix}
        \in \GL{3}{\Real}.
      \]
  \end{enumerate}
\end{question}





\begin{question}[subtitle = Simultane Diagonalisierbarkeit]
  \begin{enumerate}
    \item
      Es sei $V$ ein endlichdimensionaler $K$-Vektorraum, und es seien $f, g \colon V \to V$ zwei diagonalisierbare Endomorphismen mit $f \circ g = g \circ f$.
      Zeigen Sie, dass auch $f \circ g$ diagonalisierbar ist.
    \item
      Bestimmen Sie alle $a, b \in \Real$, so dass die beiden Matrizen
      \[
        \begin{pmatrix}
          a & 1 \\
          0 & 3
        \end{pmatrix}
        \quad\text{und}\quad
        \begin{pmatrix*}[r]
          -1  & 2 \\
          0  & b
        \end{pmatrix*}
      \]
      simultan diagonalisierbar sind.
      
%     \item
%       Es seien $A, B \in \matrices{3}{\Real}$ mit
%       \[
%                   A
%         \coloneqq \begin{pmatrix}
%                     1  & 0  & 1 \\
%                     0  & 1  & 1 \\
%                     0  & 0  & 2
%                   \end{pmatrix}
%         \quad \text{und} \quad
%                   B
%         \coloneqq \begin{pmatrix*}[r]
%                     3 &  0  & -4  \\
%                     4 & -1  & -4  \\
%                     0 &  0  & -1
%                   \end{pmatrix*}.
%       \]
%       Bestimmen Sie $S \in \GL{3}{\Real}$, so dass $C^{-1} A C$ und $C^{-1} B C$ in Diagonalgestalt sind.
  \end{enumerate}
\end{question}





\begin{question}[subtitle = Symmetrische und schiefsymmetrische Matrizen]
  Es sei $K$ ein Körper mit $\ringchar{K} \neq 2$.
  Es sei
  \[
      \Sym{n}{K}
    = \{ A \in \matrices{n}{K} \suchthat \transpose{A} = A \}
  \]
  der Raum der symmetrischen Matrizen und
  \[
      \Alt{n}{K}
    = \{ A \in \matrices{n}{K} \suchthat \transpose{A} = -A\}
  \]
  der Raum der schiefsymmetrischen Matrizen.
  \begin{enumerate}
    \item
      Zeigen Sie, dass $\Sym{n}{K}$ und $\Alt{n}{K}$ Untervektorräume von $\matrices{n}{K}$ sind, und dass $\matrices{n}{K} = \Sym{n}{K} \oplus \Alt{n}{K}$ gilt.
      
      (\emph{Hinweis}:
       Für die Abbildung $f \colon \matrices{n}{K} \to \matrices{n}{K}$, $A \mapsto \transpose{A}$ gilt $f^2 = \id$.)
    \item
      Geben Sie Basen von $\Sym{n}{K}$ und $\Alt{n}{K}$ an.
  \end{enumerate}
\end{question}











% \begin{question}[subtitle = Links- und Rechtsshift]
%   Es sei $V = K^\Natural = \{(a_1, a_2, a_3, \dotsc) \suchthat a_i \in K\}$ der Vektorraum der $K$-wertigen Folgen.
%   Es sei
%   \[
%             R
%     \colon  V
%     \to     V,
%     \quad   (a_1, a_2, a_3, \dotsc)
%     \mapsto (0,   a_1, a_2, \dotsc)
%   \]
%   der Rechts-Shift-Operator, sowie
%   \[
%             L
%     \colon  V
%     \to     V,
%     \quad   (a_1, a_2, a_3, \dotsc)
%     \mapsto (a_2, a_3, a_4, \dotsc)
%   \]
%   der Links-Shift-Operator.
%   \begin{enumerate}
%     \item
%       Bestimmen Sie die Eigenwerte von $R$, sowie die zugehörigen Eigenräume.
%     \item
%       Bestimmen Sie die Eigenwerte von $L$, sowie die zugehörigen Eigenräume.
%   \end{enumerate}
% \end{question}





% \begin{question}
%   Es sei $V$ ein endlichdimensionaler $K$-Vektorraum und $f \colon V \to V$ ein Endomorphismus mit charakteristischen Polymom
%   \[
%       \charpol{f}(t)
%     = (t - \lambda_1)^{n_1} \dotsm (t - \lambda_s)^{n_s},
%   \]
%   wobei $\lambda_i \neq \lambda_j$ für $i \neq j$ gilt.
%   Welche der folgenden Bedingungen sind notwendig, und welche Bedingungen sind hinreichend dafür, dass $f$ diagonlisierbar ist?
%   \begin{enumerate}
%     \item
%       Es gilt $n_i = 1$ für alle $i$.
%     \item
%       Es gilt $\dim \eigenspace{V}{f}{\lambda_i} = n_i$ für alle $i$.
%     \item
%       Es gibt eine Basis $\basis{B}$ von $V$, so dass $\repmatrixendo{f}{\basis{B}}$ eine Diagonalmatrix ist.
%   \end{enumerate}
% \end{question}





\begin{question}
  Es sei $f \colon V \to V$ ein Endomorphismus.
  \begin{enumerate}
    \item
      Es sei $v \in V$ ein Eigenvektor von $f$ zum Eigenwert $\lambda \in K$.
      Zeigen Sie, dass $v$ für jedes Polynom $p \in K[t]$ ein Eigenvektor von $p(f)$ zum Eigenwert $p(\lambda)$ ist.
    \item
      Es sei $K$ algebraisch abgeschlossen und $p \in K[t]$.
      Zeigen Sie, dass es für jeden Eigenwert $\mu$ von $p(f)$ einen Eigenwert $\lambda$ von $f$ mit $\mu = p(\lambda)$ gibt.
      \\
      (\emph{Tipp}:
       Zeigen Sie zunächst, dass der Endomorphismus $(p - \lambda)(f)$ nicht injektiv ist.
       Zerlegen Sie anschließend $p - \lambda$ in Linearfaktoren.
  \end{enumerate}
\end{question}





\begin{question}[subtitle = Diagonalisieren über $\Field_5$]
  Es sei
  \[
              A
    \coloneqq \begin{pmatrix}
                3 & 4 & 1 \\
                0 & 1 & 2 \\
                1 & 2 & 2
              \end{pmatrix}
    \in       \matrices{3}{\Field_5}.
  \]
  Bestimmen Sie eine Matrix $S \in \GL{3}{\Field_5}$, so dass $S^{-1} A S$ in Diagonalform ist.
\end{question}





\end{document}
