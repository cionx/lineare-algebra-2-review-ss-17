\documentclass[a4paper, 10pt]{scrartcl}
\usepackage{../styles/generalstyle}
\usepackage{../styles/exercisestyle}

\subject{Repetitorium Lineare Algebra II, SS 2017}
\title{Tag 1}
\author{}
\date{}

\begin{document}





\begin{question}[subtitle = Eigenwerte und Diagonalisierbarkeit]
  \begin{enumerate}
    \item
      Es sei $A \in \GL{2}{\Complex}$ mit $\tr A = 0$.
      Zeigen Sie, dass $A$ diagonalisierbar ist.
    \item
      Zeigen Sie, dass jede Matrix $A \in \matrices{3}{\Real}$ einen reellen Eigenwert hat.
    \item
      Folgern Sie, dass jede nicht-triagonalisierbare Matrix $A \in \matrices{3}{\Real}$ über $\Complex$ diagonalisierbar ist.
      \\
      (\emph{Tipp}:
       Zeigen Sie, dass für jeden Eigenwert $\lambda$ von $A$ auch $\conjugate{\lambda}$ ein Eigenwert ist.)
    \item
      Es sei $A \in \matrices{n}{\Complex}$ und $k \geq 0$ mit $A^k = \Id$.
      Zeigen Sie, dass $A$ diagonalisierbar ist, und bestimmen Sie alle möglichen Eigenwerte für $A$.
    \item
      Es sei $A \in \matrices{2}{\Complex}$ mit $\tr A = 0$ und $\tr A^2 = -2$.
      Bestimmen Sie $\det A$.
      Entscheiden Sie auch, ob $A$ diagonalisierbar ist.
    \item
      Es sei $A \in \matrices{2}{\Complex}$ mit $\tr A = 2$ und $\tr A^2 = 4$.
      Zeigen Sie, dass $A$ diagonalisierbar ist, und bestimmen Sie die Eigenwerte von $A$.
    \item
      Es sei $A \in \matrices{n}{\Complex}$ mit $A^2 + A = 6 \Id$ und $\det A = 144$.
      Bestimmen Sie $n$.
    \item
      Es sei $A \in \matrices{n}{\Complex}$ mit $A^3 = 3A - 2 \Id$ und $A^3 + A^2 = A + \Id$.
      Zeigen Sie, dass $A = \Id$.
  \end{enumerate}
\end{question}





\begin{question}[subtitle = Determinante und Potenzen der Spur]
  Zeigen Sie für alle $A \in \matrices{3}{\Complex}$ die Gleichheit
  \[
      \det A
    =   \frac{1}{6} (\tr A)^3
      - \frac{1}{2} (\tr A^2)(\tr A)
      + \frac{1}{3} (\tr A^3).
  \]
\end{question}





\begin{question}[subtitle = Diagonalisieren]
  \begin{enumerate}
    \item
      Es sei
      \[
                  A
        \coloneqq \begin{pmatrix}
                    2       & 1       & \cdots  & 1       \phantom{\ddots}  \\
                    1       & \ddots  & \ddots  & \vdots  \phantom{\ddots}  \\
                    \vdots  & \ddots  & \ddots  & 1       \phantom{\ddots}  \\
                    1       & \cdots  & 1       & 2       \phantom{\ddots}
                  \end{pmatrix}
        \in       \matrices{n}{\Real}.
      \]
      Bestimmen Sie die Eigenwerte von $A$ und zeigen Sie, dass $A$ diagonalisierbar ist.
    \item
      Es sei
      \[
                  A
        \coloneqq \begin{pmatrix}
                    0     & \Id_n \\
                    \Id_n & 0
                  \end{pmatrix}
        \in       \matrices{2n}{\Real}.
      \]
      Geben Sie eine Basis von $K^{2n}$ aus Eigenvektoren von $A$ an.
      Bestimmen Sie anschließend $\charpol{A}(t)$ sowie $\det A$.
      
      (\emph{Tipp}: $A$ vertauscht die Basisvektoren $e_i$ und $e_{n+i}$.)
  \end{enumerate}
\end{question}





\begin{question}[subtitle = Wurzeln und Potenzen]
  Es seien
  \[
              A
    \coloneqq \begin{pmatrix*}[r]
                7 & -12 \\
                4 &  -7
              \end{pmatrix*}
              \in \matrices{2}{\Complex}
    \quad\text{und}\quad
              B
    \coloneqq \frac{1}{\sqrt{2}}
              \begin{pmatrix}
                1 & i \\
                i & 1
              \end{pmatrix}
              \in \matrices{2}{\Complex}.
  \]
  \begin{enumerate}
    \item
      Geben Sie eine Matrix $C \in \matrices{2}{\Complex}$ mit $A = C^2$ an.
    \item
      Berechnen Sie $B^{2017}$.
      \\
      (\emph{Tipp}:
       Ignorieren Sie ggf.\ zunächst den Vorfaktor $1/\sqrt{2}$.)
  \end{enumerate}
\end{question}





\begin{question}[subtitle = Cayley--Hamilton]
  Es sei $K$ ein Körper.
  \begin{enumerate}
    \item
      Zeigen Sie für $A \in \matrices{n}{K}$, dass die Potenzen $\Id, A, A^2, \dotsc, A^n$ linear abhängig sind.
    \item
      Es sei $A \in \GL{n}{K}$.
      Zeigen Sie, dass es ein Polynom $p \in K[t]$ mit $p(A) = A^{-1}$ gibt.
      Bestimmen Sie ein solches Polynom für die Matrix
      \[
                  A
        \coloneqq \begin{pmatrix}
                    1 & 0 & 1 \\
                    1 & 1 & 1 \\
                    0 & 0 & 1
                  \end{pmatrix}
        \in \GL{3}{\Real}.
      \]
  \end{enumerate}
\end{question}





\begin{solution}
  \begin{enumerate}
    \item
      Für das charakteristische Polynom $\charpol{A}(t) = (-1)^n t^n + a_{n-1} t^{n-1} + \dotsb + a_1 t + a_0$ gilt nach dem Satz von Cayley--Hamilton, dass
      \[
          0
        = \charpol{A}(t)
        = (-1)^n A^n + a_{n-1} A^{n-1} + \dotsb + a_1 A + a_0 \Id.
      \]
      Wir haben somit eine nicht-trivale Linearkombination von $0$ durch die Matrizen $A^n, A^{n-1}, \dotsc, \Id$.
    \item
      Für das charakterische Polynom $\charpol{A}(t) = (-1)^n t^n + a_{n-1} t^{n-1} + \dotsb + a_1 t + a_0$ gilt $a_0 = \det A$.
      Nach dem Satz von Cayley--Hamilton gilt somit, dass
      \[
          0
        = \charpol{A}(t)
        = (-1)^n A^n + a_{n-1} A^{n-1} + \dotsb + a_1 A + (\det A) \Id.
      \]
      Durch Umstellen dieser Gleichung ergibt sich, dass
      \begin{align*}
            \Id
        &=  - \frac{(-1)^n A^n + a_{n-1} A^{n-1} + \dotsb + a_1 A}{\det A} \\
        &=  A \cdot \left( - \frac{(-1)^n A^{n-1} + a_{n-1} A^{n-2} + \dotsb + a_1}{\det A} \right).
      \end{align*}
      Das Inverse $A^{-1}$ ist also durch
      \[
          A^{-1}
        = - \frac{(-1)^n A^{n-1} + a_{n-1} A^{n-2} + \dotsb + a_1}{\det A}
        = p(A)
      \]
      für das Polynom
      \[
                  p(t)
        \coloneqq - \frac{(-1)^n t^{n-1} + a_{n-1} t^{n-2} + \dotsb + a_1}{\det A}
        =         - \frac{ \charpol{A}(t) - \det A}{(\det A) t}
        =         \frac{\det A - \charpol{A}(t)}{(\det A) t}
      \]
      gegeben.
      
      Für die gegebene Matrix $A \in \GL{3}{\Real}$ gilt
      \[
          \charpol{A}(t)
        = -t^3 + 3 t^2 - 3 t + 1.
      \]
      Hieraus ergibt sich das Polynom $p(t) \coloneqq t^2 - 3 t + 3$ mit $A^{-1} = p(A)$.
  \end{enumerate}
\end{solution}





\begin{question}[subtitle = Simultane Diagonalisierbarkeit]
  \begin{enumerate}
    \item
      Es sei $V$ ein endlichdimensionaler $K$-Vektorraum, und es seien $f, g \colon V \to V$ zwei diagonalisierbare Endomorphismen mit $f \circ g = g \circ f$.
      Zeigen Sie, dass auch $f \circ g$ diagonalisierbar ist.
    \item
      Bestimmen Sie alle $a, b \in \Real$, so dass die beiden reellen Matrizen
      \[
        \begin{pmatrix}
          a & 1 \\
          0 & 3
        \end{pmatrix}
        \quad\text{und}\quad
        \begin{pmatrix*}[r]
          -1  & 2 \\
           0  & b
        \end{pmatrix*}
      \]
      simultan diagonalisierbar sind.
      
%     \item
%       Es seien $A, B \in \matrices{3}{\Real}$ mit
%       \[
%                   A
%         \coloneqq \begin{pmatrix}
%                     1  & 0  & 1 \\
%                     0  & 1  & 1 \\
%                     0  & 0  & 2
%                   \end{pmatrix}
%         \quad \text{und} \quad
%                   B
%         \coloneqq \begin{pmatrix*}[r]
%                     3 &  0  & -4  \\
%                     4 & -1  & -4  \\
%                     0 &  0  & -1
%                   \end{pmatrix*}.
%       \]
%       Bestimmen Sie $S \in \GL{3}{\Real}$, so dass $C^{-1} A C$ und $C^{-1} B C$ in Diagonalgestalt sind.
  \end{enumerate}
\end{question}





\begin{solution}
  \begin{enumerate}
    \item
      Die Endomorphismen $f$ und $g$ sind simultan diagonalisierbar, da sie jeweils einzeln diagonalisierbar sind, und kommutieren.
      Es gibt also eine Basis $\basis{B}$ von $V$, bezüglich der die Endomorphismen $f$ und $g$ durch die Matrizen
      \[
          \repmatrixendo{f}{\basis{B}}
        = \begin{pmatrix}
            \lambda_1 &         &           \\
                      & \ddots  &           \\
                      &         & \lambda_n
          \end{pmatrix}
        \quad\text{und}\quad
          \repmatrixendo{g}{\basis{B}}
        = \begin{pmatrix}
            \mu_1 &         &       \\
                  & \ddots  &       \\
                  &         & \mu_n
          \end{pmatrix}
      \]
      dargestellt werden.
      Bezüglich $\basis{B}$ wird $f \circ g$ durch die Matrix
      \begin{align*}
            \repmatrixendo{f \circ g}{\basis{B}}
         =  \repmatrixendo{f}{\basis{B}} \repmatrixendo{g}{\basis{B}}
        &=  \begin{pmatrix}
              \lambda_1 &         &           \\
                        & \ddots  &           \\
                        &         & \lambda_n
            \end{pmatrix}
            \begin{pmatrix}
              \mu_1 &         &       \\
                    & \ddots  &       \\
                    &         & \mu_n
            \end{pmatrix}
        \\
        &=  \begin{pmatrix}
              \lambda_1 \mu_1 &         &                 \\
                              & \ddots  &                 \\
                              &         & \lambda_n \mu_n
            \end{pmatrix}
      \end{align*}
      dargestellt.
      Also ist $\basis{B}$ eine Basis von $V$ aus Eigenvektoren von $f \circ g$, und $f \circ g$ somit diagonalisierbar.
      
    \item
      Wir bezeichnen die beiden Matrizen mit
      \[
          A_a
        = \begin{pmatrix}
            a & 1 \\
            0 & 3
          \end{pmatrix}
        \quad\text{und}\quad
          B_b
        = \begin{pmatrix*}[r]
            -1  & 2 \\
             0  & b
          \end{pmatrix*}.
      \]
      Die Matrizen $A_a$ und $B_b$ sind genau dann simultan diagonalisierbar, wenn sie jeweils einzeln diagonalisierbar sind, und sie kommutieren.
      \begin{itemize}
        \item
          Die Matrix $A_a$ ist eine obere Dreiecksmatrix, ihr charakteristisches Polynom ist also
          \[
              \charpol{A_a}(t)
            = (t-a)(t-3).
          \]
          Ist $a \neq 3$, so zerfällt $A_a$ in paarweise verschiedene Linearfaktoren, weshalb $A_a$ in diesen Fällen diagonalisierbar ist.
          Für $a = 3$ ist $3$ der einzige Eigenwert von $A_a = A_3$;
          die Matrix $A_3$ ist nicht diagonalisierbar, da der Eigenraum
          \[
              \eigenspace{(\Real^2)}{A_3}{3}
            = \ker (A_3 - 3 \Id)
            = \ker \begin{pmatrix}
                     0 & 1  \\
                     0 & 0
                   \end{pmatrix}
            = \generated{ \vect{1 \\ 0} }
          \]
          nur eindimensional ist.
          
          Also ist $A_a$ genau dann diagonalisierbar, wenn $a \neq 3$ gilt.
        \item
          Analog ergibt sich, dass die Matrix $B_b$ genau dann diagonalisierbar ist, wenn $b \neq -1$ gilt.
        \item
          Es gelten
          \begin{gather*}
              A_a B_b
            = \begin{pmatrix}
                a & 1 \\
                0 & 3
              \end{pmatrix}
              \begin{pmatrix*}[r]
                -1  & 2 \\
                 0  & b
              \end{pmatrix*}
            = \begin{pmatrix}
                          - a & 2a + b  \\
                 \phantom{-}0 & 3b
              \end{pmatrix}
          \shortintertext{und}
              B_b A_a
            = \begin{pmatrix*}[r]
                -1  & 2 \\
                 0  & b
              \end{pmatrix*}
              \begin{pmatrix}
                a & 1 \\
                0 & 3
              \end{pmatrix}
            = \begin{pmatrix}
                          - a & 5   \\
                 \phantom{-}0 & 3b
              \end{pmatrix}
          \end{gather*}
          Die Matrizen $A_a$ und $B_b$ kommutieren also genau dann, wenn $2 a + b = 5$ gilt, wenn also $b = 5 - 2 a$ gilt.
      \end{itemize}
    Wir erhalten also ingesamt, dass die Matrizen $A_a$ und $B_b$ genau dann simultan diagonalisierbar sind, wenn $b = 5 - 2 a$ und $b \neq -1$ gelten (der Fall $a = 3$ entspricht dem Fall $b = -1$).
  \end{enumerate}
\end{solution}





\begin{question}[subtitle = Symmetrische und schiefsymmetrische Matrizen]
  Es sei $K$ ein Körper mit $\ringchar{K} \neq 2$.
  Es sei
  \[
      \Sym{n}{K}
    = \{ A \in \matrices{n}{K} \suchthat \transpose{A} = A \}
  \]
  der Raum der symmetrischen Matrizen und
  \[
      \Alt{n}{K}
    = \{ A \in \matrices{n}{K} \suchthat \transpose{A} = -A\}
  \]
  der Raum der schiefsymmetrischen Matrizen.
  \begin{enumerate}
    \item
      Zeigen Sie, dass $\Sym{n}{K}$ und $\Alt{n}{K}$ Untervektorräume von $\matrices{n}{K}$ sind, und dass $\matrices{n}{K} = \Sym{n}{K} \oplus \Alt{n}{K}$ gilt.
      
      (\emph{Hinweis}:
       Für die Abbildung $f \colon \matrices{n}{K} \to \matrices{n}{K}$, $A \mapsto \transpose{A}$ gilt $f^2 = \id$.)
    \item
      Geben Sie Basen von $\Sym{n}{K}$ und $\Alt{n}{K}$ an.
  \end{enumerate}
\end{question}





\begin{solution}
  \begin{enumerate}
    \item
      Es gilt $\transpose{(\transpose{A})} = A$ für alle $A \in \matrices{n}{K}$, und somit $f^2 = \id_{\matrices{n}{K}}$ für die Abbildung $f \colon \matrices{n}{K} \to \matrices{n}{K}$, $A \mapsto A^T$.
      Also gilt $q(f) = 0$ für das Polynom $q(t) \coloneqq t^2 - 1 \in K[t]$.
      
      Das Polynom $q$ zerfällt in Linearfaktoren $q(t) = (t - 1)(t + 1)$, und da $\ringchar K \neq 2$ gilt, sind die beiden Linearfaktoren verschieden.
      Da $\minpol{f} \divides q$ gilt, folgt damit, dass $\minpol{f}$ in die beiden möglichen Linearfaktoren $t-1$ und $t+1$ zerfällt.
      Somit ist $f$ diagonalisierbar mit möglichen Eigenwerten $1$ und $-1$.
      
      Es gilt also
      \[
          \matrices{n}{K}
        = \eigenspace{\matrices{n}{K}}{f}{1} \oplus \eigenspace{\matrices{n}{K}}{f}{-1}.
      \]
      Dabei gelten
      \begin{gather*}
          \eigenspace{\matrices{n}{K}}{f}{1}
        = \{ A \in \matrices{n}{K} \suchthat \transpose{A} = A \}
        = \Sym{n}{K}
      \shortintertext{sowie}
          \eigenspace{\matrices{n}{K}}{f}{1}
        = \{ A \in \matrices{n}{K} \suchthat \transpose{A} = -A \}
        = \Alt{n}{K}.
      \end{gather*}
      
    \item
      Für die Standardbasis $(E_{ij})_{i,j = 1, \dotsc, n}$ von $\matrices{n}{K}$ gilt $f(E_{ij}) = \transpose{E_{ij}} = E_{ji}$.
      \begin{itemize}
        \item
          Die Basisvektoren $E_{ii}$ sind also bereits Eigenvektoren von $f$ zum Eigenwert $1$.
        \item
          Für $i \neq j$ werden die Basisvektoren $E_{ij}$ und $E_{ji}$ von $f$ miteinander vertauscht.
          Man kann deshalb $E_{ij}$ und $E_{ji}$ durch die beiden Vektoren
          \[
            E_{ij} + E_{ji}
            \quad\text{und}\quad
            E_{ij} - E_{ji}
          \]
          ersetzen, wobei $E_{ij} + E_{ji}$ ein Eigenvektor von $f$ zum Eigenwert $1$ ist, und $E_{ij} - E_{ji}$ ein Eigenvektor zum Eigenwert $-1$.
      \end{itemize}
      Somit erhält man insgesamt die folgende Basis von $\matrices{n}{K}$, bestehend aus Eigenvektoren von $f$:
      \[
              \{ E_{ii} \suchthat i = 1, \dotsc, n \}
        \cup  \{ E_{ij} + E_{ji} \suchthat 1 \leq i < j \leq n \}
        \cup  \{ E_{ij} - E_{ji} \suchthat 1 \leq i < j \leq n \}.
      \]
      Für $\Sym{n}{K} = \eigenspace{\matrices{n}{K}}{f}{1}$ ergibt sich somit die Basis
      \[
              \{ E_{ii} \suchthat i = 1, \dotsc, n \}
        \cup  \{ E_{ij} + E_{ji} \suchthat 1 \leq i < j \leq n \},
      \]
      und für $\Alt{n}{K} = \eigenspace{\matrices{n}{K}}{f}{-1}$ die Basis
      \[
        \{ E_{ij} - E_{ji} \suchthat 1 \leq i < j \leq n \}.
      \]
  \end{enumerate}
\end{solution}





% \begin{question}[subtitle = Links- und Rechtsshift]
%   Es sei $V = K^\Natural = \{(a_1, a_2, a_3, \dotsc) \suchthat a_i \in K\}$ der Vektorraum der $K$-wertigen Folgen.
%   Es sei
%   \[
%             R
%     \colon  V
%     \to     V,
%     \quad   (a_1, a_2, a_3, \dotsc)
%     \mapsto (0,   a_1, a_2, \dotsc)
%   \]
%   der Rechts-Shift-Operator, sowie
%   \[
%             L
%     \colon  V
%     \to     V,
%     \quad   (a_1, a_2, a_3, \dotsc)
%     \mapsto (a_2, a_3, a_4, \dotsc)
%   \]
%   der Links-Shift-Operator.
%   \begin{enumerate}
%     \item
%       Bestimmen Sie die Eigenwerte von $R$, sowie die zugehörigen Eigenräume.
%     \item
%       Bestimmen Sie die Eigenwerte von $L$, sowie die zugehörigen Eigenräume.
%   \end{enumerate}
% \end{question}





% \begin{question}
%   Es sei $V$ ein endlichdimensionaler $K$-Vektorraum und $f \colon V \to V$ ein Endomorphismus mit charakteristischen Polymom
%   \[
%       \charpol{f}(t)
%     = (t - \lambda_1)^{n_1} \dotsm (t - \lambda_s)^{n_s},
%   \]
%   wobei $\lambda_i \neq \lambda_j$ für $i \neq j$ gilt.
%   Welche der folgenden Bedingungen sind notwendig, und welche Bedingungen sind hinreichend dafür, dass $f$ diagonlisierbar ist?
%   \begin{enumerate}
%     \item
%       Es gilt $n_i = 1$ für alle $i$.
%     \item
%       Es gilt $\dim \eigenspace{V}{f}{\lambda_i} = n_i$ für alle $i$.
%     \item
%       Es gibt eine Basis $\basis{B}$ von $V$, so dass $\repmatrixendo{f}{\basis{B}}$ eine Diagonalmatrix ist.
%   \end{enumerate}
% \end{question}





\begin{question}
  Es sei $f \colon V \to V$ ein Endomorphismus.
  \begin{enumerate}
    \item
      Es sei $v \in V$ ein Eigenvektor von $f$ zum Eigenwert $\lambda \in K$.
      Zeigen Sie, dass $v$ für jedes Polynom $p \in K[t]$ ein Eigenvektor von $p(f)$ zum Eigenwert $p(\lambda)$ ist.
    \item
      Es sei $K$ algebraisch abgeschlossen und $p \in K[t]$.
      Zeigen Sie, dass es für jeden Eigenwert $\mu$ von $p(f)$ einen Eigenwert $\lambda$ von $f$ mit $\mu = p(\lambda)$ gibt.
      \\
      (\emph{Tipp}:
       Zeigen Sie zunächst, dass der Endomorphismus $(p - \lambda)(f)$ nicht injektiv ist.
       Zerlegen Sie anschließend $p - \lambda$ in Linearfaktoren.
  \end{enumerate}
\end{question}





\begin{solution}
  \begin{enumerate}
    \item
      Nach Annahme gilt $f(v) = \lambda v$.
      Induktiv ergibt sich damit für alle $k \geq 0$, dass $f^k(v) = \lambda^k v$ gilt.
      Für ein beliebiges Polynom
      \[
          p(t)
        = a_n t^n + a_{n-1} t^{n-1} + \dotsb + a_1 t + a_0 \in K[t]
      \]
      ergibt sich damit, dass
      \begin{align*}
            p(f)(v)
        &=  (a_n f^n + a_{n-1} f^{n-1} + \dotsb + a_1 f + a_0 \id_V)(v)                 \\
        &=  a_n f^n(v) + a_{n-1} f^{n-1}(v) + \dotsb + a_1 f(v) + a_0 \id_V(v)          \\
        &=  a_n \lambda^n v + a_{n-1} \lambda^{n-1} v + \dotsb + a_1 \lambda v + a_0 v  \\
        &=  (a_n \lambda^n + a_{n-1} \lambda^{n-1} + \dotsb + a_1 \lambda + a_0) v      \\
        &=  p(\lambda) v.
      \end{align*}
      Da nach Annahme auch $v \neq 0$ gilt, ist $v$ somit ein Eigenvektor von $p(f)$ zum Eigenwert $p(\lambda)$.
      
    \item
      Die Abbildung $p(f) - \mu \id_V$ ist nicht injektiv, da $\mu$ ein Eigenwert von $p(f)$ ist.
      Für das Polynom $q(t) \coloneqq p(t) - \mu$ ist also die Abbildung $q(f)$ nicht injektiv.
      
      Da $K$ algebraisch abgeschlossen ist, zerfällt $q(t)$ in Linearfaktoren
      \[
        q(t) = (t - \lambda_1) \dotsm (t - \lambda_n).
      \]
      Da die Komposition
      \[
          q(f)
        = (f - \lambda_1 \id_V) \circ \dotsb \circ (f - \lambda_n \id_V)
      \]
      nicht injektiv ist, muss bereits eine der Abbildungen $f - \lambda_i \id_V$ nicht injektiv sein (denn die Komposition injektiver Abbildungen ist ebenfalls wieder injektiv).
      Für ein entsprechendes $i$ ist dann $\lambda \coloneqq \lambda_i$ ein Eigenwert von $f$, da $f - \lambda \id_V$ nicht injektiv ist.
      
      Da $\lambda$ eine Nullstelle von $q(t) = p(t) - \mu$ ist, gilt dabei $0 = q(\lambda) = p(\lambda) - \mu$, und somit $\mu = p(\lambda)$.
  \end{enumerate}
\end{solution}





\begin{question}[subtitle = Diagonalisieren über $\Field_5$]
  Es sei
  \[
              A
    \coloneqq \begin{pmatrix}
                3 & 4 & 1 \\
                0 & 1 & 2 \\
                1 & 2 & 2
              \end{pmatrix}
    \in       \matrices{3}{\Field_5}.
  \]
  Bestimmen Sie eine Matrix $S \in \GL{3}{\Field_5}$, so dass $S^{-1} A S$ in Diagonalform ist.
\end{question}





\begin{solution}
  Es gilt
  \[
      \charpol{A}(t)
    = - t^3 + t^2 + 4 t + 1
  \]
  Durch Ausprobieren ergibt sich für $\charpol{A}(t)$ die Nullstelle $\lambda_1 = 1$.
  Durch Abspalten eines entsprechendes Linearfaktors ergibt sich, dass
  \begin{align*}
        - t^3 + t^2 + 4 t + 1
    &=  - (t^3 - t^2 + t - 1)
     =  - (t - 1)(t^2 + 1)
     =  - (t - 1)(t^2 - 4)
    \\
    &=  - (t - 1)(t - 2)(t + 2)
     =  - (t - 1)(t - 2)(t - 3).
  \end{align*}
  Die Eigenwerte von $A$ sind also $1$, $2$ und $3$.
  (Da $\charpol{A}(t)$ in paarweise verschiedene Linearfaktoren zerfällt, erkennt man bereits, dass $A$ diagonalisierbar ist.)
  Die jeweils zugehörigen Eigenräume sind
  \begin{gather*}
      \eigenspace{(\Field_5)}{A}{1}
    = \ker  (A - \Id)
    = \ker  \begin{pmatrix}
              2 & 4 & 1 \\
              0 & 0 & 2 \\
              1 & 2 & 1
            \end{pmatrix}
    = \generated{ \rvect{2 \\ -1 \\ 0} }
    = \generated{ \vect{2 \\ 4 \\ 0} }
    = \generated{ \vect{1 \\ 2 \\ 0} },
  \\
      \eigenspace{(\Field_5)}{A}{2}
    = \ker  (A - 2 \Id)
    = \ker  \begin{pmatrix}
              1 & 4 & 1 \\
              0 & 4 & 2 \\
              1 & 2 & 0
            \end{pmatrix}
    = \generated{ \rvect{-2 \\ 1 \\ -2} }
    = \generated{ \vect{3 \\ 1 \\ 3} }
    = \generated{ \vect{1 \\ 2 \\ 1} },
  \shortintertext{und}
    \begin{aligned}
          \eigenspace{(\Field_5)}{A}{3}
       =  \ker  (A - 3 \Id)
       =  \ker  \begin{pmatrix}
                  0 & 4 & 1 \\
                  0 & 3 & 2 \\
                  1 & 2 & 4
                \end{pmatrix}
      &=  \ker  \begin{pmatrix*}[r]
                  0 & -1  & 1 \\
                  0 & -2  & 2 \\
                  1 &  2  & 4
                \end{pmatrix*}
      \\
      &=  \generated{ \rvect{-3 \\ 1 \\ -1} }
       =  \generated{ \vect{2 \\ 1 \\ 4} }
       =  \generated{ \vect{1 \\ 3 \\ 2} }.
    \end{aligned}
  \end{gather*}
  Für die Matrix
  \[
              S
    \coloneqq \begin{pmatrix}
                1 & 1 & 1 \\
                2 & 2 & 3 \\
                0 & 1 & 2 \\
              \end{pmatrix}
    \in \GL{3}{\Field_5}
  \]
  gilt also
  \[
      S^{-1} A S
    = \begin{pmatrix}
        1 &   &   \\
          & 2 &   \\
          &   & 3
      \end{pmatrix}.
  \]
\end{solution}





\pagebreak
\section*{Lösungen}



\printsolutions




\end{document}
